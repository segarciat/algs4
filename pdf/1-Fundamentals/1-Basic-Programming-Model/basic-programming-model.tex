\documentclass[12pt, a4paper]{article}

\usepackage[utf8]{inputenc}
% Limit the page margin to only 1 inch.
\usepackage[margin=1in]{geometry}

%Imports biblatex package
\usepackage[
backend=biber,
style=alphabetic
]{biblatex}
\addbibresource{../../algs4e.bib}

\usepackage{amsmath}
\usepackage{amssymb}

% Provides useful environments, such as:
% - \begin{proof} ...\end{proof}
\usepackage{amsthm}

\usepackage[inline]{enumitem} % allows using letters in enumerate list environment

% Enable importing external graphic files and provides useful commannds, like \graphicspath{}
\usepackage{graphicx}
% Images are located in a directory called images in the current directory.
\graphicspath{{./images/}}

% Code Listings. Source:
% https://stackoverflow.com/questions/3175105/inserting-code-in-this-latex-document-with-indentation
\usepackage{listings}
\usepackage{color}

\definecolor{dkgreen}{rgb}{0,0.6,0}
\definecolor{gray}{rgb}{0.5,0.5,0.5}
\definecolor{mauve}{rgb}{0.58,0,0.82}

\lstset{frame=tb,
	language=Java,
	aboveskip=3mm,
	belowskip=3mm,
	showstringspaces=false,
	columns=flexible,
	basicstyle={\small\ttfamily},
	numbers=none,
	numberstyle=\tiny\color{gray},
	keywordstyle=\color{blue},
	commentstyle=\color{dkgreen},
	stringstyle=\color{mauve},
	breaklines=true,
	breakatwhitespace=true,
	tabsize=3
}

\newcommand{\prob}{\text{P}}
%\newcommand{\complement}{\mathsf{c}}

% Define an environment called "ex" (for Exercise) so that I can do: \begin{ex}{1.5}...\end{ex}
\newenvironment{ex}[2][Exercise]
{\par\medskip\noindent \textbf{#1 #2.}}
{\medskip}

% Define a solution environment, similar to ex (exercise) environment.
\newenvironment{sol}[1][Solution]
{\par\medskip\noindent \textbf{#1.} }
{\medskip}

\begin{document}
	\noindent Sergio E. Garcia Tapia \hfill
	
	\noindent \emph{Algorithms} by Sedgewick and Wayne (4th edition) \cite{sedgewick_wayne}\hfill
	
	\noindent September 8th, 2024\hfill 
	\section*{1.1: Basic Programming Model}
	\begin{ex}{1}
		Give the value of each of the following expressions:
		\begin{enumerate}[label=(\alph*)]
			\item \texttt{( 0 + 15) / 2}
			\item \texttt{2.0e-6 * 100000000.1}
			\item \texttt{true \&\& false || true \&\& true}
		\end{enumerate}
	\end{ex}
	\begin{sol}
		\begin{enumerate}[label=(\alph*)]
			\item \texttt{7} because integer division uses truncation.
			\item \texttt{2.000000002E-6}
			\item \texttt{true}
		\end{enumerate}
	\end{sol}
	\begin{ex}{2}
		Give the type and value of each of the following expressions:
		\begin{enumerate}[label=(\alph*)]
			\item \texttt{(1 + 2.236) / 2}
			\item \texttt{1 + 2 + 3 + 4.0}
			\item \texttt{4.1 >= 4}
			\item \texttt{1 + 2 + "3"}
		\end{enumerate}
	\end{ex}
	\begin{sol}
		\begin{enumerate}[label=(\alph*)]
			\item \texttt{double} with value  \texttt{1.6.18}
			\item \texttt{double} with value \texttt{10.0}
			\item \texttt{boolean} with value \texttt{true}
			\item \texttt{String} with value \texttt{"33"}
		\end{enumerate}
	\end{sol}
	\begin{ex}{3}
		Write a program that takes three integer command-line arguments and prints
		\texttt{equal} if all three are equal, and \texttt{not equal} otherwise.
	\end{ex}
	\begin{sol}
		See \texttt{com.segarciat.algs4.ch1.sec.ex13.Compare3Integers}.
		The command-line argument are \texttt{String} objects, so they can be converted
		to integers with \texttt{Integer.parseInt()}. Assuming the three integers are
		\texttt{a}, \texttt{b}, and \texttt{c}, we now just verify the value of the
		boolean expression \texttt{a == b \&\& b == c}.
	\end{sol}
	\begin{ex}{4}
		What (if anything) is wrong with each of the following statements?
		\begin{enumerate}[label=(\alph*)]
			\item \texttt{if (a > b) then c = 0;}
			\item \texttt{if a > b \{ c = 0; \}}
			\item \texttt{if (a > b) c = 0;}
			\item \texttt{if (a > b) c = 0 else b = 0;}
		\end{enumerate}
	\end{ex}
	\begin{sol}
		\
		\begin{enumerate}[label=(\alph*)]
			\item \texttt{then} is not a valid Java keyword. If we remove it, the code snippet
			will be valid.
			\item We need parentheses around the boolean condition of the \texttt{if} statement,
			in this case, around \texttt{a > b}. If we add this, the snippet will be valid.
			\item The snippet is valid.
			\item We need a semicolon to terminate the assignment statement \texttt{c = 0}.
			If we add this, the snippet will be valid.
		\end{enumerate}
	\end{sol}
	\begin{ex}{5}
		Write a code  snippet that prints \texttt{true} if the \texttt{double} variables
		\texttt{x} and \texttt{y} are both strictly between \texttt{0} and \texttt{1}
		and \texttt{false} otherwise.
	\end{ex}
	\begin{sol}
		\
		\begin{lstlisting}
	System.out.println(x > 0 && x < 1 && y > 0 && y < 1);
		\end{lstlisting}
	\end{sol}
	\begin{ex}{6}
		What does the following program print?
		\begin{lstlisting}
int f = 0;
int g = 1;
for (int i = 0; i <= 15; i++)
{
	StdOut.println(f);
	f = f + g;
	g = f - g;
}
		\end{lstlisting}
	\end{ex}
	\begin{sol}
		Note we are given:
		\begin{align*}
			f_0 &= 0,\\
			g_0 &= 1,\\
			f_{n} &= f_{n-1} + g_{n-1},\quad n\geq 1,\\
			g_{n} &= f_{n} - g_{n-1},\quad n\geq 1.
		\end{align*}
		Notice the recurrence of $g_n$ follows because the value just computed in the current
		iteration of the \texttt{for} loop is  used to compute the new value for \texttt{g}.
		Note that$f_0 = 0$, $f_1 = f_0 + g_0 = 0 + 1 = 1$, and
		\begin{align*}
		f_{n + 1} &= f_n + g_n\\
		&= (f_{n-1} + g_{n-1}) + g_n\\
		&=f_{n-1}  + (g_{n-1} +  g_n)\\
		&=f_{n-1} + f_n
		\end{align*}
		Altogether, we have:
		\begin{align*}
			f_0 &= 0, \\
			f_1 &= 1, \\
			f_{n + 1} &= f_{n-1} + f_n,\quad n \geq 1.
		\end{align*}
		Hence $n\mapsto f_n$ is the Fibonacci sequence. The program will print the first 15
		Fibonacci numbers:
		\begin{lstlisting}[language={}]
1
1
2
3
5
8
13
21
34
55
89
144
233
377
610
		\end{lstlisting}
	\end{sol}
	\begin{ex}{7}
		Give the value printed by each of the following code fragments.
		\begin{enumerate}[label=(\alph*)]
			\item 
			\begin{lstlisting}
double t = 9.0;
while (Math.abs(t - 9.0/t) > 0.001)
	t = (9.0/t + t) / 2.0;
StdOut.printf("%.5f\n", t);
			\end{lstlisting}
			\item
			\begin{lstlisting}
int sum = 0;
for (int i = 1; i < 1000; i++)
	for (int j = 0; j < i; j++)
		sum++;
StdOut.println(sum);
			\end{lstlisting}
			\item
			\begin{lstlisting}
int sum = 0;
for (int i = 1; i < 1000; i *= 2)
	for (int j = 0; j < 1000; j++)
		sum++;
StdOut.println(sum);
			\end{lstlisting}
		\end{enumerate}
	\end{ex}
	\begin{sol}
		\
		\begin{enumerate}[label=(\alph*)]
			\item The iterations are computed as:
			\begin{align*}
				t_0 &= 9.0 \quad; \quad t_0 - \frac{9.0}{t_0} = 1 \quad ; \quad  \to \quad
				t_1 = \frac{9.0/t_0 + t_0}{2.0} = 5.0\\
				t_1 &= 5.0 \quad; \quad t_1 - \frac{9.0}{t_1} = 3.2 \quad ; \quad \to \quad
				t_2 = \frac{9.0/t_1 +t_1}{2.0} = 3.4\\
				t_2 &= 3.4 \quad; \quad t_2 - \frac{9.0}{t_2} \approx 0.75294 \quad ; \quad \to \quad
				t_3 = \frac{9.0/t_2 +t_2}{2.0} = 3.023529411764706 \\
				t_3 &\approx 3.02352 \quad; \quad t_3 - \frac{9.0}{t_3} \approx 0.04687 \quad ; \quad \to \quad
				t_4 \approx \frac{9.0/t_3 +t_3}{2.0} = 3.00009155413138 \\
				t_4 &= 3.00009155413138 \quad ; \quad t_4 - \frac{9.0}{t_4} = 0.00018310546879263256
			\end{align*}
			The iteration ends once $t_4$ has been computed because it's below the threshold
			of \texttt{0.004} that controls the \texttt{while} loop. Since the format
			specifier requires 5 places after the decimal, the output will be:
			\begin{lstlisting}[language={}]
3.00009
			\end{lstlisting}
			\item The outer \texttt{i} loop runs 999 times. The inner \texttt{j} loop
			runs \texttt{i} times, and each time, it increment \texttt{sum} by 1.
			The value of \texttt{sum} is given by:
			\begin{align*}
				\sum_{i = 1}^{999} \sum_{j = 0}^{i - 1} =
				\sum_{i = 1}^{999} i =
				\frac{999 \cdot (999 + 1)}{2} =
				499500
			\end{align*}
			Therefore the out will be:
			\begin{lstlisting}
499500
			\end{lstlisting}
			\item In this case, we start with $1$ and double \texttt{i} each  time.
			When \texttt{i} reaches $2^{10} =1024$, the \texttt{i} loops will end.
			Hence, the loop will run for $i=2^0, i=2^1,\ldots,i=2^{9}$. Since the
			\texttt{j} loops increases \texttt{sum} a total of \texttt{i} times,
			we find that \texttt{sum} will now be:
			\begin{align*}
				\sum_{k}\sum_{i = 1}^{1000}[i = 2^k] \sum_{j=0}^{i-1}
				&= \sum_{k}\sum_{1\leq i\leq 1000}[i = 2^k]\cdot i\\
				&=\sum_{1\leq 2^k\leq 1000}2^k\\
				&=\sum_{0\leq k\leq 9}2^k\\
				&=2^{10} - 1\\
				&=1023
			\end{align*}
			Therefore the output will be:
			\begin{lstlisting}
1023
			\end{lstlisting}
		\end{enumerate}
	\end{sol}
	\begin{ex}{8}
		What do each of the following print?
		\begin{enumerate}[label=(\alph*)]
			\item \texttt{System.out.println(`b');}
			\item \texttt{System.out.println(`b' + `c');}
			\item \texttt{System.out.println((char) (`a' + 4));}
		\end{enumerate}
	\end{ex}
	\begin{sol}
		\
		\begin{enumerate}[label=(\alph*)]
			\item Java will display the \texttt{char} as the corresponding symbol:
			\begin{lstlisting}[language={}]
b
			\end{lstlisting}
			\item When Java adds two \texttt{char} values, it will promote the result to
			an \texttt{int}. In Java, \texttt{char} values are 16-bit Unicode characters.
			Since\texttt{`b'} has decimal value 98 in Unicode, and \texttt{`c'} has decimal
			value \texttt{99}, the result is:
			\begin{lstlisting}[language={}]
197
			\end{lstlisting}
			\item The \texttt{char} value \texttt{`a'} has decimal value \texttt{97},
			so when it is added to \texttt{4}, it becomes integer value \texttt{101}.
			The effect of \texttt{(char)} is to cast the result back to a \texttt{char}.
			The integer \texttt{101} fits into a \texttt{char}, and it corresponds to 
			\texttt{`e'}:
			\begin{lstlisting}[language={}]
e
			\end{lstlisting}
		\end{enumerate}
	\end{sol}
	\begin{ex}{9}
		Write a code fragment that puts the binary representation of a  positive integer $n$
		into a \texttt{String s}.
	\end{ex}
	\begin{sol}
		\
		If we divide \texttt{n} by 2, then the remainder of the division is the least significant
		bit in the binary representation of \texttt{n}. If we were to divide by the resulting
		quotient by \texttt{2}, then the remainder of that division is the next most significant bit.
		Continuing this way, the value of \texttt{n} falls to \texttt{0} as we continue to
		divide by 2. The solution is actually given in \cite{sedgewick_wayne}:
		\begin{lstlisting}
String s = "";
for (int k = n; n > 0; n /= 2)
	s = (k % 2) + s;
		\end{lstlisting}
		We could extend this to handle \texttt{0} by changing it to a
		\texttt{do \{/*...*/\} while(/*...*/);}.loop.
	\end{sol}
	\begin{ex}{10}
		What is wrong with the following code fragment?
		\begin{lstlisting}
int[] a;
for (int i = 0; i < 10; i++)
	a[i] = i * i;
		\end{lstlisting}
	\end{ex}
	\begin{sol}
		It fails to use \texttt{new} to allocate memory for the array before using it in the
		\texttt{for} loop.
	\end{sol}
	\begin{ex}{11}
		Write a code fragment that prints the contents of a two-dimensional boolean array,
		using \texttt{*} to represent \texttt{true} and a space to represent \texttt{false}.
		Include row and column numbers.
	\end{ex}
	\begin{sol}
		\
		\begin{lstlisting}
for (int i = 0; i < m; i++) {
	for(int j = 0; j < n; j++)
		System.out.printf("(%d,%d): %s ", i, j, (a[i][j]) ? "*" : " ");
		
	System.out.println();
}
		\end{lstlisting}
	\end{sol}
	\begin{ex}{12}
		What does the following code fragment print?
		\begin{lstlisting}
int[] a = new int[10];
for (int i = 0; i < 10; i++)
	a[i] = 9 - i;
for (int i = 0; i < 10; i++)
	a[i] = a[a[i]];
for (int i = 0; i < 10; i++)
	System.out.println(a[i]);
		\end{lstlisting}
	\end{ex}
	\begin{sol}
		\
		The first lop sets \texttt{a}  to \texttt{\{9, 8, 7, 6, 5, 4, 3, 2, 1, 0\}}.
		The second loop \texttt{\{0, 1, 2, 3, 4, 4, 3, 2, 1, 0\}}. Thus the output is:
		\begin{lstlisting}
0
1
2
3
4
4
3
2
1
0
		\end{lstlisting}
	\end{sol}
	\begin{ex}{13}
		Write a code fragment to print the \emph{transposition} (rows and columns changed)
		of a two-dimensional array with \emph{m} rows and \emph{n} columns.
	\end{ex}
	\begin{sol}
		\
		\begin{lstlisting}
for (int i = 0; i < m; i++) {
	for (int j = 0; j < n; j++)
		System.out.printf("%d ", a[j][i]);
	System.out.println();
}
		\end{lstlisting}
	\end{sol}
	\begin{ex}{14}
		Write a static method \texttt{lg()} that takes an \texttt{int} value \texttt{n}
		as argument and returns the largest \texttt{int} not larger than the base-2 logarithm
		of \texttt{n}. Do \emph{not} use \texttt{Math}.
	\end{ex}
	\begin{sol}
		See the class \texttt{com.segarciat.algs4.ch1.sec.ex14.LgFloor}.
		Note that the logarithm is only defined for positive numbers,
		so we begin by throwing an exception if $n \leq 0$. Assuming now that
		$n > 0$, suppose that $2^m$ is the largest power of $2$ in its base-2 (binary)
		representation. Since $\log_2$ is monotonic, we know that:
		\begin{align*}
			\log_2(2^m)&\leq \log_2(n)\\
			m&\leq  \log_2(n)
		\end{align*}
		Put another way, $m = \lfloor \log_2(n)\rfloor$, the \emph{floor} of the base-2
		logarithm; this is the number requested in this question. For example, we can
		make a table listing some sample values:
		\begin{center}
			\begin{tabular}{c|cccccccccccccccc}
				$n$ & 1 & 2 & 3 & 4 & 5 & 6 & 7 & 8 & 9 & 10 & 11 & 12 & 13 & 14 & 15 & 16\\
				\hline
				$\lfloor \log_2(n)\rfloor$ & 0 & 1 & 1 & 2 & 2 & 2 & 2 & 3 & 3 & 3 & 3 & 3 & 3
				& 3 & 3 & 4
			\end{tabular}
		\end{center}
		Note that if $2^m$ is the largest power in the binary representation of $n$, then
		$n$ can be represented by $m + 1$ bits. To determine the number of bits in the
		binary representation of $n$ we can repeatedly divide by 2 (or perform logical right
		arithmetic shifts) until the quantity becomes 0. Subtracting 1 from this yields
		$m$. Equivalently, we can continue as long as the result of dividing by 2 is still
		greater than 1, and skip the subtraction.
	\end{sol}
	\begin{ex}{15}
		Write a static method \texttt{histogram()} that takes an array \texttt{a[]}
		of \texttt{int} values and an integer \texttt{m} as arguments and returns
		an array \texttt{m} whose \texttt{i}th entry is the number of times the integer
		\texttt{i} appeared in the argument array. If the values in \texttt{a[]} are all
		between \texttt{0} and \texttt{m-1}, the sum of the values in the returned array
		should equal to \texttt{a.length}.
	\end{ex}
	\begin{sol}
		See \texttt{com.segarciat.algs4.ch1.sec.ex19.Histogram}.
	\end{sol}
	\begin{ex}{16}
		Give the value of \texttt{exR1(6)}:
		\begin{lstlisting}
public static String exR1(int n)
{
	if (n <= 0) return "";
	return exR1(n-3) + n + exR1(n-2) + n;
}
		\end{lstlisting}
	\end{ex}
	\begin{sol}
		\
			The first call is as follows:
		\begin{lstlisting}[language={}]
exR1(6) -> exR1(3) + 6 + exR1(4) + 6
		\end{lstlisting}
		Now we look at \texttt{exR1(3)}:
		\begin{lstlisting}[language={}]
exR1(3) -> exR1(0) + 3 + exR1(1) + 3
		\end{lstlisting}
		By the base case, \texttt{exR1(0)} is \texttt{""}. Meanwhile, we keep going for \texttt{exR1(1)}:
		\begin{lstlisting}[language={}]
exR1(1) -> exR1(-2) + 1 + exR1(-1) + 1
		\end{lstlisting}
		Since \texttt{exR1(-2)} and \texttt{exR1(-1)} evaluate to \texttt{""} due to the
		base case, we get \texttt{exR1(1)} is \texttt{"11"}. Now \texttt{exR1(3)} is \texttt{"3113"}.
		Next we need \texttt{exR1(4)}:
		\begin{lstlisting}[language={}]
exR1(4) -> exR1(1) + 4 + exR1(2) + 4
		\end{lstlisting}
		We already know that \texttt{exR1(1)} is \texttt{"11"}. For \texttt{exR1(2)}:
		\begin{lstlisting}[language={}]
exR1(2) -> exR1(-1) + 2 + exR1(0) + 2
		\end{lstlisting}
		Hence \texttt{exR1(2)} is \texttt{"22"}. Altogether, we find that \texttt{exR1(4)} is
		\texttt{"114224"}. Finally, the value of \texttt{exR1(6)} is:
		\begin{lstlisting}[language={}]
311361142246
		\end{lstlisting}
	\end{sol}
	\begin{ex}{17}
		Criticize the following recursive function:
		\begin{lstlisting}
public static String exR2(int n)
{
	String s = exR2(n-3) + n + n + exR2(n-2) + n;
	if (n <= 0) return "";
	return s;
}
		\end{lstlisting}
	\end{ex}
	\begin{sol}
		\
		Because the base case comes after the recursive step, a program that invokes this
		function will crash with \texttt{StackOverflowError}.
	\end{sol}
	\begin{ex}{18}
		Consider the following recursive function:
		\begin{lstlisting}
public static int mystery(int a, int b)
{
	if (b == 0)		return 0;
	if (b % 2 == 0)	return mystery(a+a, b/2);
	return mystery(a+a, b/2) + a;
}
		\end{lstlisting}
	What are the values of \texttt{mystery(2, 25)} and \texttt{mystery(3, 11)}? Given positive integers
	\texttt{a} and \texttt{b}, describe what \texttt{mystery(a, b)} computes. Answer the same
	question, but replace the three \texttt{+} operators with \texttt{*} and replace
	\texttt{return 0} with \texttt{return 1}.
	\end{ex}
	\begin{sol}
		\
		Begin with \texttt{mystery(2, 25)}:
		\begin{lstlisting}
mystery(2, 25) -> 2 + mystery(4, 12):
mystery(4, 12) -> mystery(8, 6):
mystery(8, 6 )  -> mystery(16, 3):
mystery(16, 3) -> 16 + mystery(32, 1):
mystery(32, 1) -> 32 + mystery(64, 0):
mystery(64, 0) -> 0
		\end{lstlisting}
		Tracing back the calls, the result is $2 + 16 + 32 + 0 = 50 = 2\cdot 25$.
		Similarly:
		\begin{lstlisting}
mystery(3, 11) -> 3 + mystery(6, 5):
mystery(6, 5 ) -> 6 + mystery(12, 2):
mystery(12, 2) -> mystery(24, 1):
mystery(24, 1) -> 24 + mystery(48, 0):
mystery(48, 0) -> 0
		\end{lstlisting}
		Tracing back the calls, the result is $3 + 6 + 24 + 0 = 33 = 3\cdot 11$.
		It appears that the \texttt{mystery(a, b)} computes the product $a\cdot b$.
		In essence, we are using the binary representation of \texttt{b} decide which
		weights of the multiples of \texttt{a} we should add.

		Next, we replace \texttt{+} with \texttt{*} and \texttt{return 0} with \texttt{return 1}:
		\begin{lstlisting}
mystery(2, 25) -> 2 * mystery(4, 12):
mystery(4, 12) -> mystery(16, 6):
mystery(16, 6) -> mystery(256, 3):
mystery(256, 3)-> 256 * mystery(65536, 1):
mystery(65536, 1) -> 65536 * mystery(4294967296, 0):
mystery(4294967296, 0) -> 1
		\end{lstlisting}
		The result is $2 \cdot 256 \cdot 65536 \cdot 1 = 33554432 = 2^{25}$.
		Meanwhile:
		\begin{lstlisting}
mystery(3, 11) -> 3 * mystery(9, 5):
mystery(9, 5 ) -> 9 * mystery(81, 2):
mystery(81, 2) -> mystery(6561, 1):
mystery(6561, 1) -> 6561 * mystery(43046721, 0):
mystery(43046721, 0) -> 1
		\end{lstlisting}
		The result is $3\cdot 9\cdot 6561\cdot 1 = 177147$, which is $3^{11}$.
		In this case, \texttt{mystery(a, b)} appears to be computing $a^b$ (meaning
		$a$ to the power of $b$).
	\end{sol}
	\begin{ex}{19}
		Run the following program on your computer (see Section 1.1 page 57 for the
		snippet). What is the largest value of \texttt{n} for which this program takes
		less than 1 hour to compute the value of \texttt{fibonacci(n)}? Develop a better
		implementation of \texttt{fibonacci(n)} that saves computed values in an array.
	\end{ex}
	\begin{sol}
		See the class \texttt{com.segarciat.algs4.ch1.sec.ex19.Fibonacci}. On my computer,
		when $n=58$, it takes just under a hour, and during $n=59$, it takes over an hour.
		See my \texttt{fibonacciFaster} method, which completes through $n=90$ in less than a
		second.
	\end{sol}
	\begin{ex}{20}
		Write a recursive static method that computes the value of $\ln (n!)$.
	\end{ex}
	\begin{sol}
		See the class \texttt{com.segarciat.algs4.ch1.sec.ex20.FactorialLog}.
		The implementation is fairly trivial if we recall the power rule of logarithms.
		If $x,y$ are any two positive real numbers, then
		\[
		\ln(xy) = \ln(x) + \ln(y)
		\]
		Since $n!=n\cdot (n-1)!$ for $n\geq 1$ and $0!=1$, we have:
		\[
		\ln(n!) = \ln(n\cdot (n-1)!) = \ln(n) + \ln[(n-1)!]
		\]
	\end{sol}

	\begin{ex}{21}
		Write a program that reads in lines from standard input with each line containing
		a name and two integers and then uses \texttt{printf()} to print a table with a column
		of the names, the integers, and the result of dividing the first by the second,
		accurate to three decimal places. You could use a program like this to tabulate batting
		averages for baseball players or grades for students.
	\end{ex}
	\begin{sol}
		See the class \texttt{com.segarciat.algs4.ch1.sec.ex21.StdinDivision}. Notice
		that this behaves much like the \texttt{cat} command in a  UNIX-based system.
		Ideally, input comes from a while so that  output will be redirected to a file so
		that it is not interspersed with the input.
	\end{sol}
	\begin{ex}{22}
		Write a version of \texttt{BinarySearch} that uses the recursive \texttt{indexOf()}
		given on page 25 and \emph{traces} the method calls. Each time the recursive method
		is called, print the argument values \texttt{lo} and \texttt{hi}, indented by the depth
		of the recursion. \emph{Hint}: Add an argument to the recursive method that keeps track
		of the depth.
	\end{ex}
	\begin{sol}
		See the class \texttt{com.segarciat.algs4.ch1.sec.ex23.BinarySearchTrace}. Since the
		purpose of the \texttt{depth} argument is to affect indentation, I decided to
		pass a \texttt{StringBuilder} object to control the indentation, which
		is appropriate because the recursion in this case is unidirectional.
	\end{sol}
	\begin{ex}{23}
		Add to the \texttt{BinarySearch} test client the ability to respond to
		a second argument: \texttt{+} to print numbers from standard input that
		\texttt{are not} in the whitelist,  \texttt{-} to print numbers that \emph{are}
		in the whitelist.
	\end{ex}
	\begin{sol}
		See the class \texttt{com.segarciat.algs4.ch1.sec.ex23.Filtering}.
	\end{sol}
	\begin{ex}{24}
		Give the sequence of values of $p$ and $q$ that are computed when Euclid's
		algorithm is used to compute  the greatest common division of $105$ and $24$. Extend
		the code given on page 4 to develop a program \texttt{Euclid} that takes two integers
		from the command  line and computes their greatest common divisor, printing out the
		two arguments for each call on the recursive method. Use your program to compute the
		greatest common divisor of $1111111$ and $1234567$.
	\end{ex}
	\begin{sol}
		Euclid's algorithm computes successive remainders. Hence if $p_0=p$, $q_0=q$,
		then $r_{n} = p_{n-1}\mod q_{n-1}$ for integer $n\geq 1$. If $r_n$ is $0$, then
		the greatest common divisor is $r_n$. Otherwise, we set $p_n = q_{n-1}$ and $q_n = r_{n}$.
		\begin{center}
			\begin{tabular}{c|c|c}
				$p$ & $q$ & $r$ \\
				\hline
				$105$ &  $24$ & $9$ \\
				$24$ & $9$ & $6$ \\
				$9$ & $6$ & $3$ \\
				$6$ & $3$ & $0$
			\end{tabular}
		\end{center}
		See the class \texttt{com.segarciat.algs4.ch1.sec.ex24.Euclid}.
	\end{sol}
	\begin{ex}{25}
		Use mathematical induction to prove that Euclid's  algorithm computes  the  greatest
		common  divisor for any pair of nonnegative integers $p$ and $q$.
	\end{ex}
	\begin{sol}
		\
		\begin{proof}
			Suppose that $q = 0$. Then the algorithm returns $p$ as the  greatest common
			divisor and we are done.
			
			Suppose now that $p$ and $q$ are nonnegative integers with $q>0$.
			Let $k\in \mathbb{N}$, and define:
			\begin{align*}
				p_0 & := p \\
				q_0 & := q \\
				r_{k-1} & := p_{k-1} \mod q_{k-1},\\
				p_{k} & := q_{k-1} \\
				q_{k} & := r_{k-1}
			\end{align*}
			That is, $r_{k-1}$ is the remainder of dividing $p_{k-1}$ by $q_{k-1}$.
			If $r_{k-1}\neq 0$, then the algorithm continues to produce values for
			these sequences. The \texttt{p\% q} operation in the algorithm translates into
			the $p_{k-1}\mod q_{k-1}$ operation above, which  always produces an integer $r_{k-1}$
			satisfying $0 \leq r_{k-1} < q_{k-1}$. Hence, the sequence of successive quotients,
			$k\mapsto q_k$, form a strictly monotonically decreasing sequence of integers.
			Since it is strictly decreasing, the values decrease by at least $1$ after each
			step, so the sequence must converge to $0$ after at most $q$ steps, meaning
			$q_k=0$ for some $k\in\mathbb{N}$. By assumption we  know $q_0 \neq 0$, so
			let $N$ the smallest nonnegative integer such that $q_{N + 1} = 0$, and set $d = q_N$.
			Then the algorithm will return $d$.
			
			We must prove that:
			\begin{enumerate}[label=(\roman*)]
				\item $d$ is a common divisor of $p$ and $q$.
				\item $d$ is the greatest among all the divisors of $p$ and $q$.
			\end{enumerate}
			To prove (i), note that $r_N = q_{N+1} = 0$, which means $r_N = 0$, and hence,
			$q_N$ divides $p_N$. Since $d = q_N$, this implies that $d$ is a common divisor
			of $p_N$ and $q_N$. If $N=0$, then $p_N=p$ and $q_N=q$, and we are done. Suppose
			that $N > 0$. Then $p_N = q_{N-1}$ and $q_N = r_{N-1}$, and $d$ divides $q_{N-1}$
			and $r_{N-1}$. By the definition  of $r_{N-1}$, there is a non-negative integer
			$s_{N-1}$ such that
			\[
			p_{N-1} = s_{N-1} \cdot q_{N-1} + r_{N-1}
			\]
			Since $d$ divides $q_{N-1}$ and $r_{N-1}$, it follows  that $d$ divides $p_{N-1}$.
			Proceeding by induction, we conclude that $d$ divides $p_k$ and $q_k$ for all
			$k\in\mathbb{N}$. Since $q_{0}=p_1$ and $r_0=q_1$, the definition of $p_0$ in terms
			of $q_0$ and $r_0$ implies that $d$ divides $p_0$ also. Hence, and in turn, $d$ divides
			$p_0 = p$ and $q_0 = q$, proving that $d$ is a common divisor of $p$ and $q$.
			This completes the proof of (i).
			
			To prove (ii), suppose that $d'$ is the greatest common divisor of $p$ and $q$.
			Then $d \leq d'$, by definition. If we can prove that we also have $d' \leq d$,
			then both conditions together imply that $d = d'$. Recalling that $d = q_N$,
			suppose that $N = 0$. Since $d'$ is the greatest common divisor, it must divide
			$p$ and $q$. In particular, $d'$ divides $d$ because $q=q_0=d$. Hence, we must
			have $d'\leq d$. Suppose now that $N>0$, that $k\in, \mathbb{N}$, and $d'$ divides
			$p_{k-1}$ and $q_{k-1}$. By  definition of $r_{k-1}$, we can write
			\begin{align*}
				r_{k-1} &= p_{k-1} - s_{k-1}\cdot q_{k-1}
			\end{align*}
			for some non-negative integer $s_{k-1}$. Since $d'$ divides $p_{k-1}$ and $q_{k-1}$,
			it must divide $r_{k-1}$. In particular, $d'$ divides $q_{k-1}$ and
			$r_{k-1}$. Since $p_k = q_{k-1}$ and $q_k = r_{k-1}$, this means $d'$ divides
			$p_k$ and $q_k$. By induction, $d'$  divides $p_k$ and $q_k$ for all $k\in\mathbb{N}$.
			In particular, $d'$ divides $q_N$, and since $d = q_N$, this means $d'$ divides $d$.
			We conclude that $d' \leq d$.
		\end{proof}
	\end{sol}
	\begin{ex}{26}
		\emph{Sorting three numbers}. Suppose that the variables \texttt{a}, \texttt{b},
		\texttt{c}, and \texttt{c} are all of the same numeric primitive type. Show that
		the following code puts \texttt{a}, \texttt{b}, and \texttt{c} in ascending order:
		\begin{lstlisting}
		if (a > b) { t = a; a = b; b = t; }
		if (a > c) { t = a; a = c; c = t; }
		if (b > c) { t = b; b = c; c = t; }
		\end{lstlisting}
	\end{ex}
	\begin{sol}
		\begin{proof}
			Suppose that \texttt{a > b} so that the body of the first \texttt{if} runs.
			The effect is to swap \texttt{a} and \texttt{b}. After the first \texttt{if}
			statement, we have \texttt{a <= b}, but we do not yet know how \texttt{c} compares.
			
			Suppose that \texttt{a > c}. This means that \texttt{b >= a} and \texttt{a >= c}.
			If the body of the second \texttt{if} runs, then it swaps \texttt{a} and \texttt{c}.
			At this point, we will have \texttt{b >= c} and \texttt{c >= a}. If the body of
			the \texttt{if} does not run, then we can certainly assure that \texttt{b >= a}
			and \texttt{c >= a}.
			
			At this point we can be sure that \texttt{a} has the smallest value. The last
			\texttt{if} tests to see if \texttt{b} is greater than \texttt{c}, in which case
			it swaps them. After this, we will have \texttt{c >= b} and \texttt{b >= a},
			so that \texttt{a}, \texttt{b}, and \texttt{c} are in ascending order.
		\end{proof}
	\end{sol}
	\begin{ex}{27}
		\emph{Binomial distribution}. Estimate the number of recursive calls that would be
		used by the code
		\begin{lstlisting}
	public static double binomial(int n, int k, double p)
	{
		if ((n == 0) && (k == 0)) return 1.0;
		if ((n  < 0) || (k < 0 )) return 0.0;
		return (1 - p)*binomial(n-1, k, p) + p*binomial(n-1, k-1, p);
	}
		\end{lstlisting}
		to compute \texttt{binomial(100, 50, 0.25)}. Develop a better implementation
		that is based on saving computed values in an array.
	\end{ex}
	\begin{sol}
		The sequence of invocations, top-to-bottom, is as follows:
		\begin{align*}
			\texttt{binomial(100, 50, p)}\\
			\texttt{binomial(99, 50, p)},\quad \texttt{binomial(99, 49, p)}\\
			\texttt{binomial(98, 50, p)},\quad \texttt{binomial(98, 49, p)},
			\texttt{binomial(98, 49, p)},\quad \texttt{binomial(98,48,p)}\\
			\vdots
		\end{align*}
		Notice that each row corresponds to a separate value of $n$. Moreover, if
		$r$ denotes the $r$-th row, starting at row $r=0$ (the initial invocation),
		then $i$ can be as large as $100$, and the $r$-th row has $2^r$ such invocations.
		Therefore, a (not-so-tight) upper bound on the number of recursive invocations is:
		\begin{align*}
			\sum_{r=0}^{100}2^r = 2^{101}-1-1=2^{101}-2
		\end{align*}
		Notice the extra $-1$ because the initial invocation \texttt{binomial(100, 50, p)}
		is not a recursive call. Beyond row $i=100$ we will have $n=-1$, which will
		not lead to a recursive invocation due to the guarding \texttt{if} statements.
		However, when $i=50$, the value of $k$ is $0$ for the rightmost  column, and $50$
		for the leftmost column. Hence, the rightmost instance of the call, namely
		\texttt{binomial(50, 0, p)}, will not lead to a recursive call thereafter, because
		of the \texttt{if} that guards when $k=0$. Hence, starting at $i=51$ (meaning $n=49$),
		some recursive invocations will begin to return because of $k$ reaching its threshold.
	\end{sol}
	\begin{ex}{28}
		\emph{Remove duplicates}. Modify the test client in \texttt{BinarySearch} to remove
		any duplicate keys in the whitelist after the sort.
	\end{ex}
	\begin{sol}
		See the \texttt{com.segarciat.algs4.ch1.sec1.ex28.RemoveDuplicates} class.
		After devising a way to remove the duplicates, I used the \texttt{Array.copyOf()}
		method to create a new array large enough to hold the unique values.
	\end{sol}
	\begin{ex}{29}
		\emph{Equal keys}. Add to \texttt{BinarySearch} a static method \texttt{rank()}
		that takes a sorted array of \texttt{int} values (some of which may be equal) and
		a key as arguments and returns the number of elements that are smaller than the key
		and a similar  method \texttt{count()} that returns the number of elements equal
		to the key. \emph{Note}: If \texttt{i} and \texttt{j} are the values returned by
		\texttt{rank(a, key)} and \texttt{count(a, key)}, respectively, then \texttt{a[i..i+j-1]}
		are the values in the array that are equal to key.
	\end{ex}
	\begin{sol}
		My \texttt{rank()} algorithm uses a modified version of the \texttt{indexOf()}
		algorithm. We can collapse the three \texttt{if}-\texttt{else} branches
		from \texttt{indexOf()} into two:
		\begin{enumerate}[label=(\roman*)]
			\item If the target entry is larger than the middle key, increment \texttt{lo}
			to search starting at \texttt{mid + 1}.
			\item Otherwise, decrement \texttt{hi} to end the search at \texttt{mid - 1}.
		\end{enumerate}
		This guarantees that, at each step, \texttt{key} is larger than \texttt{a[lo]}.
		As usual, the search ends when \texttt{lo} exceeds \texttt{hi}. Because we never
		\texttt{break}, the algorithm takes all keys into consideration. This of course
		necessitates that the array be sorted to exploit transitivity.
		
		The \texttt{count()} algorithm can the be implemented in a couple different ways.
		One way is as \texttt{rank(a, key + 1) - rank(a, key)}. Another way is to implement
		a modified version of \texttt{rank()}, call it \texttt{rankGe()} that makes a single
		change: replace \texttt{key > a[mid]} with \texttt{key >= a[mid]} in the \texttt{if}
		statement that controls the assignment operation on \texttt{lo}. Then we can implement
		\texttt{count()} as \texttt{rankGe(a, key) - rank(a, key)}. The former approach would
		work only for integer data, but the latter would work for other data types.
	\end{sol}
	\pagebreak
	\printbibliography
\end{document}