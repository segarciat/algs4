\documentclass[12pt, a4paper]{article}

\usepackage[utf8]{inputenc}
% Limit the page margin to only 1 inch.
\usepackage[margin=1in]{geometry}

%Imports biblatex package
\usepackage[
backend=biber,
style=alphabetic
]{biblatex}
\addbibresource{../../algs4e.bib}

% Enables the `align' environment.
\usepackage{amsmath}
% Provides useful environments, such as:
% - \begin{proof} ...\end{proof}
\usepackage{amsthm}
% Enables using \mathbb{}, for example \mathbb{N} for the set of natural numbers.
\usepackage{amssymb}

% Allows using letters in enumerate list environment. Use, for example:
%\begin{enumerate}[label=(\alph*)]
% ...
%\end{enumerate}
\usepackage[inline]{enumitem}

% Enable importing external graphic files and provides useful commannds, like \graphicspath{}
\usepackage{graphicx}
% Images are located in a directory called images in the current directory.
\graphicspath{{./images/}}

% Make links look better by default.
% See: https://tex.stackexchange.com/questions/823/remove-ugly-borders-around-clickable-cross-references-and-hyperlinks
\usepackage[hidelinks]{hyperref}
\usepackage{xcolor}
\hypersetup{
	colorlinks,
	linkcolor={red!50!black},
	citecolor={blue!50!black},
	urlcolor={blue!80!black}
}


% Code Listings. Source:
% https://stackoverflow.com/questions/3175105/inserting-code-in-this-latex-document-with-indentation
\usepackage{listings}
\usepackage{color}

\definecolor{dkgreen}{rgb}{0,0.6,0}
\definecolor{gray}{rgb}{0.5,0.5,0.5}
\definecolor{mauve}{rgb}{0.58,0,0.82}

\lstset{frame=tb,
	language=Java,
	aboveskip=3mm,
	belowskip=3mm,
	showstringspaces=false,
	columns=flexible,
	basicstyle={\small\ttfamily},
	numbers=none,
	numberstyle=\tiny\color{gray},
	keywordstyle=\color{blue},
	commentstyle=\color{dkgreen},
	stringstyle=\color{mauve},
	breaklines=true,
	breakatwhitespace=true,
	tabsize=3
}

\newcommand{\prob}{\text{P}}
%\newcommand{\complement}{\mathsf{c}}

% Define an environment called "ex" (for Exercise) so that I can do: \begin{ex}{1.5}...\end{ex}
\newenvironment{ex}[2][Exercise]
{\par\medskip\noindent \textbf{#1 #2.}}
{\medskip}

% Define a solution environment, similar to ex (exercise) environment.
\newenvironment{sol}[1][Solution]
{\par\medskip\noindent \textbf{#1.} }
{\medskip}

\begin{document}
	\noindent Sergio E. Garcia Tapia \hfill
	
	\noindent \emph{Algorithms} by Sedgewick and Wayne (4th edition) \cite{sedgewick_wayne}\hfill
	
	\noindent September 22nd, 2024\hfill 
	\section*{1.3: Bags, Queues, and Stacks}
	
	\begin{ex}{1}
		Add a method \texttt{isFull()} to \texttt{FixedCapacityStackOfStrings}.
	\end{ex}

	\begin{sol}
		See the \texttt{com.segarciat.algs4.ch1.sec3.ex1.FixedCapacityStackOfStrings}
		class.
	\end{sol}
	
	\begin{ex}{2}
		Give the output printed by \texttt{java Stack} for the input
		\begin{lstlisting}[language={}]
	it was - the best - of times - - - it was - the - -
		\end{lstlisting}
	\end{ex}
	
	\begin{sol}
		The \texttt{-} causes the latest added word to be removed. The contents
		of the stack at each step are as follows:
		\begin{lstlisting}
it
it was
it
it the
it the best
it the
it the of
it the of times
it the of
it the
it
it it
it it was
it it
it it the
it it
it
		\end{lstlisting}
		The output is the last line, \texttt{it}.
	\end{sol}
	\begin{ex}{3}
		Suppose that a client performs an intermixed sequence of (stack) \emph{push} and
		\emph{pop} operations. The push operations put the integers 0 through 9 in order
		onto the stack; the pop operations print out the return values. Which of the following sequence(s) could \emph{not} occur?
		\begin{enumerate}[label=(\alph*)]
			\item \texttt{4 3 2 1 0 9 8 7 6 5}
			\item \texttt{4 6 8 7 5 3 2 9 0 1}
			\item \texttt{2 5 6 7 4 8 9 3 1 0}
			\item \texttt{4 3 2 1 0 5 6 7 8 9}
			\item \texttt{1 2 3 4 5 6 9 8 7 0}
			\item \texttt{0 4 6 5 3 8 1 7 2 9}
			\item \texttt{1 4 7 9 8 6 5 3 0 2}
			\item \texttt{2 1 4 3 6 5 8 7 9 0}
		\end{enumerate}
	\end{ex}
	\begin{sol}
		\
		\begin{enumerate}[label=(\alph*)]
			\item Valid. This sequence involves pushing 0 through 4, then popping five times.
			Then, pushing 5 through  9, ad then popping five times.
			\item Invalid. The sequence involves pushing 0 through 4 and pop once to print 4. Then,
			we push 5 and 6, and pop once to print 6. Next, push 7 and 8 and then pop 8,
			and pop 7. Popping again would give 5. Popping again yields 3, and then 2.
			We can then push 9 and pop it. At this point we've got 0 and 1 left on the stack.
			The next item popped should be 1, so this sequence must be incorrect.
			\item Valid. We push 0, 1, 2, then pop 2. Next, we push 3, 4, 5, and pop 5. Next,
			we push 6 and pop it, then push 7 and pop it. We pop next (4). Next we
			push 8 and pop it, push 9 and pop it. Next we pop 3. Finally, we pop 1 and 0.
			\item Valid. We push 0, 1, 2, 3, and 4, then pop them all off, so the stack is empty.
			Next, we push 5 and pop it, push 6 and pop it, push 7 and pop it, push 8 and pop it,
			and push 9 and pop it.
			\item Valid. For inputs 0, 1, 2, 3, 4, 5, 6, we push and immediately pop. Then we
			push 7, 8, 9, and pop 4 times.
			\item Invalid. We push 0 and pop. Then, we push 1, 2, 3, 4 and pop 4. We now push 5 and 6,
			then we pop 6, 5, and 3. We push 7 and 8 and pop 8. If we pop next, we should get
			2 from the stack, which does not match the next value in the sequence (1).
			\item Invalid. We push 0 and 1, then pop 1. We push 2, 3, 4, then pop 4. We push 5, 6, 7,
			then pop 7. We push 8 and 9. Now we pop 9, pop 8, pop 6, pop 6, pop 5, and
			the next pop operation would be 2, but the sequence says 0.
			\item Valid We push 0, 1, 2, and pop 2 and 1. We push 3 and 4, then pop both. We push
			5 and 6, then pop both. We push 7 and 8, then pop both. We push 9 then pop it immediately.
			Number 0 remains, and we indeed pop it.
		\end{enumerate}
	\end{sol}
	
	\begin{ex}{4}
		Write a stack client \texttt{Parentheses} that reads in a text stream from standard
		input and uses a stack to determine whether its parentheses are properly balanced.
		For example, your program should print \texttt{true} for \texttt{[()]\{\}\{[()()]()\}}
		and \texttt{false} for \texttt{[(])}.
	\end{ex}
	\begin{sol}
		See the \texttt{com.segarciat.algs4.ch1.sec3.ex04.Parentheses} class.
	\end{sol}

	\begin{ex}{5}
		What does the following code fragment print when \texttt{n} is \texttt{50}?
		Give a high-level description of what it does when presented with a positive
		integer \texttt{n}.
		
		\begin{lstlisting}
		Stack<Integer> Stack = new Stack<Integer>();
		while (n > 0)
		{
			stack.push(n % 2);
			n = n / 2;
		}
		for (int d: stack) StdOut.print(d);
		StdOut.println();
		\end{lstlisting}
	\end{ex}
	\begin{sol}
		It prints the binary representation of \texttt{n}.
	\end{sol}

	\begin{ex}{6}
		What does the following code fragment do to the queue \texttt{q}?
		\begin{lstlisting}
		Stack<String> stack = new Stack<String>();
		while (!q.isEmpty())
			stack.push(q.dequeue());
		while (!stack.isempty())
			q.enqueue(stack.pop());
		\end{lstlisting}
	\end{ex}
	\begin{sol}
		The fragment reverses the order of the entries in the queue \texttt{q}.
	\end{sol}
	\begin{ex}{7}
		Add a method \texttt{peek()} to \texttt{Stack} that returns the most recently inserted
		item on the stack (without popping it).
	\end{ex}
	\begin{sol}
		See the \texttt{com.segarciat.algs4.ch1.sec3.ex07.Stack} class.
	\end{sol}
	\begin{ex}{8}
		Give the contents and size of the array for \texttt{ResizingArrayStackOfStrings}
		with the input
		\begin{lstlisting}[language={}]
		it was - the best - of times - - - it was - the - -
		\end{lstlisting}
	\end{ex}
	\begin{sol}
		The contents are as follows:
		\begin{lstlisting}[language={}]
null
it
it was
it null
it the
it the best null
it the null null
it the of null
it the of times
it the of null
it the null null
it null
it it
it it was null
it it null null
it it the null
it it null null
it null
		\end{lstlisting}
		Hence, the array ends with a size of 2, having \texttt{it} in its first entry
		and \texttt{null} in its second entry.
	\end{sol}
	\begin{ex}{9}
		Write a program that takes from standard input an expression without left parentheses
		and prints the equivalent infix expression with the parentheses inserted. For example,
		given the input:
		\begin{lstlisting}[language={}]
		1 + 2 ) * 3 - 4 ) * 5 - 6 ) ) )
		\end{lstlisting}
		your program should print
		\begin{lstlisting}[language={}]
		( ( 1 + 2 ) * ( ( 3 - 4 ) * (5 - 6 ) ) )
		\end{lstlisting}
	\end{ex}
	\begin{sol}
		See the \texttt{com.segarciat.algs4.ch1.sec3.ex09.BalancedInfix} class.
	\end{sol}
	\begin{ex}{10}
		Write a filter \texttt{InfixToPostfix} that converts an arithmetic expression
		from infix to postfix.
	\end{ex}
	\begin{sol}
		See the \texttt{com.segarciat.algs4.ch1.sec3.ex10.InfixToPostfix} class.
	\end{sol}
	\begin{ex}{11}
		Write a program \texttt{EvaluatePostfix} that takes a postfix expression from
		standard input, evaluates it, and prints the value. (Piping the output of
		your program from the previous exercise to this program gives an equivalent
		behavior of \texttt{Evaluate}).
	\end{ex}
	\begin{sol}
		See the \texttt{com.segarciat.algs4.ch1.sec3.ex11.EvaluatePostfix} class.
	\end{sol}
	\begin{ex}{12}
		Write an iterable \texttt{Stack} \emph{client} that has a static method \texttt{copy()}
		that takes a stack of strings as argument and returns a copy of the stack.
		\emph{Note}: This ability is a prime example of the value of having an iterator,
		because it allows development of such functionality without changing the basic
		API.
	\end{ex}
	\begin{sol}
		See the \texttt{com.segarciat.algs4.ch1.sec3.ex12.StackCopy} class.
	\end{sol}
	\begin{ex}{13}
		Suppose that a client performs an intermixed sequence of (queue) \emph{enqueue}
		and \emph{dequeue} operations. The enqueue operations put the integers 0 through 9
		in order onto the queue; the dequeue operations print out the return value. Which
		of the following sequence(s) could \emph{not} occur?
		\begin{enumerate}[label=(\alph*)]
			\item \texttt{0 1 2 3 4 5 6 7 8 9}
			\item \texttt{4 6 8 7 5 3 2 9 0 1}
			\item \texttt{2 5 6 7 4 8 9 3 1 0}
			\item \texttt{4 3 2 1 0 5 6 7 8 9}
		\end{enumerate}
	\end{ex}
	\begin{sol}
		\begin{enumerate}[label=(\alph*)]
			\item Valid.
			\item Impossible.
			\item Impossible.
			\item Impossible.
		\end{enumerate}
		This exercise is trivial because a queue preserves the order of the input.
		Thus, sequence (a) should always be the result. This unlike stacks, as in Exercise 1.3.3.
	\end{sol}
	\begin{ex}{14}
		Develop a class \texttt{ResizingArrayQueueOfStrings} that implements the queue abstraction
		with a fixed-size array, and then extend your implementation to remove the size restriction.
	\end{ex}
	\begin{sol}
		See the \texttt{com.segarciat.algs4.ch1.sec3.ex14.ResizingArrayQueueOfStrings} class.
	\end{sol}
	\begin{ex}{1.3.15}
		Write a \texttt{Stack} or \texttt{Queue} client that takes a command-line argument
		\texttt{k} and prints the \texttt{k}th from the last string found on standard input
		(assuming that standard input has \texttt{k} or more strings). Use memory proportional
		to \texttt{k}.
	\end{ex}
	\begin{sol}
		See the \texttt{com.segarciat.algs4.ch1.sec3.ex15.KthFromLast} class. I did this
		exercise in two ways by implementing private \texttt{static} functions
		\texttt{usingQueue()} and \texttt{usingStack()}. The queue approach was much
		simpler, and the stack approach required me to use two stacks, as well as
		needing to replace the stack every so often.
	\end{sol}
	\begin{ex}{1.3.16}
		Using \texttt{readAllInts()} on page 126 as a model, write a static method
		\texttt{readAllDates()} for \texttt{Date} that reads dates from standard input
		in the format specified on page 119 and returns an array containing them.
	\end{ex}
	\begin{sol}
		See the \texttt{com.segarciat.algs4.ch1.sec3.ex16.ParsingDatesToArray} class.
	\end{sol}
	\begin{ex}{1.3.17}
		Do Exercise 1.3.16 for \texttt{Transaction}.
	\end{ex}
	\begin{sol}
		See the \texttt{com.segarciat.algs4.ch1.sec3.ex17.ParsingTransactionsToArray} class.
	\end{sol}
	\begin{ex}{1.3.18}
		Suppose \texttt{x} is a linked-list and not the last node on the list. What is
		the effect of the following code fragment?
		\begin{lstlisting}
		x.next = x.next.next;
		\end{lstlisting}
	\end{ex}
	\begin{sol}
		The fragment removes the successor of \texttt{x} in the linked list. Now
		The successor itself, call it \texttt{y}, was a linked list also,  and
		it pointed to linked list, call it \texttt{z}. Now \texttt{x} points to
		\texttt{z}.
	\end{sol}
	\begin{ex}{1.3.19}
		Give a code fragment that removes the last node in a linked list whose first
		node is \texttt{first}.
	\end{ex}
	\begin{sol}
		\begin{lstlisting}
			if (first == null)
				throw new NoSuchElementException("list is empty");
			Node previous = null;
			Node current = first;
			while (current.next != null) {
				prev = current;
				current = current.next
			}
			if (prev == null)
				first = null
			else
				prev.next = null;
		\end{lstlisting}
	\end{sol}
	\begin{ex}{20}
		Write a method \texttt{delete()} that takes an \texttt{int} argument \texttt{k}
		and deletes the \texttt{k}th element in a linked list, if it exists.
	\end{ex}
	\begin{sol}
		See the \texttt{com.segarciat.algs4.ch1.sec3.ex20.LinkedList} class. I decided
		to implement the linked list so that the most recent element is added to the end
		(like a queue) and not the front (unlike a stack).
	\end{sol}
	\begin{ex}{21}
		Write a method \texttt{find()} that takes a linked list and a string \texttt{key}
		as arguments and returns \texttt{true} if some node in the list has \texttt{key}
		as its item field, \texttt{false} otherwise.
	\end{ex}
	\begin{sol}
		The specification of this exercise was slightly unclear to me. On the surface,
		it seems I need a method that takes a \texttt{String} argument, and either a
		\texttt{Node<String>} argument, or \texttt{LinkedList<String>}, for example.
		Assuming that the linked list is an abstract data type, there is no direct access to
		the items in the list. Thus unless the method belongs to the linked list class, it's
		not possible to assert the value if the ``current" element under consideration.
		Moreover in this section we've worked mostly with type-generic classes, so
		this method does not seem like it needs to be string-specific as long as we
		can use the \texttt{equals()} method.I decided to use the class from Exercise 1.3.20,
		to which I added the \texttt{find()} method.
		
		See the \texttt{com.segaciat.algs4.ch1.sec3.ex21.LinkedList} class.
	\end{sol}
	\begin{ex}{22}
		Suppose that \texttt{x} is a linked list \texttt{Node}. What does the following
		code fragment do?
		\begin{lstlisting}[language={}]
		t.next = x.next;
		x.next = t;
		\end{lstlisting}
	\end{ex}
	\begin{sol}
		The line \texttt{t.next = x.next} makes it so that \texttt{x} and \texttt{t} point to
		the same item (call it \texttt{y}). The line \texttt{x.next = t} makes it so that
		\texttt{x} now points to \texttt{t}. Thus, before we have \texttt{x->}\texttt{y},
		and now we  have \texttt{x->t->y}. Thus, \texttt{t} is inserted immediately after
		\texttt{x}.
	\end{sol}
	\begin{ex}{23}
		Why does the following code fragment not do the same thing as the previous question?
		\begin{lstlisting}[language={}]
		x.next = t;
		t.next = x.next;
		\end{lstlisting}
	\end{ex}
	\begin{sol}
		Say \texttt{x.next} was \texttt{y}. The line \texttt{x.next = t} makes \texttt{x}
		point to \texttt{t}, but now nothing points to \texttt{y}. The next line
		\texttt{t.next = x.next} now makes it so that \texttt{t} points to \texttt{x.next},
		which is now \texttt{t}. Thus, \texttt{x} points to \texttt{t}, and \texttt{t}
		points to itself.
	\end{sol}
	\begin{ex}{24}
		Write a method \texttt{removeAfter()} that takes a linked-list\texttt{Node}
		as argument and removes the node following the given one (and does nothing if the
		argument node is null).
	\end{ex}
	\begin{sol}
		See the \texttt{com.segarciat.algs4.ch1.sec3.ex24.LinkedList} class.
	\end{sol}
	\begin{ex}{25}
		Write a method \texttt{insertAfter()} that takes two linked-list \texttt{Node}
		arguments and inserts the second after the first on the list  (and does nothing
		if either argument is null).
	\end{ex}
	\begin{sol}
		See the \texttt{com.segarciat.algs4.ch1.sec3.ex25.LinkedList} class.
	\end{sol}
	\begin{ex}{26}
		Write a method \texttt{remove()} that takes a linked list and a string \texttt{key}
		as arguments and removes all of the nodes in the list that have \texttt{key}
		as its item field.
	\end{ex}
	\begin{sol}
		See the \texttt{com.segarciat.algs4.ch1.sec3.ex26.LinkedList} class.
	\end{sol}
		\begin{ex}{27}
		Write a method \texttt{max()} that takes a reference to the first node in a linked
		list as argument and returns the value of the maximum key in the list. Assume
		that all keys are positive integers, and return \texttt{0} if the list is empty.
	\end{ex}
	\begin{sol}
		See the \texttt{com.segarciat.algs4.ch1.sec3.ex27.IntegerLinkedList} class.
	\end{sol}
	\begin{ex}{28}
		Develop a recursive solution to the previous question.
	\end{ex}
	\begin{sol}
		See the \texttt{com.segarciat.algs4.ch1.sec3.ex28.IntegerLinkedList} class.
	\end{sol}
	\begin{ex}{29}
		Write a \texttt{Queue} implementation that uses a \emph{circular} linked list, which
		is the same as a linked list except that no links are \emph{null} and the value
		of \texttt{last.next} is \texttt{first} whenever  the list is not empty. Keep
		only one \texttt{Node} instance variable (\texttt{last}).
	\end{ex}
	\begin{sol}
		See the \texttt{com.segarciat.algs4.ch1.sec3.ex29.Queue} class.
	\end{sol}
	\begin{ex}{30}
		Write a function that takes the first \texttt{Node} in a linked list as argument and
		(destructively) reverses the list, returning the first \texttt{Node} in the result.
	\end{ex}
	\begin{sol}
		See the \texttt{com.segarciat.algs4.ch1.sec3.ex30.LinkedList} class.
	\end{sol}
	\begin{ex}{1.3.31}
		Implement a nested class \texttt{DoubleNode} for building doubly-linked lists, where
		each node contains a reference to the item preceding it and the item following it
		in the list (\texttt{null} if there is no such item). Then implement a static method
		for the following tasks: insert at the beginning, insert at the end, remove from the
		beginning, remove from the end, insert before a given node, insert after a given node,
		and remove a given node.
	\end{ex}
	\begin{sol}
		See the \texttt{com.segarciat.algs4.ch1.sec3.ex31.LinkedList} class which has
		a \texttt{DoubleNode<Item>} nested class (which means that it is \texttt{static }also).
		All methods were implemented as \texttt{static} methods of the \texttt{DoubleNode<Item>}
		class. Hence, the job of the methods is to update the links, and thus they are independent
		of whether the outer class uses a \texttt{first}, a \texttt{last}, or some other
		combination of instance variables to keep track of the node references in its linked list.
	\end{sol}
	\begin{ex}{1.3.32}
		\emph{Stequeue}. A stack-ended queue or \emph{stequeue} is a data type that supports \emph{push},
		\emph{pop}, and \emph{enqueue}. Articulate an API for this ADT. Develop a
		linked list-based implementation.
	\end{ex}
	\begin{sol}
		The API is as follows:
		\begin{lstlisting}[language={}]
public class Steque<Item> implements Iterable<Item>
	Steque()
	public int size()
	public boolean isEmpty()
	public void enqueue(Item item)
	public void push(Item item)
	public Item pop()
	public Iterator<Item> iterator()
		\end{lstlisting}
		See the \texttt{com.segarciat.algs4.algs4.ch1.sec3.ex32.Steque} class.
	\end{sol}
	\begin{ex}{1.3.33}
		\emph{Deque}. A double-ended dequeue or \emph{deque} (pronounced ``deck") is
		like a stack or a queue but supports adding and removing items at both ends.
		A dequeue stores collections of items and supports the following API:
		\begin{lstlisting}[language={}]
public class Deque<Item> implements Iterable<Item>
	Deque()
	boolean isEmpty()
	int size()
	void pushLeft(Item item)
	void pushRight(Item item)
	Item popLeft()
	Item popRight()
		\end{lstlisting}
		Write a class \texttt{Deque} that uses a doubly-linked list to implement this API
		and a class \texttt{ResizingArrayQueue} that uses a resizing array.
	\end{ex}
	\begin{sol}
		See the \texttt{Dequeue} and \texttt{ResizingArrayQueue} classes I have implemented.
		For \texttt{Dequeue}, I maintained both a \texttt{first} and a \texttt{last} node
		in order to guarantee ``constant" time. For \texttt{ResizingArrayQueue}, I had
		to work harder due to the \texttt{pushLeft} operation. Specifically, I wanted to
		avoid having to shift every element to the right when needing to add an item to
		the start of the deque. I maintained a \texttt{head} index and a \texttt{tail}
		index, both of which point to a valid location at all times, except for when
		the deque is empty. However, it is not always the case that \texttt{head} is
		smaller than or equal to \texttt{tail} at all times. Rather, both indices
		are allowed to wrap around, enabling both \texttt{pushLeft} and \texttt{pushRight}
		to complete in constant time (except for when resizing is necessary).
	\end{sol}
	\pagebreak
	\printbibliography
\end{document}