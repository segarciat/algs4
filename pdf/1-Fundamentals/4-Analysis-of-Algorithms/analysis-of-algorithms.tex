\documentclass[12pt, a4paper]{article}

\usepackage[utf8]{inputenc}
% Limit the page margin to only 1 inch.
\usepackage[margin=1in]{geometry}

%Imports biblatex package
\usepackage[
backend=biber,
style=alphabetic
]{biblatex}
\addbibresource{../../algs4e.bib}

% Enables the `align' environment.
\usepackage{amsmath}
% Provides useful environments, such as:
% - \begin{proof} ...\end{proof}
\usepackage{amsthm}
% Enables using \mathbb{}, for example \mathbb{N} for the set of natural numbers.
\usepackage{amssymb}

% Allows using letters in enumerate list environment. Use, for example:
%\begin{enumerate}[label=(\alph*)]
% ...
%\end{enumerate}
\usepackage[inline]{enumitem}

% Enable importing external graphic files and provides useful commannds, like \graphicspath{}
\usepackage{graphicx}
% Images are located in a directory called images in the current directory.
\graphicspath{{./images/}}

% Make links look better by default.
% See: https://tex.stackexchange.com/questions/823/remove-ugly-borders-around-clickable-cross-references-and-hyperlinks
\usepackage[hidelinks]{hyperref}
\usepackage{xcolor}
\hypersetup{
	colorlinks,
	linkcolor={red!50!black},
	citecolor={blue!50!black},
	urlcolor={blue!80!black}
}


% Code Listings. Source:
% https://stackoverflow.com/questions/3175105/inserting-code-in-this-latex-document-with-indentation
\usepackage{listings}
\usepackage{color}

\definecolor{dkgreen}{rgb}{0,0.6,0}
\definecolor{gray}{rgb}{0.5,0.5,0.5}
\definecolor{mauve}{rgb}{0.58,0,0.82}

\lstset{frame=tb,
	language=Java,
	aboveskip=3mm,
	belowskip=3mm,
	showstringspaces=false,
	columns=flexible,
	basicstyle={\small\ttfamily},
	numbers=none,
	numberstyle=\tiny\color{gray},
	keywordstyle=\color{blue},
	commentstyle=\color{dkgreen},
	stringstyle=\color{mauve},
	breaklines=true,
	breakatwhitespace=true,
	tabsize=3
}

\newcommand{\prob}{\text{P}}
%\newcommand{\complement}{\mathsf{c}}

% Define an environment called "ex" (for Exercise) so that I can do: \begin{ex}{1.5}...\end{ex}
\newenvironment{ex}[2][Exercise]
{\par\medskip\noindent \textbf{#1 #2.}}
{\medskip}

% Define a solution environment, similar to ex (exercise) environment.
\newenvironment{sol}[1][Solution]
{\par\medskip\noindent \textbf{#1.} }
{\medskip}

\begin{document}
	\noindent Sergio E. Garcia Tapia \hfill
	
	\noindent \emph{Algorithms} by Sedgewick and Wayne (4th edition) \cite{sedgewick_wayne}\hfill
	
	\noindent September 30th, 2024\hfill 
	\section*{1.4: Analysis of Algorithms}
	\begin{ex}{1}
		Show that the number of different triples that can be chosen from $n$ items is
		precisely $n(n-1)(n-2)/6$. \emph{Hint}: Use mathematical induction  or a counting argument.
	\end{ex}
	\begin{sol}
		\begin{proof}
			This is the problem of choosing a combination of $3$ out of $n$, which is given
			by $\binom{n}{3}$, and
			\begin{align*}
				\binom{n}{3} = \frac{n!}{3!(n-3)!}=\frac{n\cdot (n-1)\cdot (n-2)\cdot (n-3)!}{3!(n-3)!}  = \frac{n(n-1)(n-2)}{6}
			\end{align*}
		\end{proof}
	\end{sol}
	\begin{ex}{2}
		Modify \texttt{ThreeSum} to work properly even when the \texttt{int} values are so
		large that adding two of them might cause integer overflow.
	\end{ex}
	\begin{sol}
		There are two cases when it comes to overflow:
		\begin{enumerate}[label=(\roman*)]
			\item \emph{Positive overflow}. The sum exceeds \texttt{Integer.MAX\_VALUE}.
			If two terms sum to \texttt{Integer.MAX\_VALUE + 1}, overflow occurs, and
			the value wraps around to \texttt{Integer.MIN\_VALUE}. Thus, if
			$a + b = \texttt{Integer.MAX\_VALUE + 1}$ and $c = \texttt{Integer.MIN\_VALUE}$,
			then we have a valid sum. However, if $a+b$ sums to anything larger, then
			no value of $c$ will do because $c$ cannot be smaller than \texttt{Integer.MIN\_VALUE}.
			\item \emph{Negative overflow}. The sum of two negative numbers $a$ and $b$, yielding
			a value below \texttt{Integer.MIN\_VALUE}. In this case, it's impossible to have
			$a + b = c$ for any 32-bit two's complement integer $c$.
		\end{enumerate}
	\end{sol}
	
	\pagebreak
	\printbibliography
\end{document}