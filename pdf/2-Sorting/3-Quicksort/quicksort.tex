\documentclass[12pt, a4paper]{article}

\usepackage[utf8]{inputenc}
% Limit the page margin to only 1 inch.
\usepackage[margin=1in]{geometry}

%Imports biblatex package
\usepackage[
backend=biber,
style=alphabetic
]{biblatex}
\addbibresource{../../algs4e.bib}

% Enables the `align' environment.
\usepackage{amsmath}
% Provides useful environments, such as:
% - \begin{proof} ...\end{proof}
\usepackage{amsthm}
% Enables using \mathbb{}, for example \mathbb{N} for the set of natural numbers.
\usepackage{amssymb}

% Allows using letters in enumerate list environment. Use, for example:
%\begin{enumerate}[label=(\alph*)]
% ...
%\end{enumerate}
\usepackage[inline]{enumitem}

% Enable importing external graphic files and provides useful commannds, like \graphicspath{}
\usepackage{graphicx}
% Images are located in a directory called images in the current directory.
\graphicspath{{./images/}}

% Make links look better by default.
% See: https://tex.stackexchange.com/questions/823/remove-ugly-borders-around-clickable-cross-references-and-hyperlinks
\usepackage[hidelinks]{hyperref}
\usepackage{xcolor}
\hypersetup{
	colorlinks,
	linkcolor={red!50!black},
	citecolor={blue!50!black},
	urlcolor={blue!80!black}
}


% Code Listings. Source:
% https://stackoverflow.com/questions/3175105/inserting-code-in-this-latex-document-with-indentation
\usepackage{listings}
\usepackage{color}

\definecolor{dkgreen}{rgb}{0,0.6,0}
\definecolor{gray}{rgb}{0.5,0.5,0.5}
\definecolor{mauve}{rgb}{0.58,0,0.82}

\lstset{frame=tb,
	language=Java,
	aboveskip=3mm,
	belowskip=3mm,
	showstringspaces=false,
	columns=flexible,
	basicstyle={\small\ttfamily},
	numbers=none,
	numberstyle=\tiny\color{gray},
	keywordstyle=\color{blue},
	commentstyle=\color{dkgreen},
	stringstyle=\color{mauve},
	breaklines=true,
	breakatwhitespace=true,
	tabsize=3
}

\newcommand{\prob}{\text{P}}
%\newcommand{\complement}{\mathsf{c}}

% Define an environment called "ex" (for Exercise) so that I can do: \begin{ex}{1.5}...\end{ex}
\newenvironment{ex}[2][Exercise]
{\par\medskip\noindent \textbf{#1 #2.}}
{\medskip}

% Define a solution environment, similar to ex (exercise) environment.
\newenvironment{sol}[1][Solution]
{\par\medskip\noindent \textbf{#1.} }
{\medskip}

\begin{document}
	\noindent Sergio E. Garcia Tapia \hfill
	
	\noindent \emph{Algorithms} by Sedgewick and Wayne (4th edition) \cite{sedgewick_wayne}\hfill
	
	\noindent October 25th, 2024\hfill 
	\section*{2.3: Quicksort}
	\begin{ex}{1}
		Show, in the style of the trace given with \texttt{partition()}, how that method
		partitions the array \texttt{E A S Y Q U E S T I O N}.
	\end{ex}
	\begin{sol}
		We set \texttt{lo} to \texttt{0} which means \texttt{E} is the partition key.
		Then we start with \texttt{i = lo} and \texttt{j = 12} (this is \texttt{hi = 11}
		plus 1). We use the \texttt{i} index to scan from the left, starting with
		\texttt{++i} (and hence \texttt{lo + 1}) and comparing it against \texttt{a[lo]}
		which is \texttt{E}. If we encounter something equal to or larger than \texttt{E},
		we stop. Similarly, we scan from the right with index j, starting with \texttt{--j}
		(meaning \texttt{hi - 1} or \texttt{11} is the first index) and then continue
		until we encounter a key smaller or equal to the partition key \texttt{E}.
		\begin{center}
			\begin{tabular}{c|cc|cccccccccccc}
				{} & {} & {} & \multicolumn{12}{c}{\texttt{a[]}}\\
				{} & \texttt{i} & \texttt{j} & 0 & 1 & 2 & 3 & 4 & 5 & 6 & 7 & 8 & 9 & 10 & 11\\
				\hline
				{\color{red} initial values}
				& 0 & 12
				& {\color{red}E} & A & S & Y & Q & U & E & S & T & I & O & N\\
				
				{\color{red} scan left, scan right}
				& 2 & 6 
				& {\color{gray}E} & A & S & {\color{gray} Y} & {\color{gray} Q} & {\color{gray} U}
				& E & S & T & I & O & N\\
				
				{\color{red} exchange}
				& {\color{red} 2} & {\color{red} 6 }
				& {\color{gray}E} & {\color{gray}A} & {\color{red}E} & {\color{gray} Y} & {\color{gray} Q} & {\color{gray} U} & {\color{red}S} & {\color{gray}S}
				& {\color{gray}T} &{\color{gray}I} & {\color{gray}O} & {\color{gray}N}\\
				
				{\color{red} scan left, scan right}
				& {\color{red} 3} & {\color{red} 2 }
				& {\color{gray}E} & {\color{gray}A} & {\color{black}$\underleftarrow{\text{E}}$} & {\color{black} $\underrightarrow{\text{Y}}$} & {\color{black} $\underleftarrow{\text{Q}}$} & {\color{black} $\underleftarrow{\text{U}}$} & {\color{gray}S} & {\color{gray}S}
				& {\color{gray}T} &{\color{gray}I} & {\color{gray}O} & {\color{gray}N}\\
				
				{\color{red} final exchange}
				& {\color{red} 3} & {\color{red} 2 }
				& {\color{red}E} & {\color{gray}A} & {\color{red} E} & {\color{gray} Y} & {\color{gray} Q} & {\color{gray} U} & {\color{gray}S} & {\color{gray}S}
				& {\color{gray}T} &{\color{gray}I} & {\color{gray}O} & {\color{gray}N}\\
				
				{\color{red} result}
				& {} & {\color{red} 2 }
				& {\color{black}E} & {\color{black}A} & {\color{black} E} & {\color{black} Y} & {\color{black} Q} & {\color{black} U} & {\color{black}S} & {\color{black}S}
				& {\color{black}T} &{\color{black}I} & {\color{black}O} & {\color{black}N}\\
			\end{tabular}
		\end{center}
	\end{sol}
	\begin{ex}{2}
		Show, in the style of the quicksort trace given in this section, how quicksort sorts
		the array \texttt{E A S Y Q U E S T I O N} (for the purposes of this exercise, ignore
		the initial shuffle).
	\end{ex}
	\begin{sol}
		\begin{center}
			\begin{tabular}{ccc|cccccccccccc}
				{} & {} & {} & \multicolumn{12}{c}{\texttt{a[]}}\\
				\texttt{lo} & \texttt{j} & \texttt{hi} & 0 & 1 & 2 & 3 & 4 & 5 & 6 & 7 & 8 & 9 & 10 & 11\\
				
				\hline
				
				{} & {} & {} & {\color{black}E} & {\color{black} A} & {\color{black} S} & {\color{black} Y} & {\color{black} Q} & {\color{black} U} & {\color{black} E} & {\color{black} S} & {\color{black} T} & {\color{black} I} & {\color{black} O}
				& {\color{black} N}\\
				
				0 & {\color{red}2} & 11 & {\color{black}E} & {\color{black} A} & {\color{red} E} & {\color{black} Y} & {\color{black} Q} & {\color{black} U} & {\color{black} S} & {\color{black} S} & {\color{black} T} & {\color{black} I} & {\color{black} O}
				& {\color{black} N}\\
				
				0 & {\color{red}1} & 1 & {\color{black}A} & {\color{red} E} & {\color{gray} E} & {\color{gray} Y} & {\color{gray} Q} & {\color{gray} U} & {\color{gray} S} & {\color{gray} S} & {\color{gray} T} & {\color{gray} I} & {\color{gray} O}
				& {\color{gray} N}\\
				
				{\color{gray}0} & {} & {\color{gray}0} & {\color{red}A} & {\color{gray} E} & {\color{gray} E} & {\color{gray} Y} & {\color{gray} Q} & {\color{gray} U} & {\color{gray} S} & {\color{gray} S} & {\color{gray} T} & {\color{gray} I} & {\color{gray} O}
				& {\color{gray} N}\\
				
				{\color{black}3} & {\color{red}11} & {\color{black}11} & {\color{gray}A} & {\color{gray} E} & {\color{gray} E} & {\color{black} N} & {\color{black} Q} & {\color{black} U} & {\color{black} S} & {\color{black} S} & {\color{black} T} & {\color{black} I} & {\color{black} O} & {\color{red} Y}\\
				
				{\color{black}3} & {\color{red}4} & {\color{black}10} & {\color{gray}A} & {\color{gray} E} & {\color{gray} E} & {\color{black} I} & {\color{red} N} & {\color{black} U} & {\color{black} S} & {\color{black} S} & {\color{black} T} & {\color{black} Q} & {\color{black} O} & {\color{gray} Y}\\
				
				{\color{gray}3} & {} & {\color{gray}3} & {\color{gray}A} & {\color{gray} E} & {\color{gray} E} & {\color{red} I} & {\color{gray} N} & {\color{gray} U} & {\color{gray} S} & {\color{gray} S} & {\color{gray} T} & {\color{gray} Q} & {\color{gray} O} & {\color{gray} Y}\\
				
				{\color{black}5} & {\color{red}10} & {\color{black}10} & {\color{gray}A} & {\color{gray} E} & {\color{gray} E} & {\color{gray} I} & {\color{gray} N} & {\color{black} O} & {\color{black} S} & {\color{black} S} & {\color{black} T} & {\color{black} Q} & {\color{red} U} & {\color{gray} Y}\\
				
				{\color{black}5} & {\color{red}5} & {\color{black}9} & {\color{gray}A} & {\color{gray} E} & {\color{gray} E} & {\color{gray} I} & {\color{gray} N} & {\color{red} O} & {\color{black} S} & {\color{black} S} & {\color{black} T} & {\color{black} Q} & {\color{gray} U} & {\color{gray} Y}\\
				
				{\color{black}6} & {\color{red}7} & {\color{black}9} & {\color{gray}A} & {\color{gray} E} & {\color{gray} E} & {\color{gray} I} & {\color{gray} N} & {\color{gray} O} & {\color{black} Q} & {\color{red} S} & {\color{black} T} & {\color{black} S} & {\color{gray} U} & {\color{gray} Y}\\
				
				{\color{gray}6} & {\color{red}} & {\color{gray}6} & {\color{gray}A} & {\color{gray} E} & {\color{gray} E} & {\color{gray} I} & {\color{gray} N} & {\color{gray} O} & {\color{red} Q} & {\color{gray} S} & {\color{gray} T} & {\color{gray} S} & {\color{gray} U} & {\color{gray} Y}\\
				
				{\color{black}8} & {\color{red}9} & {\color{black} 9} & {\color{gray}A} & {\color{gray} E} & {\color{gray} E} & {\color{gray} I} & {\color{gray} N} & {\color{gray} O} & {\color{gray} Q} & {\color{gray} S} & {\color{black} S} & {\color{red}T} & {\color{gray} U} & {\color{gray} Y}\\
				
				{\color{gray}8} & {\color{red}} & {\color{gray} 8} & {\color{gray}A} & {\color{gray} E} & {\color{gray} E} & {\color{gray} I} & {\color{gray} N} & {\color{gray} O} & {\color{gray} Q} & {\color{gray} S} & {\color{red} S} & {\color{gray}T} & {\color{gray} U} & {\color{gray} Y}\\
				
				{} & {} & {} & {\color{gray}A} & {\color{black} E} & {\color{black} E} & {\color{black} I} & {\color{black} N} & {\color{black} O} & {\color{black} Q} & {\color{black} S} & {\color{black} S} & {\color{black}T} & {\color{black} U} & {\color{black} Y}\\
			\end{tabular}
		\end{center}
	\end{sol}
	\begin{ex}{3}
		What is the maximum number of times during the execution of \texttt{Quick.sort()}
		that that the largest item can be exchanged, for an array of length \texttt{n}?
	\end{ex}
	\begin{sol}
		Suppose that the array's entries are all distinct. If the largest key is at the end,
		then no exchange will ever occur:
		\begin{lstlisting}[language={}]
* * * * * L
		\end{lstlisting}
		Here, \texttt{L} stands for largest. If the largest key is at the beginning, then it
		will be the first partition key, and one exchange will occur.
		\begin{lstlisting}[language={}]
L * * * * *
* * * * * L
		\end{lstlisting}
		Thus, at least one exchange occurs. To explore whether more than one can occur,
		we can consider the case when it is not the partition item, and not at the last position.
		
		If the partition item is not the largest key, then right scan will never end due to
		a comparison with the largest largest key. This is because the right scan continues as
		long as a key larger than the partition key is encountered. One consequence
		is that the largest item will not be involved in the final exchange, because
		\texttt{j} will always move past the largest key if it encounters it. Therefore,
		all exchanges with the largest item occur when the right scan encounters the largest
		key and the scan indices have not crossed. In that case, since it is encountered
		by the right scan, any swap exchanging it with the key encountered on the left
		scan will move the largest key forward. That is, the largest key is never moved
		to a lower position. Since we're assuming that it is not the partitioning key
		and not at the end, this suggest there's at most $n-2$ moves forward.
		
		Though $n-2$ is an upper bound on the number of exchanges, I don't know if it is
		the maximum because I cannot think of a distribution of the keys that would cause
		all $n-2$ exchanges to occur (or for that matter, a scenario where more than
		2 exchanges occur). For example, it could be $2$ times if it's to the right
		of the partitioning key, and the value at the end of the array is larger than
		the partitioning key.
		\begin{lstlisting}[language={}]
2 L * * 1 3
2 1 * * L 3
		\end{lstlisting}
		In that case, the largest key is exchanged up, but not to the last position, so
		that at least one more exchange will be at a different partitioning stage in order
		to place it at the end. See \href{https://stackoverflow.com/a/47702552}{user named Panic
		on StackOverflow} who claims the maximum is $\lfloor n / 2\rfloor$ and gives an example
		for $n=10$.
	\end{sol}
	\begin{ex}{4}
		Suppose that the initial random shuffle is omitted. Give six arrays of ten elements
		for which \texttt{Quick.sort()} uses the worst-case number of compares?
	\end{ex}
	\begin{sol}
		\begin{lstlisting}[language={}]
// sorted array, distinct keys
1 2 3 4 5 6 7 8 9 10

// inversely sorted, distinct keys
10 9 8 7 6 5 4 3 2 1

// largest followed by increasing sequence
10 1 2 3 4 5 6 7 8 9

// smallest followed by decreasing sequence
1 10 9 8 7 6 5 4 3 2

// bitonic (increase then decrease)
1 2 3 4 5 10 9 8 7 6

		\end{lstlisting}
	\end{sol}
	\begin{ex}{5}
		Give a code fragment that sorts an array that is known to consist of items having
		just two distinct keys.
	\end{ex}
	\begin{sol}
		The 3-way partitioning method would work for this. However, I came up with
		the following:
		\begin{lstlisting}
			// Find first index of largest key.
			int i = 0;
			while (i < a.length - 1 && !less(a[i + 1], a[i]))
				i++;
			for (int j = i + 1; j < a.length; j++) {
				if (less(a[j], a[j - 1]))
					exch(a, i++, j);
			}
		\end{lstlisting}
	\end{sol}
	\begin{ex}{6}
		Write a program to compute the exact value of $C_n$, and compare the exact value
		with the approximation $2n\ln n$, for $n=100$, $1,000$, and $10,000$.
	\end{ex}
	\begin{sol}
		See the \texttt{com.segarciat.algs4.ch2.sec3.ex06.QuicksortCompares} class.
	\end{sol}
	\begin{ex}{8}
		About how many compares will \texttt{Quick.sort()} make when sorting an array of $n$
		items that are all equal?
	\end{ex}
	\begin{sol}
		Consider partition call. Since the partition key is always equal to the item
		it is compared against, the left and right scans always move by 1 before
		a swap is necessary. As a result, the number of times that the left scan
		index increases and the number of times the right scan index decreases are
		within 1 of one another (depending the parity of $n$). Thus, the scan indices
		cross around the center of the array, and partition index \texttt{j} falls
		around the middle. At this point, about $\sim n$ compares have occurred,
		once for each compare against the left pointer and one for each compare
		against the right pointer. Since all keys are equal, the process now proceeds
		by induction when it is cut in half, now yielding about $n/2$ compares for
		each half. If we see it as a binary tree, then its height is about $\sim \lg n$
		and at each level there's about $n$  compares. Thus we get $\sim n\lg n$
		compares overall.
	\end{sol}
	\begin{ex}{9}
		Explain what happens when \texttt{Quick.sort()} is run on an array having items
		with just two distinct keys, and then explain what happens when it is run on an
		array having just three distinct keys.
	\end{ex}
	\begin{sol}
		First consider the case with two distinct keys (for simplicity, say their values
		are 1 and 2). Suppose the smallest key is at the beginning, meaning a 1. The right
		scan index \texttt{i} will stop at every comparison. The left scan index \texttt{j}
		will stop each time a 1 is encountered, causing a swap to occur. By the time the 
		scan indices cross, the array will sorted. However, the sorting algorithm will
		continue, and it will now work on two halves that both consist of only equal
		items. The sort the continues as in Exercise 8. A similar case occurs when
		sorting the pivot is the largest item.
		
		When it consists of only three distinct items, call them 1, 2, 3. Then
		there's a few cases:
		\begin{itemize}
			\item If the first pivot is 1, then once again the left scan index always stops,
			and so does the right one. The end result of this first partition
			\texttt{a[lo..j]} has all the 1s, and\texttt{a[j+1..hi]} has all the 2s
			and 3s. Now this continues as in the 2 item case.
			\item If the first pivot is 3, then the right scan index stops only
			when a 3 is encountered. The left scan index stops every time. By
			the end, \texttt{a[j..hi]} has all the 3s,  and \texttt{a[lo..j-1]}
			has the 1s and 2s. Now we proceed as in the 2 item case.
			\item If the first pivot is 2, then by the end, \texttt{a[lo..j-1]}
			has 1s and 2s, and \texttt{a[j+1..hi]} has 2s and 3s. Thus both cases
			proceed as int he 2 element case.
		\end{itemize}
	\end{sol}
	\pagebreak
	\printbibliography
\end{document}