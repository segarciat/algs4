\documentclass[12pt, a4paper]{article}

\usepackage[utf8]{inputenc}
% Limit the page margin to only 1 inch.
\usepackage[margin=1in]{geometry}

%Imports biblatex package
\usepackage[
backend=biber,
style=alphabetic
]{biblatex}
\addbibresource{../../algs4e.bib}

% Enables the `align' environment.
\usepackage{amsmath}
% Provides useful environments, such as:
% - \begin{proof} ...\end{proof}
\usepackage{amsthm}
% Enables using \mathbb{}, for example \mathbb{N} for the set of natural numbers.
\usepackage{amssymb}

% Allows using letters in enumerate list environment. Use, for example:
%\begin{enumerate}[label=(\alph*)]
% ...
%\end{enumerate}
\usepackage[inline]{enumitem}

% Enable importing external graphic files and provides useful commannds, like \graphicspath{}
\usepackage{graphicx}
% Images are located in a directory called images in the current directory.
\graphicspath{{./images/}}

% Make links look better by default.
% See: https://tex.stackexchange.com/questions/823/remove-ugly-borders-around-clickable-cross-references-and-hyperlinks
\usepackage[hidelinks]{hyperref}
\usepackage{xcolor}
\hypersetup{
	colorlinks,
	linkcolor={red!50!black},
	citecolor={blue!50!black},
	urlcolor={blue!80!black}
}


% Code Listings. Source:
% https://stackoverflow.com/questions/3175105/inserting-code-in-this-latex-document-with-indentation
\usepackage{listings}
\usepackage{color}

\definecolor{dkgreen}{rgb}{0,0.6,0}
\definecolor{gray}{rgb}{0.5,0.5,0.5}
\definecolor{mauve}{rgb}{0.58,0,0.82}

\lstset{frame=tb,
	language=Java,
	aboveskip=3mm,
	belowskip=3mm,
	showstringspaces=false,
	columns=flexible,
	basicstyle={\small\ttfamily},
	numbers=none,
	numberstyle=\tiny\color{gray},
	keywordstyle=\color{blue},
	commentstyle=\color{dkgreen},
	stringstyle=\color{mauve},
	breaklines=true,
	breakatwhitespace=true,
	tabsize=3
}

\newcommand{\prob}{\text{P}}
%\newcommand{\complement}{\mathsf{c}}

% Define an environment called "ex" (for Exercise) so that I can do: \begin{ex}{1.5}...\end{ex}
\newenvironment{ex}[2][Exercise]
{\par\medskip\noindent \textbf{#1 #2.}}
{\medskip}

% Define a solution environment, similar to ex (exercise) environment.
\newenvironment{sol}[1][Solution]
{\par\medskip\noindent \textbf{#1.} }
{\medskip}

\begin{document}
	\noindent Sergio E. Garcia Tapia \hfill
	
	\noindent \emph{Algorithms} by Sedgewick and Wayne (4th edition) \cite{sedgewick_wayne}\hfill
	
	\noindent October 25th, 2024\hfill 
	\section*{2.3: Quicksort}
	\begin{ex}{1}
		Show, in the style of the trace given with \texttt{partition()}, how that method
		partitions the array \texttt{E A S Y Q U E S T I O N}.
	\end{ex}
	\begin{sol}
		We set \texttt{lo} to \texttt{0} which means \texttt{E} is the partition key.
		Then we start with \texttt{i = lo} and \texttt{j = 12} (this is \texttt{hi = 11}
		plus 1). We use the \texttt{i} index to scan from the left, starting with
		\texttt{++i} (and hence \texttt{lo + 1}) and comparing it against \texttt{a[lo]}
		which is \texttt{E}. If we encounter something equal to or larger than \texttt{E},
		we stop. Similarly, we scan from the right with index j, starting with \texttt{--j}
		(meaning \texttt{hi - 1} or \texttt{11} is the first index) and then continue
		until we encounter a key smaller or equal to the partition key \texttt{E}.
		\begin{center}
			\begin{tabular}{c|cc|cccccccccccc}
				{} & {} & {} & \multicolumn{12}{c}{\texttt{a[]}}\\
				{} & \texttt{i} & \texttt{j} & 0 & 1 & 2 & 3 & 4 & 5 & 6 & 7 & 8 & 9 & 10 & 11\\
				\hline
				{\color{red} initial values}
				& 0 & 12
				& {\color{red}E} & A & S & Y & Q & U & E & S & T & I & O & N\\
				
				{\color{red} scan left, scan right}
				& 2 & 6 
				& {\color{gray}E} & A & S & {\color{gray} Y} & {\color{gray} Q} & {\color{gray} U}
				& E & S & T & I & O & N\\
				
				{\color{red} exchange}
				& {\color{red} 2} & {\color{red} 6 }
				& {\color{gray}E} & {\color{gray}A} & {\color{red}E} & {\color{gray} Y} & {\color{gray} Q} & {\color{gray} U} & {\color{red}S} & {\color{gray}S}
				& {\color{gray}T} &{\color{gray}I} & {\color{gray}O} & {\color{gray}N}\\
				
				{\color{red} scan left, scan right}
				& {\color{red} 3} & {\color{red} 2 }
				& {\color{gray}E} & {\color{gray}A} & {\color{black}$\underleftarrow{\text{E}}$} & {\color{black} $\underrightarrow{\text{Y}}$} & {\color{black} $\underleftarrow{\text{Q}}$} & {\color{black} $\underleftarrow{\text{U}}$} & {\color{gray}S} & {\color{gray}S}
				& {\color{gray}T} &{\color{gray}I} & {\color{gray}O} & {\color{gray}N}\\
				
				{\color{red} final exchange}
				& {\color{red} 3} & {\color{red} 2 }
				& {\color{red}E} & {\color{gray}A} & {\color{red} E} & {\color{gray} Y} & {\color{gray} Q} & {\color{gray} U} & {\color{gray}S} & {\color{gray}S}
				& {\color{gray}T} &{\color{gray}I} & {\color{gray}O} & {\color{gray}N}\\
				
				{\color{red} result}
				& {} & {\color{red} 2 }
				& {\color{black}E} & {\color{black}A} & {\color{black} E} & {\color{black} Y} & {\color{black} Q} & {\color{black} U} & {\color{black}S} & {\color{black}S}
				& {\color{black}T} &{\color{black}I} & {\color{black}O} & {\color{black}N}\\
			\end{tabular}
		\end{center}
	\end{sol}
	\begin{ex}{2}
		Show, in the style of the quicksort trace given in this section, how quicksort sorts
		the array \texttt{E A S Y Q U E S T I O N} (for the purposes of this exercise, ignore
		the initial shuffle).
	\end{ex}
	\begin{sol}
		\begin{center}
			\begin{tabular}{ccc|cccccccccccc}
				{} & {} & {} & \multicolumn{12}{c}{\texttt{a[]}}\\
				\texttt{lo} & \texttt{j} & \texttt{hi} & 0 & 1 & 2 & 3 & 4 & 5 & 6 & 7 & 8 & 9 & 10 & 11\\
				
				\hline
				
				{} & {} & {} & {\color{black}E} & {\color{black} A} & {\color{black} S} & {\color{black} Y} & {\color{black} Q} & {\color{black} U} & {\color{black} E} & {\color{black} S} & {\color{black} T} & {\color{black} I} & {\color{black} O}
				& {\color{black} N}\\
				
				0 & {\color{red}2} & 11 & {\color{black}E} & {\color{black} A} & {\color{red} E} & {\color{black} Y} & {\color{black} Q} & {\color{black} U} & {\color{black} S} & {\color{black} S} & {\color{black} T} & {\color{black} I} & {\color{black} O}
				& {\color{black} N}\\
				
				0 & {\color{red}1} & 1 & {\color{black}A} & {\color{red} E} & {\color{gray} E} & {\color{gray} Y} & {\color{gray} Q} & {\color{gray} U} & {\color{gray} S} & {\color{gray} S} & {\color{gray} T} & {\color{gray} I} & {\color{gray} O}
				& {\color{gray} N}\\
				
				{\color{gray}0} & {} & {\color{gray}0} & {\color{red}A} & {\color{gray} E} & {\color{gray} E} & {\color{gray} Y} & {\color{gray} Q} & {\color{gray} U} & {\color{gray} S} & {\color{gray} S} & {\color{gray} T} & {\color{gray} I} & {\color{gray} O}
				& {\color{gray} N}\\
				
				{\color{black}3} & {\color{red}11} & {\color{black}11} & {\color{gray}A} & {\color{gray} E} & {\color{gray} E} & {\color{black} N} & {\color{black} Q} & {\color{black} U} & {\color{black} S} & {\color{black} S} & {\color{black} T} & {\color{black} I} & {\color{black} O} & {\color{red} Y}\\
				
				{\color{black}3} & {\color{red}4} & {\color{black}10} & {\color{gray}A} & {\color{gray} E} & {\color{gray} E} & {\color{black} I} & {\color{red} N} & {\color{black} U} & {\color{black} S} & {\color{black} S} & {\color{black} T} & {\color{black} Q} & {\color{black} O} & {\color{gray} Y}\\
				
				{\color{gray}3} & {} & {\color{gray}3} & {\color{gray}A} & {\color{gray} E} & {\color{gray} E} & {\color{red} I} & {\color{gray} N} & {\color{gray} U} & {\color{gray} S} & {\color{gray} S} & {\color{gray} T} & {\color{gray} Q} & {\color{gray} O} & {\color{gray} Y}\\
				
				{\color{black}5} & {\color{red}10} & {\color{black}10} & {\color{gray}A} & {\color{gray} E} & {\color{gray} E} & {\color{gray} I} & {\color{gray} N} & {\color{black} O} & {\color{black} S} & {\color{black} S} & {\color{black} T} & {\color{black} Q} & {\color{red} U} & {\color{gray} Y}\\
				
				{\color{black}5} & {\color{red}5} & {\color{black}9} & {\color{gray}A} & {\color{gray} E} & {\color{gray} E} & {\color{gray} I} & {\color{gray} N} & {\color{red} O} & {\color{black} S} & {\color{black} S} & {\color{black} T} & {\color{black} Q} & {\color{gray} U} & {\color{gray} Y}\\
				
				{\color{black}6} & {\color{red}7} & {\color{black}9} & {\color{gray}A} & {\color{gray} E} & {\color{gray} E} & {\color{gray} I} & {\color{gray} N} & {\color{gray} O} & {\color{black} Q} & {\color{red} S} & {\color{black} T} & {\color{black} S} & {\color{gray} U} & {\color{gray} Y}\\
				
				{\color{gray}6} & {\color{red}} & {\color{gray}6} & {\color{gray}A} & {\color{gray} E} & {\color{gray} E} & {\color{gray} I} & {\color{gray} N} & {\color{gray} O} & {\color{red} Q} & {\color{gray} S} & {\color{gray} T} & {\color{gray} S} & {\color{gray} U} & {\color{gray} Y}\\
				
				{\color{black}8} & {\color{red}9} & {\color{black} 9} & {\color{gray}A} & {\color{gray} E} & {\color{gray} E} & {\color{gray} I} & {\color{gray} N} & {\color{gray} O} & {\color{gray} Q} & {\color{gray} S} & {\color{black} S} & {\color{red}T} & {\color{gray} U} & {\color{gray} Y}\\
				
				{\color{gray}8} & {\color{red}} & {\color{gray} 8} & {\color{gray}A} & {\color{gray} E} & {\color{gray} E} & {\color{gray} I} & {\color{gray} N} & {\color{gray} O} & {\color{gray} Q} & {\color{gray} S} & {\color{red} S} & {\color{gray}T} & {\color{gray} U} & {\color{gray} Y}\\
				
				{} & {} & {} & {\color{gray}A} & {\color{black} E} & {\color{black} E} & {\color{black} I} & {\color{black} N} & {\color{black} O} & {\color{black} Q} & {\color{black} S} & {\color{black} S} & {\color{black}T} & {\color{black} U} & {\color{black} Y}\\
			\end{tabular}
		\end{center}
	\end{sol}
	\begin{ex}{3}
		What is the maximum number of times during the execution of \texttt{Quick.sort()}
		that that the largest item can be exchanged, for an array of length \texttt{n}?
	\end{ex}
	\begin{sol}
		Suppose that the array's entries are all distinct. If the largest key is at the end,
		then no exchange will ever occur:
		\begin{lstlisting}[language={}]
* * * * * L
		\end{lstlisting}
		Here, \texttt{L} stands for largest. If the largest key is at the beginning, then it
		will be the first partition key, and one exchange will occur.
		\begin{lstlisting}[language={}]
L * * * * *
* * * * * L
		\end{lstlisting}
		Thus, at least one exchange occurs. To explore whether more than one can occur,
		we can consider the case when it is not the partition item, and not at the last position.
		
		If the partition item is not the largest key, then right scan will never end due to
		a comparison with the largest largest key. This is because the right scan continues as
		long as a key larger than the partition key is encountered. One consequence
		is that the largest item will not be involved in the final exchange, because
		\texttt{j} will always move past the largest key if it encounters it. Therefore,
		all exchanges with the largest item occur when the right scan encounters the largest
		key and the scan indices have not crossed. In that case, since it is encountered
		by the right scan, any swap exchanging it with the key encountered on the left
		scan will move the largest key forward. That is, the largest key is never moved
		to a lower position. Since we're assuming that it is not the partitioning key
		and not at the end, this suggest there's at most $n-2$ moves forward.
		
		Though $n-2$ is an upper bound on the number of exchanges, I don't know if it is
		the maximum because I cannot think of a distribution of the keys that would cause
		all $n-2$ exchanges to occur (or for that matter, a scenario where more than
		2 exchanges occur). For example, it could be $2$ times if it's to the right
		of the partitioning key, and the value at the end of the array is larger than
		the partitioning key.
		\begin{lstlisting}[language={}]
2 L * * 1 3
2 1 * * L 3
		\end{lstlisting}
		In that case, the largest key is exchanged up, but not to the last position, so
		that at least one more exchange will be at a different partitioning stage in order
		to place it at the end. See \href{https://stackoverflow.com/a/47702552}{user named Panic
		on StackOverflow} who claims the maximum is $\lfloor n / 2\rfloor$ and gives an example
		for $n=10$.
	\end{sol}
	\begin{ex}{4}
		Suppose that the initial random shuffle is omitted. Give six arrays of ten elements
		for which \texttt{Quick.sort()} uses the worst-case number of compares?
	\end{ex}
	\begin{sol}
		\begin{lstlisting}[language={}]
// sorted array, distinct keys
1 2 3 4 5 6 7 8 9 10

// inversely sorted, distinct keys
10 9 8 7 6 5 4 3 2 1

// largest followed by increasing sequence
10 1 2 3 4 5 6 7 8 9

// smallest followed by decreasing sequence
1 10 9 8 7 6 5 4 3 2

// bitonic (increase then decrease)
1 2 3 4 5 10 9 8 7 6

		\end{lstlisting}
	\end{sol}
	\begin{ex}{5}
		Give a code fragment that sorts an array that is known to consist of items having
		just two distinct keys.
	\end{ex}
	\begin{sol}
		The 3-way partitioning method would work for this. However, I came up with
		the following:
		\begin{lstlisting}
			// Find first index of largest key.
			int i = 0;
			while (i < a.length - 1 && !less(a[i + 1], a[i]))
				i++;
			for (int j = i + 1; j < a.length; j++) {
				if (less(a[j], a[j - 1]))
					exch(a, i++, j);
			}
		\end{lstlisting}
	\end{sol}
	\begin{ex}{6}
		Write a program to compute the exact value of $C_n$, and compare the exact value
		with the approximation $2n\ln n$, for $n=100$, $1,000$, and $10,000$.
	\end{ex}
	\begin{sol}
		See the \texttt{com.segarciat.algs4.ch2.sec3.ex06.QuicksortCompares} class.
	\end{sol}
	\begin{ex}{7}
		Find the expected number of sub-arrays of size $0$, $1$, and $2$ when quicksort
		is used to sort an array of $N$ items with distinct keys. If you are mathematically
		inclined, do the math; if not, run experiments to develop hypotheses.
	\end{ex}
	\begin{sol}
		One idea I had was to call $T(N)$ the expected number of arrays of size
		$0$, $1$, and $2$ for an array of $N$ distinct items. We know that
		$T(0)=1$, since the array has size $0$ and quicksort will not be
		called recursively to partition it. Similarly, $T(1)=1$. For
		$T(2)$, if we consider the original array of size $2$ to be sub-array,
		then at $T(2)\geq 1$. When quicksort encounters such an array,
		it will always partition it into an array of size 0 and an array of size 1.
		Thus, $T(2)=3$.
		
		Now consider the case where $N\geq 3$. We know \texttt{Quick.sort()}
		partitions the array at each step. For example, if $j=0$ is the first
		index chosen after \texttt{partition()} is called, then this leads to
		recursively calls to sort \texttt{a[0..-1]} and \texttt{a[1..N-1]}, which
		are arrays of size 0 and size $N-1$, respectively. Similarly, if $j=1$
		is chosen as the first partition index instead, then the array is
		split into \texttt{a[0..0]} and \texttt{a[2..N-1]}, which are arrays
		of size $1$ and $N-2$, respectively. Continuing this way, for $N\geq 3$, we
		can express $T(N)$ as a recurrence by taking the average as we break it up as follows:
		\begin{align*}
			T(0)&=1\\
			T(1)&=1\\
			T(2)&=3\\
			T(N)&= \frac{1}{N}\left([T(0) + T(N-1)] + [T(1) + T(N-2)] + \cdots+[T(N-1) + T(0)]\right)\\
			&=\frac{1}{N}\sum_{j=0}^{N-1}[T(j)+T(N-j)]\\
			&=\frac{2}{N}\cdot \sum_{j=0}^{N-1}T(j),\quad N\geq 3
		\end{align*}
		Then, I followed \href{https://math.stackexchange.com/questions/2116763/recurrence-relation-with-summation-of-previous-terms}{Paul Sinclair's suggestion on this StackExchange post}
		and defined $S(N)=\sum_{j=0}^{N}T(j)$ in order to transfer my recurrence
		into one that would be easier to solve. For $N\geq 3$, this meant that
		\begin{align*}
			S(N)-S(N-1)&=\frac{2}{N}S(N-1),\quad N\geq 3\\
		\end{align*}
		Thus, the recurrence for $S(N)$ is
		\begin{align*}
			S(0) &= 1\\
			S(1) &= 2\\
			S(2) &= 5\\
			S(N)&=\frac{N+2}{N}S(N-1),\quad N\geq 3
		\end{align*}
		I did not know how to solve this recurrence but according to
		Wolframalpha, the solution turns out to be
		\begin{align*}
			S(N)&=\frac{1}{2}\cdot c\cdot (N+1)(N+2),\quad N\geq 3.
		\end{align*}
		By the recurrence relation for $S(N)$, we find that
		\begin{align*}
			S(3) &= \frac{3+2}{3}S(2)=\frac{5}{3}\cdot 5=\frac{25}{3}
		\end{align*}
		Thus, we can solve for $c$:
		\begin{align*}
			\frac{25}{3}=S(3)=\frac{1}{2}c\cdot (3+1)(3+2)=\frac{c}{2}\cdot (4)(5)=10c
		\end{align*}
		so
		\begin{align*}
			c = \frac{25}{30}=\frac{5}{6}.
		\end{align*}
		Now for $N\geq 4$, we can use the recurrence for $S(N)$ to get
		\begin{align*}
			T(N)&=S(N)-S(N-1)\\
			&=\frac{2}{N}S(N-1)\\
			&=\frac{2}{N}\cdot \frac{1}{2}\cdot c\cdot N(N+1)\\
			&=c(N+1)\\
			&=\frac{5}{6}(N+1)
		\end{align*}
		Thus, $T(N)$ is proportional to the $N$, the size of the array, and the
		constant of proportionality is $\frac{5}{6}$. Thus, quicksort will produce
		about $\sim \frac{5}{6}N$ sub-arrays of size 0, 1, and 2, on the average.
		
		See \texttt{com.segarciat.algs4.ch2.sec3.ex07.QuicksortCompares}, a program
		I wrote to count the number of arrays of size 2 or less and how it compares to
		the array size. Below is a sample output from a particular run:
		\begin{lstlisting}[language={}]
n = 2, Count of Arrays of size 2 or less = 3, Ratio of count to n = 1.50
n = 4, Count of Arrays of size 2 or less = 4, Ratio of count to n = 1.00
n = 8, Count of Arrays of size 2 or less = 8, Ratio of count to n = 0.98
n = 16, Count of Arrays of size 2 or less = 14, Ratio of count to n = 0.89
n = 32, Count of Arrays of size 2 or less = 28, Ratio of count to n = 0.87
n = 64, Count of Arrays of size 2 or less = 53, Ratio of count to n = 0.83
n = 128, Count of Arrays of size 2 or less = 105, Ratio of count to n = 0.82
n = 256, Count of Arrays of size 2 or less = 213, Ratio of count to n = 0.83
n = 512, Count of Arrays of size 2 or less = 422, Ratio of count to n = 0.82
n = 1024, Count of Arrays of size 2 or less = 859, Ratio of count to n = 0.84
n = 2048, Count of Arrays of size 2 or less = 1707, Ratio of count to n = 0.83
n = 4096, Count of Arrays of size 2 or less = 3404, Ratio of count to n = 0.83
n = 8192, Count of Arrays of size 2 or less = 6845, Ratio of count to n = 0.84
n = 16384, Count of Arrays of size 2 or less = 13676, Ratio of count to n = 0.83
n = 32768, Count of Arrays of size 2 or less = 27291, Ratio of count to n = 0.83
n = 65536, Count of Arrays of size 2 or less = 54660, Ratio of count to n = 0.83
n = 131072, Count of Arrays of size 2 or less = 109285, Ratio of count to n = 0.83
n = 262144, Count of Arrays of size 2 or less = 218374, Ratio of count to n = 0.83
n = 524288, Count of Arrays of size 2 or less = 436873, Ratio of count to n = 0.83
n = 1048576, Count of Arrays of size 2 or less = 873932, Ratio of count to n = 0.83
n = 2097152, Count of Arrays of size 2 or less = 1747220, Ratio of count to n = 0.83
		\end{lstlisting}
		These results support the mathematical analysis because $\frac{5}{6}=0.8\overline{3}$.
	\end{sol}
	\begin{ex}{8}
		About how many compares will \texttt{Quick.sort()} make when sorting an array of $n$
		items that are all equal?
	\end{ex}
	\begin{sol}
		Consider partition call. Since the partition key is always equal to the item
		it is compared against, the left and right scans always move by 1 before
		a swap is necessary. As a result, the number of times that the left scan
		index increases and the number of times the right scan index decreases are
		within 1 of one another (depending the parity of $n$). Thus, the scan indices
		cross around the center of the array, and partition index \texttt{j} falls
		around the middle. At this point, about $\sim n$ compares have occurred,
		once for each compare against the left pointer and one for each compare
		against the right pointer. Since all keys are equal, the process now proceeds
		by induction when it is cut in half, now yielding about $n/2$ compares for
		each half. If we see it as a binary tree, then its height is about $\sim \lg n$
		and at each level there's about $n$  compares. Thus we get $\sim n\lg n$
		compares overall.
	\end{sol}
	\begin{ex}{9}
		Explain what happens when \texttt{Quick.sort()} is run on an array having items
		with just two distinct keys, and then explain what happens when it is run on an
		array having just three distinct keys.
	\end{ex}
	\begin{sol}
		First consider the case with two distinct keys (for simplicity, say their values
		are 1 and 2). Suppose the smallest key is at the beginning, meaning a 1. The right
		scan index \texttt{i} will stop at every comparison. The left scan index \texttt{j}
		will stop each time a 1 is encountered, causing a swap to occur. By the time the 
		scan indices cross, the array will sorted. However, the sorting algorithm will
		continue, and it will now work on two halves that both consist of only equal
		items. The sort the continues as in Exercise 8. A similar case occurs when
		sorting the pivot is the largest item.
		
		When it consists of only three distinct items, call them 1, 2, 3. Then
		there's a few cases:
		\begin{itemize}
			\item If the first pivot is 1, then once again the left scan index always stops,
			and so does the right one. The end result of this first partition
			\texttt{a[lo..j]} has all the 1s, and\texttt{a[j+1..hi]} has all the 2s
			and 3s. Now this continues as in the 2 item case.
			\item If the first pivot is 3, then the right scan index stops only
			when a 3 is encountered. The left scan index stops every time. By
			the end, \texttt{a[j..hi]} has all the 3s,  and \texttt{a[lo..j-1]}
			has the 1s and 2s. Now we proceed as in the 2 item case.
			\item If the first pivot is 2, then by the end, \texttt{a[lo..j-1]}
			has 1s and 2s, and \texttt{a[j+1..hi]} has 2s and 3s. Thus both cases
			proceed as int he 2 element case.
		\end{itemize}
	\end{sol}
	\begin{ex}{10}
		\emph{Chebyshev's inequality} says that the probability that a random variable is
		more than $k$ standard deviations away from the mean is less than $1/k^2$.
		For $n=1$ million, use Chebyshev's inequality to bound the probability that the
		number of compares used by quicksort is more than 100 billion ($0.1n^2)$.
	\end{ex}
	\begin{sol}
		Let $X$ be a random variable whose values correspond to the number of compares
		used by quicksort when sorting an array of $n=1$ million values. We want
		to obtain a bound for the probability that $X\geq 100$  billion.
		
		By Proposition K in \cite{sedgewick_wayne}, quicksort uses $\sim 2n\ln n$
		compares on the average. Since $n=10^6$, this means the average is
		about $\mu=161,180,956$, which is about $162$ million.
		Thus we must determined how many standard deviations 100 billion is from the mean.
		According to the proof of Proposition L, the standard deviation of the number of
		compare is about $0.65n$. Therefore, by subtracting 1 million from the mean
		and dividing by $0.65n$, we get the number of standard deviations that 100 billion
		is from the mean. Thus
		\begin{align*}
			k &\approx \frac{0.1n^2-\mu}{\sigma}\\
			&=\frac{0.1n^2 - 2n\log n}{0.65n}\\
			&=\frac{0.1n-2\log n}{0.65}\\
			&\approx 153803
		\end{align*}
		By Chebyshev's inequality, the probability that $X\geq 10^{11}$ (100 billion)
		is less than $\frac{1}{k^2}\approx 4.23\times 10^{-11}$.
	\end{sol}
	\begin{ex}{11}
		Suppose that we can over items with keys equal to the partitioning item's key
		instead of stopping the scans when we encounter them. Show that the running time of
		this version of quicksort is quadratic for all  arrays with just a constant
		number of distinct keys.
	\end{ex}
	\begin{sol}
		\begin{proof}
			Suppose quicksort is called on an array of $n$ items, with $k$ distinct
			keys, and $k$ is a constant independent of the array size $n$.
			Here are assume $k\ll n$ (meaning $k$ is much less than $n$).
			
			Suppose first $k=1$ so that all keys are equal. When quicksort is called
			on such an array, the fact that all keys are equal and that we scan over
			items with equal to the partition key means that that the index \texttt{i}
			that scans to the right will reach \texttt{hi}, the right end of the sub-array.
			Similarly, the \texttt{j} index scanning towards the left will reach \texttt{lo}.
			Since the indices cross, the loop ends, and the pivot is exchanged with itself.
			That result is that \texttt{lo} becomes the partitioning index, and the
			method is called recursively with an array of size 0 and one of size
			\texttt{hi - lo}. Put another way, the sub-array only decreases by one in
			size and we perform the full amount of comparisons. This means the number
			of comparisons is
			\begin{align*}
				2n + 2(n-1)+\cdots+2=2\cdot \frac{(n+1)n}{2}=(n+1)n
			\end{align*}
			Thus the algorithm is quadratic for $k=1$. Now suppose that algorithm
			is quadratic for $k\geq  1$, and that there are $k+1$ distinct keys.
			If the item at index 0 is the smallest, then the first call results
			in a partition of index of 0 because the \texttt{j} index scanning from
			right to left is less than every key except itself, and skips keys equal
			to itself. Thus it reaches index 0. This continues happening until the
			value of \texttt{lo} is not the smallest key. A similar thing happens
			if the item at index 0 is the largest key. In both cases, the number
			of comparisons is proportional to the size of the sub-array and the sub-array
			size only decreases by 1 because there's an empty sub-array at each step.
			
			When this the pivot is no longer the smallest nor the largest element, 
			the sub-array is partitioned into two parts, both of which have less than
			$k+1$ distinct keys. By induction, this takes quadratic time.
		\end{proof}
	\end{sol}
	\begin{ex}{12}
		Show, in the style of the trace given with the code, how the 3-way quicksort
		first partitions the array \texttt{B A B A B A B A C A D A B R A}.
	\end{ex}
	\begin{sol}
		\begin{center}
			\begin{tabular}{ccc|ccccccccccccccc}
				\texttt{lt} & \texttt{i} & \texttt{gt} &
				0 & 1 & 2 & 3 & 4 & 5 & 6 & 7 & 8 & 9 & 10 & 11 & 12 & 13 & 14\\
				\hline
				
				0 & 1 & 14 &
				{\color{red}B} & A & B & A & B & A & B & A & C & A & D & A & B & R & A \\
				
				0 & 1 & 14 &
				{\color{black}B} & {\color{red}A} & {\color{gray}B} & {\color{gray}A} & {\color{gray}B} & {\color{gray}A} & {\color{gray}B} & {\color{gray}A} & {\color{gray}C} & {\color{gray}A} & {\color{gray}D} & {\color{gray}A} & {\color{gray}B} & {\color{gray}R} & {\color{black}A} \\
				
				1 & 2 & 14 &
				{\color{gray}A} & {\color{black}B} & {\color{red}B} & {\color{gray}A} & {\color{gray}B} & {\color{gray}A} & {\color{gray}B} & {\color{gray}A} & {\color{gray}C} & {\color{gray}A} & {\color{gray}D} & {\color{gray}A} & {\color{gray}B} & {\color{gray}R} & {\color{black}A} \\
				
				1 & 3 & 14 &
				{\color{gray}A} & {\color{black}B} & {\color{gray}B} & {\color{red}A} & {\color{gray}B} & {\color{gray}A} & {\color{gray}B} & {\color{gray}A} & {\color{gray}C} & {\color{gray}A} & {\color{gray}D} & {\color{gray}A} & {\color{gray}B} & {\color{gray}R} & {\color{black}A} \\
				
				2 & 4 & 14 &
				{\color{gray}A} & {\color{gray}A} & {\color{black}B} & {\color{gray}B} & {\color{red}B} & {\color{gray}A} & {\color{gray}B} & {\color{gray}A} & {\color{gray}C} & {\color{gray}A} & {\color{gray}D} & {\color{gray}A} & {\color{gray}B} & {\color{gray}R} & {\color{black}A} \\
				
				2 & 5 & 14 &
				{\color{gray}A} & {\color{gray}A} & {\color{black}B} & {\color{gray}B} & {\color{gray}B} & {\color{red}A} & {\color{gray}B} & {\color{gray}A} & {\color{gray}C} & {\color{gray}A} & {\color{gray}D} & {\color{gray}A} & {\color{gray}B} & {\color{gray}R} & {\color{black}A} \\
				
				3 & 6 & 14 &
				{\color{gray}A} & {\color{gray}A} & {\color{gray}A} & {\color{black}B} & {\color{gray}B} & {\color{gray}B} & {\color{red}B} & {\color{gray}A} & {\color{gray}C} & {\color{gray}A} & {\color{gray}D} & {\color{gray}A} & {\color{gray}B} & {\color{gray}R} & {\color{black}A} \\
				
				3 & 7 & 14 &
				{\color{gray}A} & {\color{gray}A} & {\color{gray}A} & {\color{black}B} & {\color{gray}B} & {\color{gray}B} & {\color{gray}B} & {\color{red}A} & {\color{gray}C} & {\color{gray}A} & {\color{gray}D} & {\color{gray}A} & {\color{gray}B} & {\color{gray}R} & {\color{black}A} \\
				
				4 & 8 & 14 &
				{\color{gray}A} & {\color{gray}A} & {\color{gray}A} & {\color{gray}A} & {\color{black}B} & {\color{gray}B} & {\color{gray}B} & {\color{gray}B} & {\color{red}C} & {\color{gray}A} & {\color{gray}D} & {\color{gray}A} & {\color{gray}B} & {\color{gray}R} & {\color{black}A} \\
				
				4 & 8 & 13 &
				{\color{gray}A} & {\color{gray}A} & {\color{gray}A} & {\color{gray}A} & {\color{black}B} & {\color{gray}B} & {\color{gray}B} & {\color{gray}B} & {\color{red}A} & {\color{gray}A} & {\color{gray}D} & {\color{gray}A} & {\color{gray}B} & {\color{black}R} & {\color{gray}C} \\
				
				5 & 9 & 13 &
				{\color{gray}A} & {\color{gray}A} & {\color{gray}A} & {\color{gray}A} & {\color{gray}A} & {\color{black}B} & {\color{gray}B} & {\color{gray}B} & {\color{gray}B} & {\color{red}A} & {\color{gray}D} & {\color{gray}A} & {\color{gray}B} & {\color{black}R} & {\color{gray}C} \\
				
				6 & 10 & 13 &
				{\color{gray}A} & {\color{gray}A} & {\color{gray}A} & {\color{gray}A} & {\color{gray}A} & {\color{gray}A} & {\color{black}B} & {\color{gray}B} & {\color{gray}B} & {\color{gray}B} & {\color{red}D} & {\color{gray}A} & {\color{gray}B} & {\color{black}R} & {\color{gray}C} \\
				
				6 & 10 & 12 &
				{\color{gray}A} & {\color{gray}A} & {\color{gray}A} & {\color{gray}A} & {\color{gray}A} & {\color{gray}A} & {\color{black}B} & {\color{gray}B} & {\color{gray}B} & {\color{gray}B} & {\color{red}R} & {\color{gray}A} & {\color{black}B} & {\color{gray}D} & {\color{gray}C} \\
				
				6 & 10 & 11 &
				{\color{gray}A} & {\color{gray}A} & {\color{gray}A} & {\color{gray}A} & {\color{gray}A} & {\color{gray}A} & {\color{black}B} & {\color{gray}B} & {\color{gray}B} & {\color{gray}B} & {\color{red}B} & {\color{black}A} & {\color{gray}R} & {\color{gray}D} & {\color{gray}C} \\
				
				6 & 11 & 11 &
				{\color{gray}A} & {\color{gray}A} & {\color{gray}A} & {\color{gray}A} & {\color{gray}A} & {\color{gray}A} & {\color{black}B} & {\color{gray}B} & {\color{gray}B} & {\color{gray}B} & {\color{gray}B} & {\color{red}A} & {\color{gray}R} & {\color{gray}D} & {\color{gray}C} \\
				
				7 & 12 & 11 &
				{\color{gray}A} & {\color{gray}A} & {\color{gray}A} & {\color{gray}A} & {\color{gray}A} & {\color{gray}A} & {\color{gray}A} & {\color{black}B} & {\color{gray}B} & {\color{gray}B} & {\color{gray}B} & {\color{black}B} & {\color{red}R} & {\color{gray}D} & {\color{gray}C} \\
				
				7 & 12 & 11 &
				{\color{black}A} & {\color{black}A} & {\color{black}A} & {\color{black}A} & {\color{black}A} & {\color{black}A} & {\color{black}A} & {\color{red}B} & {\color{red}B} & {\color{red}B} & {\color{red}B} & {\color{red}B} & {\color{black}R} & {\color{black}D} & {\color{black}C} \\
			\end{tabular}
		\end{center}
	\end{sol}
	\begin{ex}{13}
		What is the \emph{recursive depth} of quicksort, in the best, worst, and average cases?
		This is the size of the stack that the system needs to keep track of the recursive
		calls. See Exercise 2.3.20 for a way to guarantee that the recursive depth is logarithmic
		in the worst case.
	\end{ex}
	\begin{sol}
		In the worst case, where one array is empty for each recursive call
		after \texttt{partition()}, the depth is $n-1$. In the best case, the array
		is divided into roughly halves each time, so the stack size would be about
		$\sim \lg n$ in that case. Since the average case is closer to the best case
		than the worst case, the stack size is about $\sim \lg n$ in the average as well.
	\end{sol}
	\begin{ex}{14}
		Prove that when running quicksort on an array with $n$ distinct items, the probability
		of comparing the $i$th and $j$th smallest items is $2/(j-i+1)$. Then use this result
		to prove Proposition K in \cite{sedgewick_wayne}.
	\end{ex}
	\begin{sol}
		\begin{proof}
			Assume $j>i$. Then $j-i+1$ is the number of integer indices in range \texttt{i..j}.
			Since the \texttt{partition()} algorithm only compares the items
			against the current pivot, meaning the current value of \texttt{a[lo]}, it
			follows that the \texttt{i}th and \texttt{j}th smallest items are only compared
			if either one is a pivot of the current sub-array. Moreover, they can
			only be compared if they belong to the same sub-array. This means that if
			the \texttt{k}th smallest element is chosen as a partition key before both
			\texttt{i} and \texttt{j}, for $k$ in the open interval $(i, j)$, then
			the \texttt{i}th and \texttt{j}th smallest element will not be compared.
			
			Let $E$ be the event that the \texttt{i}th and \texttt{j}th smallest elements
			are compared, and let $F$ be the event that the first item from the sorted
			subsequence belonging to the closed interval \texttt{[i, j]} that is chosen
			to partition the array is either \texttt{i} or \texttt{j}. In particular,
			if $F$ does not occur, then the \texttt{k}th element chosen from the sorted
			subsequence \texttt{[i, j]} satisfies $i < k < j$, so \texttt{i} and \texttt{j}
			will not be compared. That is, $E$ will not occur. It follows that this conditional
			probability is $0$: $\prob(E|F^{\mathsf{c}})=0$. Then by the Law of Total Probability,
			we have
			\begin{align*}
				\prob(E) &= \prob(E|F)\prob(F)+\prob(E|F^{\mathsf{c}})P(F^{\mathsf{c}})\\
				&=\prob(E|F)\prob(F)
			\end{align*}
			Since every element in \texttt{[i, j]} is equally likely to be chosen, 
			we find that
			\begin{align*}
				\prob(F) &= \frac{|\{\texttt{i, j}\}|}{|\texttt{[i, j]}\cap \mathbb{N}|}=\frac{2}{j-i+1}
			\end{align*}
			If $F$ occurs, then $E$ will certainly occur, so $\prob(E|F)=1$, and hence,
			$\prob(E)=\prob(F)=2/(j-i+1)$.
			
			Now to prove Proposition K, let $\overline{x}$ be the average number of compares
			done by quicksort. Note that once the $i$th and $j$th element are compared,
			they will never be compared again. This is because they are compared only when
			either one is the partition element, and beyond that, that element will be
			fixed and no more comparisons are done against it. Hence, since the
			average is the sum of the frequency of each event multiplied by the probability
			of the event, we have
			\begin{align*}
				\overline{x} &= \sum_{j>i}\prob(\text{$i$th and $j$th smallest compared})\cdot
				(\text{\# times $i$th and $j$th smallest compared})\\
				&=\sum_{j=1}^{n}\sum_{i=1}^{j-1}\frac{2}{j-i+1}\\
				&\approx \sum_{j=1}^{n}\int_{1}^{j}\frac{2}{j-x+1}\ dx\\
				&=\sum_{j=1}^{n}[-2\ln|j-x+1|]_{x=1}^{x=j}\\
				&=2\cdot \sum_{j=1}^{n}(\ln|j|-\ln|1|)\\
				&=2\cdot (\ln 1 + \ln 2 + \cdots+\ln n)\\
				&=2\cdot \ln(n!)\\
				&\approx 2n\ln n
			\end{align*}
			The first approximation (with the integral) made use of the interpretation
			of the summation as a finite Riemann sum, which serves as an approximation
			for the integral. The last approximation is Stirling's approximation.
		\end{proof}
	\end{sol}
	\begin{ex}{15}
		\emph{Nuts and bolts}. (G. J. E. Rawlins). You have a mixed pile of $n$ nuts
		and $n$ bolts and need to quickly find the corresponding pair of nuts and bolts.
		Each nut matches exactly one bolt, and each bolt matches exactly one nut.
		By fitting a nut and bolt together, you can see which is bigger, but it is not
		possible to directly compare two nuts or two bolts. Give an efficient
		method for solving the problem.
	\end{ex}
	\begin{sol}
		Start with a nut. Take any bolt. If the bolt is too big for the nut, place
		it to the right. If the nut is too small for nut, place it to the left. Otherwise,
		place it in the center. Now all bolts will be on the table, and we've also
		found the bolt matching the starting nut. Before discarding the match,
		we pick up another nut, and compare it against the matched bolt. If the bolt
		is too big, than the bolt for the new nut is on the left bunch of bolts.
		Otherwise, the bolt for the new nut is on the right. We discard the matched
		nut and bolt and move on to the correct side. We continue this way until we've
		found all of the matches. Each time we do this, we pair a nut with a bolt, so
		the method solves the problem. The fact that it is efficient follows from
		noticing that this is just applying quicksort.
	\end{sol}
	\begin{ex}{16}
		\emph{Best case}. Write a program that produces a best-case array (with no duplicates)
		for \texttt{sort()} in Algorithm 2.5: an array of $n$ items with distinct keys
		having the property that every partition will produce sub-arrays that differ in
		size by at most 1 (the same sub-array lengths that would happen for $n$ equal
		keys). (For the purpose of this exercise, ignore the initial shuffle).
	\end{ex}
	\begin{sol}
		In order to obtain the best case, we need an array where the partition
		key is always the median (or close to it) of the current sub-array.
		
		To see this concretely, say $n=16$, and that we allow the possible keys
		to be $0$ through $n-1=15$, which are also (intentionally) the valid index
		values for an array of size $n=16$.
		
		The first median is $7$, and we can compute it as
		\begin{align*}
			\left\lfloor0 + \frac{15 - 0}{2}\right\rfloor = 7
		\end{align*}
		We want it to be the first partition key, so we set \texttt{a[0] = 7}.
		Say we call quicksort on \texttt{a} with 7 as \texttt{a[0]}. Since
		the \texttt{partition()} method always performs the final exchange
		with \texttt{j}, the last item in the left sub-array (with keys smaller
		than 7), that element becomes the next partition key for the left
		sub-array. By the same reasoning, we want it to be the median,
		and the median of \texttt{0..6} is 3, so it should be 3. Now
		we have something like this so far:
		\begin{lstlisting}[language={}]
7 < < < < < < 3 > > > > > > > >
		\end{lstlisting}
		The \texttt{<} and \texttt{>} are placeholders for items that are less
		than greater than the partition key (7), before the final exchange (recall
		the items are all distinct). Notice that we have here \texttt{a[7] = 3}.
		What about the partition key for the right sub-array for keys \texttt{8..15}?
		The partition key for the right sub-array will be 8, so it should be
		the median of the right sub-array, which is 11. So now we have:
		\begin{lstlisting}[language={}]
7 < < < < < < 3 11 > > > > > > >
		\end{lstlisting}
		In this case, by knowing the median 7, we computed median of the left sub-array
		and assigned it to \texttt{a[7]}, then the median of the right sub-array
		and assigned it to \texttt{a[7+1]}.
		
		To understand how we would proceed, let's exchange 7 and 3 as quicksort would do,
		and consider what remains:
		\begin{lstlisting}[language={}]
3 * * * * * * 7 11 * * * * * * * // after swap
3 * * * * * * // focus us now lower sub-array
		\end{lstlisting}
		As before, when \texttt{partition()} is called on this sub-array, the
		final exchange that occurs will involve the next partition item. After
		the exchange, all items on the left sub-array will be less than 3.
		The next partitioning item of the left sub-array should be the median
		of the next left sub-array. The left sub-array would have \texttt{0..2},
		and its median is 1, so the next partition index should be 1:
		\begin{lstlisting}[language={}]
3 < < 1 > > >
		\end{lstlisting}
		The right sub-array \texttt{4..6} has a median of \texttt{5}, which
		becomes its next partition key:
		\begin{lstlisting}[language={}]
3 < < 1 5 > >
		\end{lstlisting}
		Thus we see that we want \texttt{a[3] = 1} and \texttt{a[3 + 1] = 5}. If we exchange
		3 and 1 as would \texttt{partition()} in quicksort, so as to continue:
		\begin{lstlisting}[language={}]
1 * * 3 5 * * // after swap
1 * * // focus on lower sub-array
		\end{lstlisting}
		Now similar to before, we consider the median of the items left of the
		partition key (1). The left sub-array is just \texttt{0..0} and the
		right sub-array is \texttt{2..2}, so we conclude that we should have:
		\begin{lstlisting}[language={}]
1 0 2
		\end{lstlisting}
		So we set \texttt{a[1] = 0} and \texttt{a[1 + 1] = 2}. Similarly for the
		sub-array with \texttt{5} as its partition key we have:
		\begin{lstlisting}[language={}]
5 4 6
		\end{lstlisting}
		If we continued this process, we would get:
		\begin{lstlisting}[language={}]
7 0 2 1 5 4 6 3 11 8 10 9 13 12 14 15
		\end{lstlisting}
		At each step, the middle value served as the index for the median
		of the lower half, and the middle value plus 1 served as the index of
		the upper half. Then to proceed, the algorithm continued in the lower
		half and upper half going left first, in a recursive way.
		
		To formalize this algorithm, start with \texttt{lo = 0}, \texttt{hi = n - 1}:
		\begin{enumerate}[label=(\roman*)]
			\item \emph{Compute the median as}: \texttt{mid = lo + (hi - lo) / 2}.
			
			\item \emph{Set median of left sub-array}: \texttt{a[mid] = median(a, lo, mid-1)}.
				This is what \texttt{a[lo]} will first exchange with in quicksort,
				and it becomes the next partition key of the left sub-array.
				
			\item \emph{Set median of right sub-array}: \texttt{a[mid+1] = median(a, mid+1, hi)}.
			This is what becomes the partition key of the right sub-array.
		\end{enumerate}
		The algorithm eventually ends because the search space is cut roughly in half
		each time. The median of an empty range such as \texttt{lo..lo} is precisely
		\texttt{lo}, so this becomes the base case. We begin the chain of recursive
		calls with \texttt{a[0] = \texttt{median(a, 0, n - 1)}}, which would set
		\texttt{a[0] = 7} in our example above with $n=16$.
		
		Below is a trace of this for $n=16$. A green on the left columns under
		the headers \texttt{lo} and \texttt{mid} denotes a return value
		as we ascend back up in recursion depth. It is the next value that
		will be placed in the array. A red index under \texttt{mid} or
		\texttt{mid + 1} is the index where that return value will be placed.
		On the right side under the array indices, a star \texttt{*} denotes
		an entry that has yet to be placed, a gray entry is one that has already
		been assigned, and a green entry is one that is assigned in the current step.
		\begin{center}
			\begin{tabular}{cccc|cccccccccccccccc}
				\texttt{lo} & \texttt{mid} & \texttt{mid + 1} & \texttt{hi} &
				0 & 1 & 2 & 3 & 4 & 5 & 6 & 7 & 8 & 9 & 10 & 11 & 12 & 13 & 14 & 15\\
				\hline
				
				0 & {} & {} & 15 &
				* & * & * & * & * & * & * & * & * & * & * & * & * & * & * & * \\
				
				0 & 7 & 8 & 15 &
				* & * & * & * & * & * & * & * & * & * & * & * & * & * & * & * \\
				
				0 & 3 & 4 & 6 &
				* & * & * & * & * & * & * & * & * & * & * & * & * & * & * & * \\
				
				0 & 1 & 2 & 2 &
				* & * & * & * & * & * & * & * & * & * & * & * & * & * & * & * \\
				
				{\color{green}0} & {} & {} &0 &
				* & * & * & * & * & * & * & * & * & * & * & * & * & * & * & * \\
				
				0 & {\color{red}1} & 2 & 2 &
				* & {\color{green}0} & * & * & * & * & * & * & * & * & * & * & * & * & * & * \\
				
				{\color{green}2} & {} & {} & 2 &
				* & {\color{gray}0} & * & * & * & * & * & * & * & * & * & * & * & * & * & * \\
				
				0 & {\color{green}1} & {\color{red}2} & 2 &
				* & {\color{gray}0} & {\color{green}2} & * & * & * & * & * & * & * & * & * & * & * & * & * \\
				
				0 & {\color{red}3} & {\color{black}4} & 6 &
				* & {\color{gray}0} & {\color{gray}2} & {\color{green}1} & * & * & * & * & * & * & * & * & * & * & * & * \\
				
				4 & {\color{black}5} & {\color{black}6} & 6 &
				* & {\color{gray}0} & {\color{gray}2} & {\color{gray}1} & * & * & * & * & * & * & * & * & * & * & * & * \\
				
				{\color{green}4} & {} & {} & 4 &
				* & {\color{gray}0} & {\color{gray}2} & {\color{gray}1} & * & * & * & * & * & * & * & * & * & * & * & * \\
				
				{\color{green}4} & {\color{red}5} & {6} & 6 &
				* & {\color{gray}0} & {\color{gray}2} & {\color{gray}1} & * & {\color{green}4} & * & * & * & * & * & * & * & * & * & * \\
				
				{\color{green}6} & {} & {} & 6 &
				* & {\color{gray}0} & {\color{gray}2} & {\color{gray}1} & * & {\color{gray}4} & * & * & * & * & * & * & * & * & * & * \\
				
				{4} & {\color{green}5} & {\color{red}6} & 6 &
				* & {\color{gray}0} & {\color{gray}2} & {\color{gray}1} & * & {\color{gray}4} & {\color{green}6} & * & * & * & * & * & * & * & * & * \\
				
				{0} & {\color{green}3} & {\color{red}4} & 6 &
				* & {\color{gray}0} & {\color{gray}2} & {\color{gray}1} & {\color{green}5} & {\color{gray}4} & {\color{gray}6} & * & * & * & * & * & * & * & * & * \\
				
				{0} & {\color{red}7} & {8} & 15 &
				* & {\color{gray}0} & {\color{gray}2} & {\color{gray}1} & {\color{gray}5} & {\color{gray}4} & {\color{gray}6} & {\color{green}3} & * & * & * & * & * & * & * & * \\
				
				8 & 11 & 12 & 15 &
				* & {\color{gray}0} & {\color{gray}2} & {\color{gray}1} & {\color{gray}5} & {\color{gray}4} & {\color{gray}6} & {\color{gray}3} & * & * & * & * & * & * & * & * \\
				
				8 & 9 & 10 & 10 &
				* & {\color{gray}0} & {\color{gray}2} & {\color{gray}1} & {\color{gray}5} & {\color{gray}4} & {\color{gray}6} & {\color{gray}3} & * & * & * & * & * & * & * & * \\
				
				{\color{green}8} & {} & {} & 8 &
				* & {\color{gray}0} & {\color{gray}2} & {\color{gray}1} & {\color{gray}5} & {\color{gray}4} & {\color{gray}6} & {\color{gray}3} & * & * & * & * & * & * & * & * \\
				
				{8} & {\color{red}9} & {10} & 10 &
				* & {\color{gray}0} & {\color{gray}2} & {\color{gray}1} & {\color{gray}5} & {\color{gray}4} & {\color{gray}6} & {\color{gray}3} & * & {\color{green}8} & * & * & * & * & * & * \\
				
				{\color{green}10} & {} & {} & 10 &
				* & {\color{gray}0} & {\color{gray}2} & {\color{gray}1} & {\color{gray}5} & {\color{gray}4} & {\color{gray}6} & {\color{gray}3} & * & {\color{gray}8} & * & * & * & * & * & * \\
				
				{8} & {\color{green}9} & {\color{red}10} & 10 &
				* & {\color{gray}0} & {\color{gray}2} & {\color{gray}1} & {\color{gray}5} & {\color{gray}4} & {\color{gray}6} & {\color{gray}3} & * & {\color{gray}8} & {\color{green}10} & * & * & * & * & * \\
				
				{8} & {\color{red}11} & {\color{black}12} & 15 &
				* & {\color{gray}0} & {\color{gray}2} & {\color{gray}1} & {\color{gray}5} & {\color{gray}4} & {\color{gray}6} & {\color{gray}3} & * & {\color{gray}8} & {\color{gray}10} & {\color{green}9} & * & * & * & * \\
				
				{12} & {13} & {14} & 15 &
				* & {\color{gray}0} & {\color{gray}2} & {\color{gray}1} & {\color{gray}5} & {\color{gray}4} & {\color{gray}6} & {\color{gray}3} & * & {\color{gray}8} & {\color{gray}10} & {\color{gray}9} & * & * & * & * \\
				
				{\color{green}12} & {} & {} & 12 &
				* & {\color{gray}0} & {\color{gray}2} & {\color{gray}1} & {\color{gray}5} & {\color{gray}4} & {\color{gray}6} & {\color{gray}3} & * & {\color{gray}8} & {\color{gray}10} & {\color{gray}9} & * & * & * & * \\
				
				{\color{black}12} & {\color{red}13} & {14} & 15 &
				* & {\color{gray}0} & {\color{gray}2} & {\color{gray}1} & {\color{gray}5} & {\color{gray}4} & {\color{gray}6} & {\color{gray}3} & * & {\color{gray}8} & {\color{gray}10} & {\color{gray}9} & * & {\color{green}12} & * & * \\
				
				{\color{black}14} & {14} & {15} & 15 &
				* & {\color{gray}0} & {\color{gray}2} & {\color{gray}1} & {\color{gray}5} & {\color{gray}4} & {\color{gray}6} & {\color{gray}3} & * & {\color{gray}8} & {\color{gray}10} & {\color{gray}9} & * & {\color{gray}12} & * & * \\
				
				{\color{green}14} & {} & {} & 13 &
				* & {\color{gray}0} & {\color{gray}2} & {\color{gray}1} & {\color{gray}5} & {\color{gray}4} & {\color{gray}6} & {\color{gray}3} & * & {\color{gray}8} & {\color{gray}10} & {\color{gray}9} & * & {\color{gray}12} & * & * \\
				
				{\color{black}14} & {\color{red}14} & {15} & 15 &
				* & {\color{gray}0} & {\color{gray}2} & {\color{gray}1} & {\color{gray}5} & {\color{gray}4} & {\color{gray}6} & {\color{gray}3} & * & {\color{gray}8} & {\color{gray}10} & {\color{gray}9} & * & {\color{gray}12} & {\color{green}14} & * \\
				
				{\color{green}15} & {} & {} & 15 &
				* & {\color{gray}0} & {\color{gray}2} & {\color{gray}1} & {\color{gray}5} & {\color{gray}4} & {\color{gray}6} & {\color{gray}3} & * & {\color{gray}8} & {\color{gray}10} & {\color{gray}9} & * & {\color{gray}12} & {\color{gray}14} & * \\
				
				{\color{black}14} & {\color{green}14} & {\color{red}15} & 15 &
				* & {\color{gray}0} & {\color{gray}2} & {\color{gray}1} & {\color{gray}5} & {\color{gray}4} & {\color{gray}6} & {\color{gray}3} & * & {\color{gray}8} & {\color{gray}10} & {\color{gray}9} & * & {\color{gray}12} & {\color{gray}14} & {\color{green}15} \\
				
				{12} & {\color{green}13} & {\color{red}14} & 15 &
				* & {\color{gray}0} & {\color{gray}2} & {\color{gray}1} & {\color{gray}5} & {\color{gray}4} & {\color{gray}6} & {\color{gray}3} & * & {\color{gray}8} & {\color{gray}10} & {\color{gray}9} & * & {\color{gray}12} & {\color{green}14} & {\color{gray}15} \\
				
				{8} & {\color{green}11} & {\color{red}12} & 15 &
				* & {\color{gray}0} & {\color{gray}2} & {\color{gray}1} & {\color{gray}5} & {\color{gray}4} & {\color{gray}6} & {\color{gray}3} & * & {\color{gray}8} & {\color{gray}10} & {\color{gray}9} & {\color{green}13} & {\color{gray}12} & {\color{gray}14} & {\color{gray}15} \\
				
				{0} & {\color{green}7} & {\color{red}8} & 15 &
				* & {\color{gray}0} & {\color{gray}2} & {\color{gray}1} & {\color{gray}5} & {\color{gray}4} & {\color{gray}6} & {\color{gray}3} & {\color{green}11} & {\color{gray}8} & {\color{gray}10} & {\color{gray}9} & {\color{gray}13} & {\color{gray}12} & {\color{gray}14} & {\color{gray}15} \\
				
				{\color{red}0} & {} & {} & 15 &
				{\color{green}7} & {\color{gray}0} & {\color{gray}2} & {\color{gray}1} & {\color{gray}5} & {\color{gray}4} & {\color{gray}6} & {\color{gray}3} & {\color{green}11} & {\color{gray}8} & {\color{gray}10} & {\color{gray}9} & {\color{gray}13} & {\color{gray}12} & {\color{gray}14} & {\color{gray}15} \\
				
				{} & {} & {} & {} &
				{\color{black}7} & {\color{black}0} & {\color{black}2} & {\color{black}1} & {\color{black}5} & {\color{black}4} & {\color{black}6} & {\color{black}3} & {\color{black}11} & {\color{black}8} & {\color{black}10} & {\color{black}9} & {\color{black}13} & {\color{black}12} & {\color{black}14} & {\color{black}15} \\
			\end{tabular}
		\end{center}
		
		See \texttt{com.segarciat.algs4.ch2.sec3.ex16.QuicksortBestCase} for my implementation
		of this algorithm. To verify the effectiveness of this algorithm, I used
		my program from Exercise 2.3.6 to compare the actual compares to the average
		number of compares, and computed ratios of these quantities. I modified the
		quicksort algorithm to not perform the random shuffle to ensure the ``best-case"
		array produced by my algorithm was used as-is. Below are the results of a sample
		run for different array sizes:
		\begin{lstlisting}[language={}]
n=2, actual=3, average=2, actualToAverageRatio=1.50
n=4, actual=8, average=8, actualToAverageRatio=1.02
n=8, actual=21, average=24, actualToAverageRatio=0.88
n=16, actual=54, average=66, actualToAverageRatio=0.82
n=32, actual=135, average=171, actualToAverageRatio=0.79
n=64, actual=328, average=424, actualToAverageRatio=0.77
n=128, actual=777, average=1017, actualToAverageRatio=0.76
n=256, actual=1802, average=2379, actualToAverageRatio=0.76
n=512, actual=4107, average=5457, actualToAverageRatio=0.75
n=1024, actual=9228, average=12321, actualToAverageRatio=0.75
n=2048, actual=20493, average=27467, actualToAverageRatio=0.75
n=4096, actual=45070, average=60597, actualToAverageRatio=0.74
n=8192, actual=98319, average=132535, actualToAverageRatio=0.74
n=16384, actual=213008, average=287765, actualToAverageRatio=0.74
n=32768, actual=458769, average=620938, actualToAverageRatio=0.74
n=65536, actual=983058, average=1332707, actualToAverageRatio=0.74
n=131072, actual=2097171, average=2847097, actualToAverageRatio=0.74
n=262144, actual=4456468, average=6057579, actualToAverageRatio=0.74
n=524288, actual=9437205, average=12841950, actualToAverageRatio=0.73
n=1048576, actual=19922966, average=27137510, actualToAverageRatio=0.73
n=2097152, actual=41943063, average=57182262, actualToAverageRatio=0.73
n=4194304, actual=88080408, average=120179035, actualToAverageRatio=0.73
n=8388608, actual=184549401, average=251987120, actualToAverageRatio=0.73
n=16777216, actual=385875994, average=527232369, actualToAverageRatio=0.73
n=33554432, actual=805306395, average=1100981024, actualToAverageRatio=0.73
		\end{lstlisting}
		The data suggests that in order for quicksort to sort the array produced
		by my algorithm, it requires about $75\%$ of the compares that it would
		normally need on the average to sort a randomly ordered array. This
		supports the effectiveness of the algorithm in producing a ``best-case" array.
	\end{sol}
	\begin{ex}{17}
		\emph{Sentinels}. Modify the code in Algorithm 2.5 to remove both bounds checks in the
		inner \texttt{while} loops. The test against the left end of the subarray is redundant
		since the partitioning item acts as a sentinel (\texttt{v} is never less than
		\texttt{a[lo]}). To enable removal of the other test, put an item whose key is the
		largest in the whole array into \texttt{a[a.length-1]} just after the shuffle. This
		item will  never move (except possibly to be swapped with an item having the same key)
		and will serve as a sentinel in all subarrays involving the end of the array.
		\emph{Note}: For a subarray that does not involve the end of the array, the leftmost
		entry to its right serves as a sentinel for the right end of the subarray.
	\end{ex}
	\begin{sol}
		See \texttt{com.segarciat.algs4.ch2.sec3.ex17.QuickSentinel.sort()}.
	\end{sol}
	\begin{ex}{18}
		\emph{Median-of-3-partitioning}. Add median-of-3 partitioning to quicksort, as described
		in the text (see page 296). Run doubling tests to determine the effectiveness of
		the change.
	\end{ex}
	\begin{sol}
		See \texttt{com.segarciat.algs4.ch2.sec3.ex18.QuickMedianOf3.sort()}.
		The improvement was in the order of about $10\%$ or so.
	\end{sol}
	\begin{ex}{2.3.19}
		\emph{Median-of-5 partitioning}. Implement a quicksort based on partitioning on
		the median of a random sample of five items from the sub-array. Put the items
		of the sample at the appropriate ends so that only the median participates in
		partitioning. Run doubling tests to determine the effectiveness of the change,
		in comparison both to the standard algorithm and to the median-of-3 partitioning
		(see the previous exercise). \emph{Extra credit}: Devise a median-of-5 algorithm
		that uses fewer than seven compares on any input.
	\end{ex}
	\begin{sol}
		My first mistake when attempting this problem was trying to sort the items.
		However, after some thought, I recalled that Proposition I in \cite{sedgewick_wayne}
		says that no compare-based sorting algorithm can guarantee to sort $n$ items
		with fewer than $\lg(n!)$ compares. With a sample of $5$ keys, we have
		$\lg(5!)\approx 6.9$, so it's not possible to sort the items with fewer
		than $7$ compares! Therefore I focused on finding the median of the 5, which should
		be the partition key, and a key no less than the median, which will be a sentinel
		at the right side. Nevertheless, I was unable to do it in less than 7 compares,
		and my approach required at most 7 compares.
		
		See \texttt{com.segarciat.algs4.ch2.sec3.ex19.QuickMedianOf5}.	
	\end{sol}
	\begin{ex}{20}
		\emph{Non-recursive quicksort}. Implement a non-recursive version of quicksort based on
		a main loop where a sub-array is popped from a stack to be partitioned, and the
		resulting sub-arrays are pushed onto the stack. *Note*: Push the larger of the
		sub-arrays onto the stack first, which guarantees that the stack will have at most
		$\lg n$ entries.
	\end{ex}
	\begin{sol}
		See \texttt{com.segarciat.algs4.ch2.sec3.ex20.NonRecQuick}.
	\end{sol}
	\pagebreak
	\printbibliography
\end{document}