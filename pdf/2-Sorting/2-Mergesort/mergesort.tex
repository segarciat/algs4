\documentclass[12pt, a4paper]{article}

\usepackage[utf8]{inputenc}
% Limit the page margin to only 1 inch.
\usepackage[margin=1in]{geometry}

%Imports biblatex package
\usepackage[
backend=biber,
style=alphabetic
]{biblatex}
\addbibresource{../../algs4e.bib}

% Enables the `align' environment.
\usepackage{amsmath}
% Provides useful environments, such as:
% - \begin{proof} ...\end{proof}
\usepackage{amsthm}
% Enables using \mathbb{}, for example \mathbb{N} for the set of natural numbers.
\usepackage{amssymb}

% Allows using letters in enumerate list environment. Use, for example:
%\begin{enumerate}[label=(\alph*)]
% ...
%\end{enumerate}
\usepackage[inline]{enumitem}

% Enable importing external graphic files and provides useful commannds, like \graphicspath{}
\usepackage{graphicx}
% Images are located in a directory called images in the current directory.
\graphicspath{{./images/}}

% Make links look better by default.
% See: https://tex.stackexchange.com/questions/823/remove-ugly-borders-around-clickable-cross-references-and-hyperlinks
\usepackage[hidelinks]{hyperref}
\usepackage{xcolor}
\hypersetup{
	colorlinks,
	linkcolor={red!50!black},
	citecolor={blue!50!black},
	urlcolor={blue!80!black}
}


% Code Listings. Source:
% https://stackoverflow.com/questions/3175105/inserting-code-in-this-latex-document-with-indentation
\usepackage{listings}
\usepackage{color}

\definecolor{dkgreen}{rgb}{0,0.6,0}
\definecolor{gray}{rgb}{0.5,0.5,0.5}
\definecolor{mauve}{rgb}{0.58,0,0.82}

\lstset{frame=tb,
	language=Java,
	aboveskip=3mm,
	belowskip=3mm,
	showstringspaces=false,
	columns=flexible,
	basicstyle={\small\ttfamily},
	numbers=none,
	numberstyle=\tiny\color{gray},
	keywordstyle=\color{blue},
	commentstyle=\color{dkgreen},
	stringstyle=\color{mauve},
	breaklines=true,
	breakatwhitespace=true,
	tabsize=3
}

\newcommand{\prob}{\text{P}}
%\newcommand{\complement}{\mathsf{c}}

% Define an environment called "ex" (for Exercise) so that I can do: \begin{ex}{1.5}...\end{ex}
\newenvironment{ex}[2][Exercise]
{\par\medskip\noindent \textbf{#1 #2.}}
{\medskip}

% Define a solution environment, similar to ex (exercise) environment.
\newenvironment{sol}[1][Solution]
{\par\medskip\noindent \textbf{#1.} }
{\medskip}

\begin{document}
	\noindent Sergio E. Garcia Tapia \hfill
	
	\noindent \emph{Algorithms} by Sedgewick and Wayne (4th edition) \cite{sedgewick_wayne}\hfill
	
	\noindent October 13th, 2024\hfill 
	\section*{2.2: Mergesort}
	\begin{ex}{1}
		Give a trace, in the style of the trace given at the beginning of this section,
		showing how the keys \texttt{A E Q S U Y E I N O S T} are merged with the abstract
		in-place \texttt{merge()} method.
	\end{ex}
	\begin{sol}
		\begin{center}
			{\scriptsize
			\begin{tabular}{c|cccccccccccc|cc|cccccc|cccccc}
				\multicolumn{13}{c}{\texttt{a[]}} & {} & {} & \multicolumn{12}{c}{\texttt{aux[]}}\\
				\texttt{k} & 0 & 1 & 2 & 3 & 4 & 5 & 6 & 7 & 8 & 9 & 10 & 11
				& i & j
				& 0 & 1 & 2 & 3 & 4 & 5 & 6 & 7 & 8 & 9 & 10 & 11\\
				
				\hline
				
				{} & A & E & Q & S & U & Y & E & I & N & O & S & T &
				{} & {}
				& {} & {} &{} &{} & {} & {} & {} & {} & {} & {} & {} & {}\\
				
				{} & A & E & Q & S & U & Y & E & I & N & O & S & T
				& {} & {}
				& A & E & Q & S & U & Y & E & I & N & O & S & T \\
				
				{} & {} & {} &{} &{} & {} & {} & {} & {} & {} & {} & {} & {}
				& 0 & 6
				& {} & {} &{} &{} & {} & {} & {} & {} & {} & {} & {} & {} \\
				
				0 & {\color{red} A} & {} &{} &{} & {} & {} & {} & {} & {} & {} & {} & {}
				& 1 & 6
				& {\color{red}A} & E & Q & S & U & Y & E & I & N & O & S & T\\
				
				1 & {\color{gray} A} & {\color{red}E} &{} &{} & {} & {} & {} & {} & {} & {} & {} & {}
				& 2 & 6
				& {} & {\color{red}E} & Q & S & U & Y & E & I & N & O & S & T \\
				
				2 & {\color{gray} A} & {\color{gray}E} & {\color{red}E} &{} & {} & {} & {} & {} & {} & {} & {} & {}
				& 2 & 7
				& {} & {} & Q & S & U & Y
				& {\color{red}E} & I & N & O & S & T \\
				
				3 & {\color{gray} A} & {\color{gray}E} & {\color{gray}E} &{\color{red}I} & {} & {} & {} & {} & {} & {} & {} & {}
				& 2 & 8
				& {} & {} & Q & S & U & Y
				& {} & {\color{red}I} & N & O & S & T \\
				
				4 & {\color{gray} A} & {\color{gray}E} & {\color{gray}E} &{\color{gray}I} & {\color{red}N} & {} & {} & {} & {} & {} & {} & {}
				& 2 & 9
				& {} & {} & Q & S & U & Y
				& {} & {} & {\color{red}N} & O & S & T \\
				
				5 & {\color{gray} A} & {\color{gray}E} & {\color{gray}E} &{\color{gray}I} & {\color{gray}N} & {\color{red}O} & {} & {} & {} & {} & {} & {}
				& 2 & 10
				& {} & {} & Q & S & U & Y
				& {} & {} & {} & {\color{red}O} & S & T \\
				
				6 & {\color{gray} A} & {\color{gray}E} & {\color{gray}E} &{\color{gray}I} & {\color{gray}N} & {\color{gray}O} & {\color{red}Q} & {} & {} & {} & {} & {}
				& 3 & 10
				& {} & {} & {\color{red}Q} & S & U & Y
				& {} & {} & {} & {} & S & T \\
				
				7 & {\color{gray} A} & {\color{gray}E} & {\color{gray}E} &{\color{gray}I} & {\color{gray}N} & {\color{gray}O} & {\color{gray}Q} & {\color{red}S} & {} & {} & {} & {}
				& 4 & 10
				& {} & {} & {} & {\color{red}S} & U & Y
				& {} & {} & {} & {} & S & T \\
				
				8 & {\color{gray} A} & {\color{gray}E} & {\color{gray}E} &{\color{gray}I} & {\color{gray}N} & {\color{gray}O} & {\color{gray}Q} & {\color{gray}S} & {\color{red}S} & {} & {} & {}
				& 4 & 11
				& {} & {} & {} & {} & U & Y
				& {} & {} & {} & {} & {\color{red}S} & T \\
				
				9 & {\color{gray} A} & {\color{gray}E} & {\color{gray}E} &{\color{gray}I} & {\color{gray}N} & {\color{gray}O} & {\color{gray}Q} & {\color{gray}S} & {\color{gray}S} & {\color{red}T} & {} & {}
				& 4 & 12
				& {} & {} & {} & {} & U & Y
				& {} & {} & {} & {} & {} & {\color{red}T} \\
				
				10 & {\color{gray} A} & {\color{gray}E} & {\color{gray}E} &{\color{gray}I} & {\color{gray}N} & {\color{gray}O} & {\color{gray}Q} & {\color{gray}S} & {\color{gray}S} & {\color{gray}T} & {\color{red}U} & {}
				& 5 & 12
				& {} & {} & {} & {} & {\color{red}U} & Y
				& {} & {} & {} & {} & {} & {} \\
				
				11 & {\color{gray} A} & {\color{gray}E} & {\color{gray}E} &{\color{gray}I} & {\color{gray}N} & {\color{gray}O} & {\color{gray}Q} & {\color{gray}S} & {\color{gray}S} & {\color{gray}T} & {\color{gray}U} & {\color{red}Y}
				& 6 & 12
				& {} & {} & {} & {} & {} & {\color{red}Y}
				& {} & {} & {} & {} & {} & {} \\
				
				{} & A & E & E & I & N & O & Q & S & S & T & U & Y
				& {} & {}
				& {} & {} & {} & {} & {} & {}
				& {} & {} & {} & {} & {} & {} \\
			\end{tabular}
			}
		\end{center}
	\end{sol}
	\begin{ex}{2}
		Give traces, in the style of the trace given with Algorithm 2.4, showing how the keys
		\texttt{E A S Y Q U E S T I O N} are sorted with top-down mergesort.
	\end{ex}
	\begin{sol}
		The following shows the sequence of calls:
		\begin{lstlisting}[language={}]
sort(a, 0, 11)
	sort(a, 0, 5) // left half
		sort(a, 0, 2)
			sort(a, 0, 1)
				merge(a, 0, 0, 1)
			sort(a, 2, 2)
				// no merge
			merge(a, 0, 1, 2)
		sort(a, 3, 5)
			sort(a, 3, 4)
				merge(a, 3, 3, 4)
			sort(a, 5, 5)
				// no merge
			merge(a, 3, 4, 5)
		merge(0, 2, 5) // done sorting left half
	sort(a, 6, 11)
		sort(a, 6, 8)
			sort(a, 6, 7)
				merge(a, 6, 6, 7)
			sort(a, 8, 8)
				// no merge
			merge(a, 6, 7, 8)
		sort(a, 9, 11)
			sort(a, 9, 10)
				merge(a, 9, 9, 10)
			sort(a, 11, 11)
				// no merge
			merge(a, 9, 10, 11)
		merge(a, 6, 8, 11) // done sorting right black
	merge(a, 0, 5, 11)
		\end{lstlisting}
		\begin{center}
			\begin{tabular}{c|cccccccccccc}
				{} & \multicolumn{12}{c}{\texttt{a[]}}\\
				{} & 0 & 1 & 2 & 3 & 4 & 5 & 6 & 7 & 8 & 9 & 10 & 11 \\
				\hline
				{} & E & A & S & Y & Q & U & E & S & T & I & O & N \\
				
				\texttt{merge(a, {\color{red}0}, 0, {\color{red}1})}
				& A & E & {\color{gray}S} & {\color{gray}Y} & {\color{gray}Q} & {\color{gray}U} & {\color{gray}E} & {\color{gray}S} & {\color{gray}T} & {\color{gray}I} & {\color{gray}O} & {\color{gray}N}\\
				
				\texttt{merge(a, {\color{red}0}, 1, {\color{red}2})}
				& {\color{black}A} & {\color{black}E} & {\color{black}S} & {\color{gray}Y} & {\color{gray}Q} & {\color{gray}U} & {\color{gray}E} & {\color{gray}S} & {\color{gray}T} & {\color{gray}I} & {\color{gray}O} & {\color{gray}N}\\
				
				\texttt{merge(a, {\color{red}3}, 3, {\color{red}4})}
				& {\color{gray}A} & {\color{gray}E} & {\color{gray}S} & {\color{black}Q} & {\color{black}Y} & {\color{gray}U} & {\color{gray}E} & {\color{gray}S} & {\color{gray}T} & {\color{gray}I} & {\color{gray}O} & {\color{gray}N}\\
				
				\texttt{merge(a, {\color{red}3}, 4, {\color{red}5})}
				& {\color{gray}A} & {\color{gray}E} & {\color{gray}S} & {\color{black}Q} & {\color{black}U} & {\color{black}Y} & {\color{gray}E} & {\color{gray}S} & {\color{gray}T} & {\color{gray}I} & {\color{gray}O} & {\color{gray}N}\\
				
				\texttt{merge(a, {\color{red}0}, 2, {\color{red}5})}
				& {\color{black}A} & {\color{black}E} & {\color{black}Q} & {\color{black}S} & {\color{black}U} & {\color{black}Y} & {\color{gray}E} & {\color{gray}S} & {\color{gray}T} & {\color{gray}I} & {\color{gray}O} & {\color{gray}N}\\
				
				\texttt{merge(a, {\color{red}6}, 6, {\color{red}7})}
				& {\color{gray}A} & {\color{gray}E} & {\color{gray}Q} & {\color{gray}S} & {\color{gray}U} & {\color{gray}Y} & {\color{black}E} & {\color{black}S} & {\color{gray}T} & {\color{gray}I} & {\color{gray}O} & {\color{gray}N}\\
				
				\texttt{merge(a, {\color{red}6}, 7, {\color{red}8})}
				& {\color{gray}A} & {\color{gray}E} & {\color{gray}Q} & {\color{gray}S} & {\color{gray}U} & {\color{gray}Y} & {\color{black}E} & {\color{black}S} & {\color{black}T} & {\color{gray}I} & {\color{gray}O} & {\color{gray}N}\\
				
				\texttt{merge(a, {\color{red}9}, 9, {\color{red}10})}
				& {\color{gray}A} & {\color{gray}E} & {\color{gray}Q} & {\color{gray}S} & {\color{gray}U} & {\color{gray}Y} & {\color{gray}E} & {\color{gray}S} & {\color{gray}T} & {\color{black}I} & {\color{black}O} & {\color{gray}N}\\
				
				\texttt{merge(a, {\color{red}9}, 10, {\color{red}11})}
				& {\color{gray}A} & {\color{gray}E} & {\color{gray}Q} & {\color{gray}S} & {\color{gray}U} & {\color{gray}Y} & {\color{gray}E} & {\color{gray}S} & {\color{gray}T} & {\color{black}I} & {\color{black}N} & {\color{black}O}\\
				
				\texttt{merge(a, {\color{red}6}, 8, {\color{red}11})}
				& {\color{gray}A} & {\color{gray}E} & {\color{gray}Q} & {\color{gray}S} & {\color{gray}U} & {\color{gray}Y} & {\color{black}E} & {\color{black}I} & {\color{black}N} & {\color{black}O} & {\color{black}S} & {\color{black}T}\\
				
				\texttt{merge(a, {\color{red}0}, 5, {\color{red}11})}
				& A & E & E & I & N & O & Q & S & S & T & U & Y\\
			\end{tabular}
		\end{center}
	\end{sol}
	\begin{ex}{3}
		Answer Exercise 2.2.2 for bottom-up mergesort.
	\end{ex}
	\begin{sol}
		\begin{center}
			\begin{tabular}{c|cccccccccccc}
				{} & \multicolumn{12}{c}{\texttt{a[i]}}\\
				{} & 0 & 1 & 2 & 3 & 4 & 5 & 6 & 7 & 8 & 9 & 10  & 11 \\
				\hline
				\texttt{len = 1} & E & A & S & Y & Q & U & E & S & T & I & O & N \\
				
	\texttt{merge(a, {\color{red}0}, 0, {\color{red}1})}
	& {\color{black}A} & {\color{black}E} & {\color{gray}S} & {\color{gray}Y} & {\color{gray}Q} & {\color{gray}U}
	& {\color{gray}E} & {\color{gray}S} & {\color{gray}T} & {\color{gray}I} & {\color{gray}O} & {\color{gray}N}\\
	
	\texttt{merge(a, {\color{red}2}, 2, {\color{red}3})}
	& {\color{gray}A} & {\color{gray}E} & {\color{black}S} & {\color{black}Y} & {\color{gray}Q} & {\color{gray}U}
	& {\color{gray}E} & {\color{gray}S} & {\color{gray}T} & {\color{gray}I} & {\color{gray}O} & {\color{gray}N}\\
	
	\texttt{merge(a, {\color{red}4}, 4, {\color{red}5})}
	& {\color{gray}A} & {\color{gray}E} & {\color{gray}S} & {\color{gray}Y} & {\color{black}Q} & {\color{black}U}
	& {\color{gray}E} & {\color{gray}S} & {\color{gray}T} & {\color{gray}I} & {\color{gray}O} & {\color{gray}N}\\
	
	\texttt{merge(a, {\color{red}6}, 6, {\color{red}7})}
	& {\color{gray}A} & {\color{gray}E} & {\color{gray}S} & {\color{gray}Y} & {\color{gray}Q} & {\color{gray}U}
	& {\color{black}E} & {\color{black}S} & {\color{gray}T} & {\color{gray}I} & {\color{gray}O} & {\color{gray}N}\\
	
	\texttt{merge(a, {\color{red}8}, 8, {\color{red}9})}
	& {\color{gray}A} & {\color{gray}E} & {\color{gray}S} & {\color{gray}Y} & {\color{gray}Q} & {\color{gray}U}
	& {\color{gray}E} & {\color{gray}S} & {\color{black}I} & {\color{black}T} & {\color{gray}O} & {\color{gray}N}\\
	
	\texttt{merge(a, {\color{red}10}, 10, {\color{red}11})}
	& {\color{gray}A} & {\color{gray}E} & {\color{gray}S} & {\color{gray}Y} & {\color{gray}Q} & {\color{gray}U}
	& {\color{gray}E} & {\color{gray}S} & {\color{gray}I} & {\color{gray}T} & {\color{black}N} & {\color{black}O}\\
	
	\texttt{len = 2}\\
	
	\texttt{merge(a, {\color{red}0}, 1, {\color{red}3})}
	& {\color{black}A} & {\color{black}E} & {\color{black}S} & {\color{black}Y} & {\color{gray}Q} & {\color{gray}U}
	& {\color{gray}E} & {\color{gray}S} & {\color{gray}I} & {\color{gray}T} & {\color{gray}N} & {\color{gray}O}\\
	
	\texttt{merge(a, {\color{red}4}, 5, {\color{red}7})}
	& {\color{gray}A} & {\color{gray}E} & {\color{gray}S} & {\color{gray}Y} & {\color{black}E} & {\color{black}Q}
	& {\color{black}S} & {\color{black}U} & {\color{gray}I} & {\color{gray}T} & {\color{gray}N} & {\color{gray}O}\\
	
	\texttt{merge(a, {\color{red}8}, 9, {\color{red}11})}
	& {\color{gray}A} & {\color{gray}E} & {\color{gray}S} & {\color{gray}Y} & {\color{gray}E} & {\color{gray}Q}
	& {\color{gray}S} & {\color{gray}U} & {\color{black}I} & {\color{black}N} & {\color{black}O} & {\color{black}T}\\
	
	\texttt{len = 4}\\
	
	\texttt{merge(a, {\color{red}0}, 3, {\color{red}7})}
	& {\color{black}A} & {\color{black}E} & {\color{black}E} & {\color{black}Q} & {\color{black}S} & {\color{black}S}
	& {\color{black}U} & {\color{black}Y} & {\color{gray}I} & {\color{gray}N} & {\color{gray}O} & {\color{gray}T}\\
	
	\texttt{len = 8}\\
	
	\texttt{merge(a, {\color{red}0}, 8, {\color{red}11})}
	& A & E & E & I & N & O & Q & S & S & T & U & Y\\
			\end{tabular}
		\end{center}
	\end{sol}
	\begin{ex}{4}
		Does the abstract in-place merge produce proper output if and only if the
		two input subarrays are in sorted order? Prove your answer, or provide a counterexample.
	\end{ex}
	\begin{sol}
		\begin{proof}
			If the arrays are sorted, the algorithm certainly places the result in proper order,
			as we have seen throughout this chapter.
			
			Suppose that one of the input arrays \texttt{a[]} is not in sorted order. Then
			there is an index \texttt{i} such that \texttt{a[i] > a[i + 1]}. The algorithm
			will not increase \texttt{i} and add \texttt{a[i + 1]} to the result array
			until \texttt{a[i]} is in the result array. That is, \texttt{a[i]} will still
			appear before \texttt{a[i + 1]} in the result array, and the result array will
			still not be properly sorted.
		\end{proof}
	\end{sol}
	\begin{ex}{5}
		Give the sequence of subarray lengths in the merges performed by both the top-down
		and bottom-up mergesort, for $n=39$.
	\end{ex}
	\begin{sol}
		For top-down mergesort, we can build the sequence top-down and then reverse it:
		\begin{lstlisting}[language={}]
a[0..38] // 39
	a[20..38] // 19
		a[30..38] // 9
			a[35..38] // 4
				a[37..38] // 2
				a[35..36] // 2
			a[30..34] // 5
				a[33..34] // 2
				a[30..32] // 3
					a[32..32] // no merge
					a[30..31] // 2
		a[20..29] // 10
			a[25..29] // 5
				a[28..29] // 2
				a[25..27] // 3
					a[27..27] // no merge
					a[25..26] // 2
			a[20..24] // 5
				a[23..24] // 2
				a[20..22] // 3
					a[22..22] // no merge
					a[20..21] // 2
	a[0..19] // 20
		a[10..19] // 10
			a[15..19] // 5
				a[18..19] // 2
				a[15..17] // 3
					a[17..17] // no merge
					a[15..16] // 2
			a[10..14] // 5
				a[13..14] // 2
				a[10..12] // 3
					a[12..12] // no merge
					a[10..11] // 2
		a[0..9] // 10
			a[5..9] // 5
				a[8..9] // 2
				a[5..7] // 3
					a[7..7] // no merge
					a[5..6] // 2
			a[0..4] // 5
				a[3..4] // 2
				a[0..2] // 3
					a[2..2] // no merge
					a[0..1] // 2
		\end{lstlisting}
		Therefore, we read the sequence from the bottom to get:
		2, 3, 2, 5, 2, 3, 2, 5, 10,
		2, 3, 2, 5, 2, 3, 2, 5, 10,
		20,
		2, 3, 2, 5, 2, 3, 2, 5, 10,
		2, 3, 2, 5, 2, 2, 4, 9, 19,
		39.
		
		For the bottom-up mergesort, it is much simpler because most sizes are powers of 2 except
		possibly the last one for a given \texttt{len} value:
		\begin{lstlisting}
2,2,2,2,2,2,2,2,2,2,2,2,2,2,2,2,2,2,2,
4, 4, 4, 4, 4, 4, 4, 4, 4, 3,
8, 8, 8, 8, 7,
16, 16,
32,
39
		\end{lstlisting}
	\end{sol}
	\begin{ex}{6}
		Write a program to compute the exact value of the number of array accesses used
		by top-down mergesort and by bottom-up mergesort. Use your program to plot the
		values of $n$ from $1$ to $512$, and to compare the exact values with the upper bound
		$6n\lg n$.
	\end{ex}
	\begin{sol}
		See the class \texttt{com.segarciat.algs4.ch2.sec2.ex06.MergesortPlot}.
		\begin{figure}
			\centering
			\includegraphics[width=0.6\textwidth]{mergesort-array-access-cost-plot}
			\caption{Plot for Exercise 2.2.6; top-down mergesort in red, bottom-up mergesort in green,
			and the $6n\lg n$ bound in blue}
			\label{ex:6}
		\end{figure}
		See Figure~\ref{ex:6}.
	\end{sol}
	\begin{ex}{7}
		Show that the number of compares used by mergesort is monotonically increasing,
		meaning $C(n+1)>C(n)$ for all $n>0$.
	\end{ex}
	\begin{sol}
		TODO.
	\end{sol}
	\begin{ex}{8}
		Suppose that Algorithm 2.4 is modified to skip the call on \texttt{merge()} whenever
		\texttt{a[mid] <= a[mid+1]}. Prove that the number of compares used to mergesort
		a sorted array is linear.
	\end{ex}
	\begin{sol}
		\begin{proof}
			With this modification, the algorithm does one compare for each recursive call.
			Let $k=\lfloor \lg(n) \rfloor$. If $i$ is an integer between $0$ and $k$ (inclusive),
			then $i$, then $i$ represents the recursion depth of merge sort. In particular,
			$i=0$ is the initial call, and the $i$th level has $2^i$ recursive calls.
			The total number of recursive calls, and hence the total number of compares, is
			bounded by
			\begin{align*}
				\sum_{i=0}^{k}2^i&=2^{k+1}-1\\
				&=2\cdot 2^k-1\\
				&=2\cdot 2^{\lfloor \lg n\rfloor } -1\\
				&\leq 2\cdot 2^{\lg n +1}- 1\\
				&=4n-1
			\end{align*}
			Similarly it is bounded below by $\sum_{i=0}^{k-1}2^i$. We conclude that it is linear.
		\end{proof}
	\end{sol}
	\begin{ex}{9}
		Use of a static array like \texttt{aux[]} is inadvisable in library software because
		multiple clients might use the class concurrently. Give an implementation of \texttt{Merge}
		that does not use a static array. Do \emph{not} make \texttt{a[]} local to \texttt{merge()}
		(see the Q \& A for this section).
		\emph{Hint}: Pass the auxiliary array as an argument to the recursive \texttt{sort()}.
	\end{ex}
	\begin{sol}
		See the \texttt{com.segarciat.algs4.ch2.sec2.ex09.Merge} class.
	\end{sol}
	\begin{ex}{10}
		\emph{Faster merge}. Implement a version of \texttt{merge()} that copies the second half of
		\texttt{a[]} to \texttt{aux[]} in \emph{decreasing order} and then does the merge back to \texttt{a[]}.
		This change allows you to remove the code to test that each of the halves has been
		exhausted from the inner loop. \emph{Note}: The resulting sort is not stable
		(see page 341).
	\end{ex}
	\begin{sol}
		See the \texttt{com.segarciat.algs4.ch2.sec2.ex10.FasterMerge} class.
	\end{sol}
	\pagebreak
	\printbibliography
\end{document}