\documentclass[12pt, a4paper]{article}

\usepackage[utf8]{inputenc}
% Limit the page margin to only 1 inch.
\usepackage[margin=1in]{geometry}

%Imports biblatex package
\usepackage[
backend=biber,
style=alphabetic
]{biblatex}
\addbibresource{../../algs4e.bib}

% Enables the `align' environment.
\usepackage{amsmath}
% Provides useful environments, such as:
% - \begin{proof} ...\end{proof}
\usepackage{amsthm}
\usepackage[most]{tcolorbox}

\newtheorem*{proposition}{Proposition}

% Enables using \mathbb{}, for example \mathbb{N} for the set of natural numbers.
\usepackage{amssymb}

% Allows using letters in enumerate list environment. Use, for example:
%\begin{enumerate}[label=(\alph*)]
% ...
%\end{enumerate}
\usepackage[inline]{enumitem}

% Enable importing external graphic files and provides useful commannds, like \graphicspath{}
\usepackage{graphicx}
% Images are located in a directory called images in the current directory.
\graphicspath{{./images/}}

% Make links look better by default.
% See: https://tex.stackexchange.com/questions/823/remove-ugly-borders-around-clickable-cross-references-and-hyperlinks
\usepackage[hidelinks]{hyperref}
\usepackage{xcolor}
\hypersetup{
	colorlinks,
	linkcolor={red!50!black},
	citecolor={blue!50!black},
	urlcolor={blue!80!black}
}


% Code Listings. Source:
% https://stackoverflow.com/questions/3175105/inserting-code-in-this-latex-document-with-indentation
\usepackage{listings}
\usepackage{color}

\definecolor{dkgreen}{rgb}{0,0.6,0}
\definecolor{gray}{rgb}{0.5,0.5,0.5}
\definecolor{mauve}{rgb}{0.58,0,0.82}

\lstset{frame=tb,
	language=Java,
	aboveskip=3mm,
	belowskip=3mm,
	showstringspaces=false,
	columns=flexible,
	basicstyle={\small\ttfamily},
	numbers=none,
	numberstyle=\tiny\color{gray},
	keywordstyle=\color{blue},
	commentstyle=\color{dkgreen},
	stringstyle=\color{mauve},
	breaklines=true,
	breakatwhitespace=true,
	tabsize=3
}

\newcommand{\prob}{\text{P}}
%\newcommand{\complement}{\mathsf{c}}

% Define an environment called "ex" (for Exercise) so that I can do: \begin{ex}{1.5}...\end{ex}
\newenvironment{ex}[2][Exercise]
{\par\medskip\noindent \textbf{#1 #2.}}
{\medskip}

% Define a solution environment, similar to ex (exercise) environment.
\newenvironment{sol}[1][Solution]
{\par\medskip\noindent \textbf{#1.} }
{\medskip}

\begin{document}
	\noindent Sergio E. Garcia Tapia \hfill
	
	\noindent \emph{Algorithms} by Sedgewick and Wayne (4th edition) \cite{sedgewick_wayne}\hfill
	
	\noindent November 2nd, 2024\hfill 
	\section*{2.4: Priority Queues}
	\begin{ex}{1}
		Suppose that the sequence \texttt{P R I O * R * * I * T * Y * * * Q U E * * * U * E}
		(where a letter means \emph{insert} and an asterisk means \emph{remove the maximum})
		is applied to an initially empty priority queue. Give the sequence of letters
		returned by the \emph{remove the maximum} operations.
	\end{ex}
	\begin{sol}
		\begin{lstlisting}[language={}]
P
P R
P R I
P R I O
P I O // max removed: R
P I O R
P I O // max removed: R
I O // max removed: P
I O I
I I // max removed: O
I I T
I I // max removed: T
I I Y
I I // max removed: Y
I // max removed: I
// max removed: I
Q
Q U
Q U E
Q E // max removed:  U
E // max removed: Q
// max removed: E
U
// max removed: U
E
		\end{lstlisting}
		At the end, \texttt{E} remains on the queue. The sequence letters returned is:
		\begin{center}
			\texttt{R R P O T Y I I U Q E U}
		\end{center}
	\end{sol}
	\begin{ex}{2}
		Criticize the following idea: To implement \emph{find the maximum} in constant
		time, why not use a stack  or a queue, but keep track of the maximum value inserted
		so far, then return that value for \emph{find the maximum}.
	\end{ex}
	\begin{sol}
		One issue is that this only guarantees that the first \emph{find the maximum}
		operation can be returned in constant time. Once that items is removed, if the client
		then asks for the next value, this operation then requires linear time.
	\end{sol}
	\begin{ex}{3}
		Provide priority-queue implementations that support \emph{insert} and
		\emph{remove the maximum}, one for each of the following underlying data structures:
		unordered array, ordered array, unordered linked list, and ordered linked list.
		Give a table of the worst-case bounds for each operation for each of your
		four implementations.
	\end{ex}
	\begin{sol}
		See package \texttt{com.segarciat.algs4.ch2.sec4.ex03}. The time complexities
		for the two main operations are given below for priority queue with $n$ items:
		\begin{center}
			\begin{tabular}{c|c|c}
				{} & Insert & Remove the maximum\\
				\hline
				\texttt{UnorderedArrayMaxPQ} & $O(1)$ & $O(n)$ \\
				\hline
				\texttt{OrderedArrayMaxPQ} & $O(n)$ & $O(1)$\\
				\hline
				\texttt{UnorderedListMaxPQ} & $O(1)$ & $O(n)$ \\
				\hline
				\texttt{OrderedListMaxPQ} & $O(n)$ & $O(1)$
			\end{tabular}
		\end{center}
	\end{sol}
	\pagebreak
	\printbibliography
\end{document}