\documentclass[12pt, a4paper]{article}

\usepackage[utf8]{inputenc}
% Limit the page margin to only 1 inch.
\usepackage[margin=1in]{geometry}

%Imports biblatex package
\usepackage[
backend=biber,
style=alphabetic
]{biblatex}
\addbibresource{../../algs4e.bib}

% Enables the `align' environment.
\usepackage{amsmath}
% Provides useful environments, such as:
% - \begin{proof} ...\end{proof}
\usepackage{amsthm}
\usepackage[most]{tcolorbox}

\newtheorem*{proposition}{Proposition}

% Enables using \mathbb{}, for example \mathbb{N} for the set of natural numbers.
\usepackage{amssymb}

% Allows using letters in enumerate list environment. Use, for example:
%\begin{enumerate}[label=(\alph*)]
% ...
%\end{enumerate}
\usepackage[inline]{enumitem}

% Enable importing external graphic files and provides useful commannds, like \graphicspath{}
\usepackage{graphicx}
% Images are located in a directory called images in the current directory.
\graphicspath{{./images/}}

% Make links look better by default.
% See: https://tex.stackexchange.com/questions/823/remove-ugly-borders-around-clickable-cross-references-and-hyperlinks
\usepackage[hidelinks]{hyperref}
\usepackage{xcolor}
\hypersetup{
	colorlinks,
	linkcolor={red!50!black},
	citecolor={blue!50!black},
	urlcolor={blue!80!black}
}


% Code Listings. Source:
% https://stackoverflow.com/questions/3175105/inserting-code-in-this-latex-document-with-indentation
\usepackage{listings}
\usepackage{color}

\definecolor{dkgreen}{rgb}{0,0.6,0}
\definecolor{gray}{rgb}{0.5,0.5,0.5}
\definecolor{mauve}{rgb}{0.58,0,0.82}

\lstset{frame=tb,
	language=Java,
	aboveskip=3mm,
	belowskip=3mm,
	showstringspaces=false,
	columns=flexible,
	basicstyle={\small\ttfamily},
	numbers=none,
	numberstyle=\tiny\color{gray},
	keywordstyle=\color{blue},
	commentstyle=\color{dkgreen},
	stringstyle=\color{mauve},
	breaklines=true,
	breakatwhitespace=true,
	tabsize=3
}

\newcommand{\prob}{\text{P}}
%\newcommand{\complement}{\mathsf{c}}

% Define an environment called "ex" (for Exercise) so that I can do: \begin{ex}{1.5}...\end{ex}
\newenvironment{ex}[2][Exercise]
{\par\medskip\noindent \textbf{#1 #2.}}
{\medskip}

% Define a solution environment, similar to ex (exercise) environment.
\newenvironment{sol}[1][Solution]
{\par\medskip\noindent \textbf{#1.} }
{\medskip}

\begin{document}
	\noindent Sergio E. Garcia Tapia \hfill
	
	\noindent \emph{Algorithms} by Sedgewick and Wayne (4th edition) \cite{sedgewick_wayne}\hfill
	
	\noindent November 10th, 2024\hfill 
	\section*{2.5: Applications}
	\begin{ex}{1}
		Consider the following implementation of the \texttt{complareTo()} method for
		\texttt{String}. How doe the third line help with efficiency?
		\begin{lstlisting}
			public int compareTo(String that)
			{
				if (this == that) return 0; // this line
				int n = Math.min(this.length(), that.length());
				for (int i = 0; i < n; i++)
				{
					if      (this.charAt(i) < that.charAt(i)) return -1;
					else if (this.charAt(i) > that.charAt(i)) return +1;
				}
				return this.length() - that.length();
			}
		\end{lstlisting}
	\end{ex}
	\begin{sol}
		In general, the method is linear in the length of the shortest of the two
		strings. However, it may be that the strings are aliased, so that effectively
		a string is being compared to itself. The indicated line detects this condition
		and reduces the duration of the compare to constant time.
	\end{sol}
	\begin{ex}{2}
		Write a program  that reads a list of words from standard input and prints
		all two-word compound words in the list. For example, if \texttt{after},
		\texttt{thought}, and \texttt{afterthought} are in the list, then
		\texttt{afterthought} is a compound word.
	\end{ex}
	\begin{sol}
		See \texttt{com.segarciat.algs4.ch2.sec5.ex02.TwoWordCompoundWords}.
	\end{sol}
	\begin{ex}{3}
		Criticize the following implementation of a class intended to represent
		account balances. Why is \texttt{compareTo()} a flawed implementation
		of the \texttt{Comparable} interface?
		\begin{lstlisting}
		public class Balance implements Comparable<Balance>
		{
			// ...
			private double amount;
			public int compareTo(Balance that)
			{
				if (this.amount < that.amount - 0.005) return -1;
				if (this.amount > that.amount + 0.005) return +1;
				return 0;
			}
			// ...
		}
		\end{lstlisting}
		Describe a way to fix this problem.
	\end{ex}
	\begin{sol}
		It appears that the implementation is attempting to assert that the two \texttt{Balance}
		instances compare equal when their \texttt{amount} is within \texttt{0.005}.
		For example, this would certify that \texttt{0.10} and \texttt{0.104} are the
		same, presumably both 10 cents. However, numbers of type \texttt{double} are
		known to be subject to rounding errors. Moreover, such an implementation
		does not define a total ordering. For example, suppose we had objects \texttt{a},
		\texttt{b}, and \texttt{c} of type \texttt{Balance}, such that
		\begin{enumerate}[label=(\roman*)]
			\item \texttt{a.amount = 0.097}
			\item \texttt{b.amount = 0.10}
			\item \texttt{c.amount = 0.103}
		\end{enumerate}
		Assuming no rounding errors, we would have \texttt{a.compareTo(b) == 0}
		and \texttt{b.compareTo(c) == 0}, but \texttt{a.compareTo(c) == -1},
		so that we don't have transitivity.
		
		To fix this, we can choose a different representation for the amount.
		We can use two instance variables: one for the amounts smaller than 1
		(for example, the number of cents if we are speaking of dollars), and
		another for the amounts that are 1 or larger (like dollars bills).
		Then the \texttt{compareTo()} method can exactly compare these quantities.
	\end{sol}
	\begin{ex}{4}
		Implement a method \texttt{String[] dedup(String[] a)} that returns the objects
		in \texttt{a[]} in sorted order, with duplicates removed.
	\end{ex}
	\begin{sol}
		See \texttt{com.segarciat.algs4.ch2.sec5.ex04.DeduplicatedStrings}.
	\end{sol}
	\begin{ex}{5}
		Explain why selection sort is not stable.
	\end{ex}
	\begin{sol}
		\cite{sedgewick_wayne} describes a sorting method as \emph{stable}
		if ``it preserves the relative order of equal keys in the array".
		The reason this is so is because at any point, the ``next minimum"
		that the algorithm searches for could be anywhere in the array.
		If the two elements equal elements are adjacent to one another, and
		the ``next minimum" is somewhere to the right of them, then
		they could end up not in relative order.
		
		Considered, for example, the following array:
		\begin{lstlisting}[language={}]
[2] 2 3 4 1
		\end{lstlisting}, and exchanges the \emph{first} 2 to get:
		\begin{lstlisting}[language={}]
1 [2] 3 4 2
		\end{lstlisting}
		Notice that the relative order of the \texttt{2}'s changed. On
		the next iteration, the \texttt{2} in the second place (which
		has not been subject to a swap) stays in place, because no other
		key in the array is smaller than it:
		\begin{lstlisting}[language={}]
1 2 [3] 4 2
		\end{lstlisting}
		Next, the next smallest is the  \texttt{2} at the end, which is swapped
		with the \texttt{3} to get:
		\begin{lstlisting}[language={}]
1 2 2 [4] 3
		\end{lstlisting}
		The elements to the left of the scan pointers are not moved anymore,
		so the \texttt{2}'s do not end up in the same relative order they started
		with.
	\end{sol}
	\begin{ex}{6}
		Implement a recursive version of \texttt{select()}.
	\end{ex}
	\begin{sol}
		See \texttt{com.segarciat.algs4.ch2.sec5.ex06.RecursiveSelect}.
	\end{sol}
	\begin{ex}{7}
		About how many compares are required, on average, to find the smallest
		of $n$ items using \texttt{select()}?
	\end{ex}
	\begin{sol}
		By \texttt{Proposition U}, the average number of compares to find
		the $k$th smallest is $\sim 2n + 2\cdot k\ln (n/k) + 2(n - k)\cdot \ln(n/(n-k))$.
		As $k\to 0$, this quantity approaches $2n$, suggesting the average.
	\end{sol}
	\begin{ex}{8}
		Write a program \texttt{Frequency} that reads strings from standard input and
		prints the number of times each string occurs, in descending order of frequency.
	\end{ex}
	\begin{sol}
		See \texttt{com.segarciat.algs4.ch2.sec5.ex08.Frequency}. I have implemented
		this by using a minimum-oriented priority queue with \texttt{String} objects
		read from standard input, and a max-oriented priority queue with \texttt{StringCountNode}
		objects, a data type I defined that simply holds a \texttt{String} read from
		standard input and its frequency.
	\end{sol}
	\begin{ex}{9}
		Develop a data type that allows you to write a client that can sort a file
		such as the one shown on below:
		\begin{lstlisting}[language={}]
# input (DJIA volumes for each day)
 1-Oct-28	3500000
 2-Oct-28	3850000
 3-Oct-28	4060000
 4-Oct-28	4330000
 5-Oct-28	4360000
...
30-Dec-99	554680000
31-Dec-99	374049984
 3-Jan-00	931800000
 4-Jan-00	1009000000
 5-Jan-00	1085500032
 
# output
  19-Aug-40 130000
  26-AUg-40	160000
  24-Jul-40	200000
  10-Aug-42 210000
  23-Jun-42 210000
  ...
  23-Jul-02 2441019904
  17-Jul-02 2566500096
  15-Jul-02 2574799872
  19-Jul-02 2654099968
  24-Jul-02 2775555936
		\end{lstlisting}
	\end{ex}
	\begin{sol}
		See \texttt{com.segarciat.algs4.ch2.sec5.ex09.DJIAVolume}.
	\end{sol}
	\begin{ex}{10}
		Create a data type \texttt{Version} that represents a software version number,
		such as \texttt{115.1.1}, \texttt{115.10.1}, \texttt{115.10.2}. Implement
		the \texttt{Comparable} interface so that \texttt{115.1.1} is less than
		\texttt{115.10.1}, and so forth.
	\end{ex}
	\pagebreak
	\printbibliography
\end{document}