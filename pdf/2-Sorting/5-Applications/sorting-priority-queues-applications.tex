\documentclass[12pt, a4paper]{article}

\usepackage[utf8]{inputenc}
% Limit the page margin to only 1 inch.
\usepackage[margin=1in]{geometry}

%Imports biblatex package
\usepackage[
backend=biber,
style=alphabetic
]{biblatex}
\addbibresource{../../algs4e.bib}

% Enables the `align' environment.
\usepackage{amsmath}
% Provides useful environments, such as:
% - \begin{proof} ...\end{proof}
\usepackage{amsthm}
\usepackage[most]{tcolorbox}

\newtheorem*{proposition}{Proposition}

% Enables using \mathbb{}, for example \mathbb{N} for the set of natural numbers.
\usepackage{amssymb}

% Allows using letters in enumerate list environment. Use, for example:
%\begin{enumerate}[label=(\alph*)]
% ...
%\end{enumerate}
\usepackage[inline]{enumitem}

% Enable importing external graphic files and provides useful commannds, like \graphicspath{}
\usepackage{graphicx}
% Images are located in a directory called images in the current directory.
\graphicspath{{./images/}}

% Make links look better by default.
% See: https://tex.stackexchange.com/questions/823/remove-ugly-borders-around-clickable-cross-references-and-hyperlinks
\usepackage[hidelinks]{hyperref}
\usepackage{xcolor}
\hypersetup{
	colorlinks,
	linkcolor={red!50!black},
	citecolor={blue!50!black},
	urlcolor={blue!80!black}
}


% Code Listings. Source:
% https://stackoverflow.com/questions/3175105/inserting-code-in-this-latex-document-with-indentation
\usepackage{listings}
\usepackage{color}

\definecolor{dkgreen}{rgb}{0,0.6,0}
\definecolor{gray}{rgb}{0.5,0.5,0.5}
\definecolor{mauve}{rgb}{0.58,0,0.82}

\lstset{frame=tb,
	language=Java,
	aboveskip=3mm,
	belowskip=3mm,
	showstringspaces=false,
	columns=flexible,
	basicstyle={\small\ttfamily},
	numbers=none,
	numberstyle=\tiny\color{gray},
	keywordstyle=\color{blue},
	commentstyle=\color{dkgreen},
	stringstyle=\color{mauve},
	breaklines=true,
	breakatwhitespace=true,
	tabsize=3
}

\newcommand{\prob}{\text{P}}
%\newcommand{\complement}{\mathsf{c}}

% Define an environment called "ex" (for Exercise) so that I can do: \begin{ex}{1.5}...\end{ex}
\newenvironment{ex}[2][Exercise]
{\par\medskip\noindent \textbf{#1 #2.}}
{\medskip}

% Define a solution environment, similar to ex (exercise) environment.
\newenvironment{sol}[1][Solution]
{\par\medskip\noindent \textbf{#1.} }
{\medskip}

\begin{document}
	\noindent Sergio E. Garcia Tapia \hfill
	
	\noindent \emph{Algorithms} by Sedgewick and Wayne (4th edition) \cite{sedgewick_wayne}\hfill
	
	\noindent November 10th, 2024\hfill 
	\section*{2.5: Applications}
	\begin{ex}{1}
		Consider the following implementation of the \texttt{complareTo()} method for
		\texttt{String}. How doe the third line help with efficiency?
		\begin{lstlisting}
			public int compareTo(String that)
			{
				if (this == that) return 0; // this line
				int n = Math.min(this.length(), that.length());
				for (int i = 0; i < n; i++)
				{
					if      (this.charAt(i) < that.charAt(i)) return -1;
					else if (this.charAt(i) > that.charAt(i)) return +1;
				}
				return this.length() - that.length();
			}
		\end{lstlisting}
	\end{ex}
	\begin{sol}
		In general, the method is linear in the length of the shortest of the two
		strings. However, it may be that the strings are aliased, so that effectively
		a string is being compared to itself. The indicated line detects this condition
		and reduces the duration of the compare to constant time.
	\end{sol}
	\begin{ex}{2}
		Write a program  that reads a list of words from standard input and prints
		all two-word compound words in the list. For example, if \texttt{after},
		\texttt{thought}, and \texttt{afterthought} are in the list, then
		\texttt{afterthought} is a compound word.
	\end{ex}
	\begin{sol}
		See \texttt{com.segarciat.algs4.ch2.sec5.ex02.TwoWordCompoundWords}.
	\end{sol}
	\begin{ex}{3}
		Criticize the following implementation of a class intended to represent
		account balances. Why is \texttt{compareTo()} a flawed implementation
		of the \texttt{Comparable} interface?
		\begin{lstlisting}
		public class Balance implements Comparable<Balance>
		{
			// ...
			private double amount;
			public int compareTo(Balance that)
			{
				if (this.amount < that.amount - 0.005) return -1;
				if (this.amount > that.amount + 0.005) return +1;
				return 0;
			}
			// ...
		}
		\end{lstlisting}
		Describe a way to fix this problem.
	\end{ex}
	\begin{sol}
		It appears that the implementation is attempting to assert that the two \texttt{Balance}
		instances compare equal when their \texttt{amount} is within \texttt{0.005}.
		For example, this would certify that \texttt{0.10} and \texttt{0.104} are the
		same, presumably both 10 cents. However, numbers of type \texttt{double} are
		known to be subject to rounding errors. Moreover, such an implementation
		does not define a total ordering. For example, suppose we had objects \texttt{a},
		\texttt{b}, and \texttt{c} of type \texttt{Balance}, such that
		\begin{enumerate}[label=(\roman*)]
			\item \texttt{a.amount = 0.097}
			\item \texttt{b.amount = 0.10}
			\item \texttt{c.amount = 0.103}
		\end{enumerate}
		Assuming no rounding errors, we would have \texttt{a.compareTo(b) == 0}
		and \texttt{b.compareTo(c) == 0}, but \texttt{a.compareTo(c) == -1},
		so that we don't have transitivity.
		
		To fix this, we can choose a different representation for the amount.
		We can use two instance variables: one for the amounts smaller than 1
		(for example, the number of cents if we are speaking of dollars), and
		another for the amounts that are 1 or larger (like dollars bills).
		Then the \texttt{compareTo()} method can exactly compare these quantities.
	\end{sol}
	\begin{ex}{4}
		Implement a method \texttt{String[] dedup(String[] a)} that returns the objects
		in \texttt{a[]} in sorted order, with duplicates removed.
	\end{ex}
	\begin{sol}
		See \texttt{com.segarciat.algs4.ch2.sec5.ex04.DeduplicatedStrings}.
	\end{sol}
	\begin{ex}{5}
		Explain why selection sort is not stable.
	\end{ex}
	\begin{sol}
		\cite{sedgewick_wayne} describes a sorting method as \emph{stable}
		if ``it preserves the relative order of equal keys in the array".
		The reason this is so is because at any point, the ``next minimum"
		that the algorithm searches for could be anywhere in the array.
		If the two elements equal elements are adjacent to one another, and
		the ``next minimum" is somewhere to the right of them, then
		they could end up not in relative order.
		
		Considered, for example, the following array:
		\begin{lstlisting}[language={}]
[2] 2 3 4 1
		\end{lstlisting}, and exchanges the \emph{first} 2 to get:
		\begin{lstlisting}[language={}]
1 [2] 3 4 2
		\end{lstlisting}
		Notice that the relative order of the \texttt{2}'s changed. On
		the next iteration, the \texttt{2} in the second place (which
		has not been subject to a swap) stays in place, because no other
		key in the array is smaller than it:
		\begin{lstlisting}[language={}]
1 2 [3] 4 2
		\end{lstlisting}
		Next, the next smallest is the  \texttt{2} at the end, which is swapped
		with the \texttt{3} to get:
		\begin{lstlisting}[language={}]
1 2 2 [4] 3
		\end{lstlisting}
		The elements to the left of the scan pointers are not moved anymore,
		so the \texttt{2}'s do not end up in the same relative order they started
		with.
	\end{sol}
	\begin{ex}{6}
		Implement a recursive version of \texttt{select()}.
	\end{ex}
	\begin{sol}
		See \texttt{com.segarciat.algs4.ch2.sec5.ex06.RecursiveSelect}.
	\end{sol}
	\begin{ex}{7}
		About how many compares are required, on average, to find the smallest
		of $n$ items using \texttt{select()}?
	\end{ex}
	\begin{sol}
		By \texttt{Proposition U}, the average number of compares to find
		the $k$th smallest is $\sim 2n + 2\cdot k\ln (n/k) + 2(n - k)\cdot \ln(n/(n-k))$.
		As $k\to 0$, this quantity approaches $2n$, suggesting the average.
	\end{sol}
	\begin{ex}{8}
		Write a program \texttt{Frequency} that reads strings from standard input and
		prints the number of times each string occurs, in descending order of frequency.
	\end{ex}
	\begin{sol}
		See \texttt{com.segarciat.algs4.ch2.sec5.ex08.Frequency}. I have implemented
		this by using a minimum-oriented priority queue with \texttt{String} objects
		read from standard input, and a max-oriented priority queue with \texttt{StringCountNode}
		objects, a data type I defined that simply holds a \texttt{String} read from
		standard input and its frequency.
	\end{sol}
	\begin{ex}{9}
		Develop a data type that allows you to write a client that can sort a file
		such as the one shown on below:
		\begin{lstlisting}[language={}]
# input (DJIA volumes for each day)
 1-Oct-28	3500000
 2-Oct-28	3850000
 3-Oct-28	4060000
 4-Oct-28	4330000
 5-Oct-28	4360000
...
30-Dec-99	554680000
31-Dec-99	374049984
 3-Jan-00	931800000
 4-Jan-00	1009000000
 5-Jan-00	1085500032
 
# output
  19-Aug-40 130000
  26-AUg-40	160000
  24-Jul-40	200000
  10-Aug-42 210000
  23-Jun-42 210000
  ...
  23-Jul-02 2441019904
  17-Jul-02 2566500096
  15-Jul-02 2574799872
  19-Jul-02 2654099968
  24-Jul-02 2775555936
		\end{lstlisting}
	\end{ex}
	\begin{sol}
		See \texttt{com.segarciat.algs4.ch2.sec5.ex09.DJIAVolume}.
	\end{sol}
	\begin{ex}{10}
		Create a data type \texttt{Version} that represents a software version number,
		such as \texttt{115.1.1}, \texttt{115.10.1}, \texttt{115.10.2}. Implement
		the \texttt{Comparable} interface so that \texttt{115.1.1} is less than
		\texttt{115.10.1}, and so forth.
	\end{ex}
	\begin{ex}{12}
		\emph{Scheduling}. Write a program \texttt{SPT.java} that reads job names and
		processing times from standard input and prints a schedule that minimizes
		average completion time using the shortest processing time first rule, as described
		on page 349.
	\end{ex}
	\begin{sol}
		See \texttt{com.segarciat.algs4.ch2.sec5.ex12}.
	\end{sol}
	\begin{ex}{13}
		\emph{Load balancing}. Write a program \texttt{LPT.java} that takes an integer
		$m$ as a command-line argument, reads job names and processing times from standard
		input and prints a schedule assigning the jobs to $m$ processors that approximately
		minimizes the time when the last job completes using the longest processing time
		first rule, as described on page 349.
	\end{ex}
	\begin{sol}
		See \texttt{com.segarciat.algs4.ch2.sec5.ex13}.
	\end{sol}
	\begin{ex}{14}
		\emph{Sort by reverse domain}. Write a data type \texttt{Domain} that represents
		domain names, including an appropriate \texttt{compareTo()} where the natural order
		is in order of the \emph{reverse} domain name. For example, the reverse domain name
		of \texttt{cs.princeton.edu} is \texttt{edu.princeton.cs}. This is useful for web
		log analysis. \emph{Hint}: Use \texttt{s.split("\textbackslash\textbackslash.")}
		to split the string \texttt{s} into tokens, delimited by dots. Write a client that
		reads domain names from standard input and prints the reverse domains in sorted
		order.
	\end{ex}
	\begin{sol}
		See \texttt{com.segarciat.algs4.ch2.sec5.ex14}.
	\end{sol}
	\begin{ex}{15}
		\emph{Spam campaign}. To initiate an illegal spam campaign, you have a list of
		email addresses from various domains (the part of the email address that follows
		the \texttt{@} symbol). To better forge the return addresses, you want to send the
		email from another user at the same domain. For example, you might want to forge an
		email from \texttt{wayne@princeton.edu} to \texttt{rs@princeton.edu}. How would you
		process the email list to make this an efficient task?
	\end{ex}
	\begin{sol}
		I would use a priority queue to sort the list by domain. Then I would begin by
		taking address from the queue, which is the first forge senders. I would take
		other address off and as long as the the it is in the same domain as the current
		sender, I would set its sender as the first sender. I would keep track of the
		last sender removed at each step. As soon as the domain of the current domain
		differs from that of the last sender removed, I would arrange for the first
		sender to receive an email from the last sender removed. Then I would update
		the first sender to be the email just removed, the one from the new domain.
	\end{sol}
	\begin{ex}{16}
		\emph{Unbiased election}. In order to thwart bias against candidates whose
		names appear toward the end of the alphabet, California sorted the candidates
		appearing on its 2003 gubernational ballot by using the following order of
		characters:
		\begin{lstlisting}[language={}]
		R W Q O J M V A H B S G Z X N T C I E K U P D Y F L
		\end{lstlisting}
		Create a data type where this is the natural order and write a client \texttt{California}
		with a single static method \texttt{main()} that sorts strings according to this
		ordering. Assume that each string is composed solely of uppercase letters.
	\end{ex}
	
	\begin{ex}{17}
		\emph{Check stability}. Extend your \texttt{check()}method from \textbf{Exercise 2.1.16}
		to call \texttt{sort()} for a given array and return \texttt{true} if \texttt{sort()}
		sorts the array \emph{in a stable manner}, \texttt{false} otherwise. Do not assume that
		\texttt{sort()} is restricted to move data only with \texttt{exchange()}.
	\end{ex}
	\begin{sol}
		See \texttt{com.segarciat.algs4.ch2.sec5.ex17.Stability}.
	\end{sol}
	\begin{ex}{18}
		\emph{Force stability}. Write a wrapper method that makes any sort stable by creating
		a new key type that allows you to append each key's index to the key, call
		\texttt{sort()}, then restore the original key after the sort.
	\end{ex}
	\begin{sol}
		See \texttt{com.segarciat.algs4.ch2.sec5.ex18.ForceStability}.
	\end{sol}
	\begin{ex}{19}
		\emph{Kendall tau distance}. Write a program \texttt{KendallTau.java} that computes
		the Kendall tau distance between two permutations in linearithmic time.
	\end{ex}
	\begin{sol}
		From page 345, we know the following:
		\begin{itemize}
			\item A permutation (rankings) is an array of \texttt{n} integers where each of
			the integers between \texttt{0} and \texttt{n-1} appears exactly once.
			\item The Kendall tau distance between two permutations is the number of pairs
			that are in different order in the two rankings.
			\item The number of inversions in an array is the Kendall tau distance between
			the array and the identity permutation.
		\end{itemize}
		Given permutations \texttt{a} and \texttt{b}, we can have the order of the elements
		in \texttt{b} define the ``sort order". For example, if we have arrays:
		\begin{lstlisting}[language={}]
a: 0 3 1 6 2 5 4
b: 1 0 3 6 4 2 5
		\end{lstlisting}
		Then, according to array \texttt{b}, the sort order is:
		\begin{lstlisting}[language={}]
s: [ 1, 0, 5, 2, 4, 6, 3 ]
		\end{lstlisting}
		Notice that if we consider \texttt{b} as a mapping from the integer indices
		to its array values, then \texttt{s} is the inverse.
		
		Now we can use mergesort to sort \texttt{a} according to the order define
		by \texttt{b} (according to \texttt{s}), counting inversions along the
		way.
	\end{sol}
	\begin{ex}{25}
		\emph{Points in the plane}. Write three \texttt{static} comparators for the
		\texttt{Point2} data type of page 77, one that compares by their $x$ coordinate,
		one that compares by their $y$-coordinate, and one that compares by their distance
		from the origin. Write two non-\texttt{static} comparators for the \texttt{Point2D}
		data type, one that compares them by their distance to a specified point and one
		that compares them by their polar angle with respect to a specified point.
	\end{ex}
	\begin{sol}
		See \texttt{com.segarciat.algs4.ch2.sec5.ex25}.
	\end{sol}
	\begin{ex}{26}
		\emph{Simple polygon}. Given $N$ points in the plane, draw a simple polygon with the
		$N$ points as vertices. \emph{Hint}: Find the point $p$ with the smallest $y$ coordinate,
		breaking ties with the smallest $x$ coordinate. Connect the points in increasing order
		of the polar angle they make with $p$.
	\end{ex}
	\begin{sol}
		See \texttt{com.segarciat.algs4.ch2.sec5.ex26.DrawPolygon}.
	\end{sol}
	\begin{ex}{27}
		\emph{One-dimensional intervals}. Write three comparators for the \texttt{Interval1D}
		data type of page 77, one that compares intervals by their minimum endpoint, one that
		compares intervals by their maximum endpoint, and one that compares intervals by
		their length.
	\end{ex}
	\begin{sol}
		See \texttt{com.segarciat.algs4.ch2.sec5.ex27}.
	\end{sol}
	\begin{ex}{28}
		\emph{Sort files by name}. Write a program \texttt{FileSorter} that takes
		the name of a directory as a command-line argument and prints out all of the files
		in that directory sorted by file name. \emph{Hint}: Use the \texttt{File} data type.
	\end{ex}
	\begin{sol}
		See \texttt{com.segarciat.algs4.ch2.sec5.ex28.FileSorter}.
	\end{sol}
	\begin{ex}{29}
		\emph{Sort files by size and date of last modification}. Write comparators for the
		type \texttt{File} to order by increasing/decreasing order of file size,
		ascending/descending order of file name, and ascending/descending order of last
		modification date. Use these comparators in a program \texttt{LS} that
		takes a command-line argument and lists the file in the current directory according
		to a specified order, e.g., \texttt{"-t"} to sort by timestamp. Support multiple
		flags to break ties. Be sure to use a stable sort.
	\end{ex}
	\begin{sol}
		See \texttt{com.segarciat.algs4.ch2.sec5.ex29.LS}.
	\end{sol}
	\pagebreak
	\printbibliography
\end{document}