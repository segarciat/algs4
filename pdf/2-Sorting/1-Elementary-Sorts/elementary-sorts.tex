\documentclass[12pt, a4paper]{article}

\usepackage[utf8]{inputenc}
% Limit the page margin to only 1 inch.
\usepackage[margin=1in]{geometry}

%Imports biblatex package
\usepackage[
backend=biber,
style=alphabetic
]{biblatex}
\addbibresource{../../algs4e.bib}

% Enables the `align' environment.
\usepackage{amsmath}
% Provides useful environments, such as:
% - \begin{proof} ...\end{proof}
\usepackage{amsthm}
% Enables using \mathbb{}, for example \mathbb{N} for the set of natural numbers.
\usepackage{amssymb}

% Allows using letters in enumerate list environment. Use, for example:
%\begin{enumerate}[label=(\alph*)]
% ...
%\end{enumerate}
\usepackage[inline]{enumitem}

% Enable importing external graphic files and provides useful commannds, like \graphicspath{}
\usepackage{graphicx}
% Images are located in a directory called images in the current directory.
\graphicspath{{./images/}}

% Make links look better by default.
% See: https://tex.stackexchange.com/questions/823/remove-ugly-borders-around-clickable-cross-references-and-hyperlinks
\usepackage[hidelinks]{hyperref}
\usepackage{xcolor}
\hypersetup{
	colorlinks,
	linkcolor={red!50!black},
	citecolor={blue!50!black},
	urlcolor={blue!80!black}
}


% Code Listings. Source:
% https://stackoverflow.com/questions/3175105/inserting-code-in-this-latex-document-with-indentation
\usepackage{listings}
\usepackage{color}

\definecolor{dkgreen}{rgb}{0,0.6,0}
\definecolor{gray}{rgb}{0.5,0.5,0.5}
\definecolor{mauve}{rgb}{0.58,0,0.82}

\lstset{frame=tb,
	language=Java,
	aboveskip=3mm,
	belowskip=3mm,
	showstringspaces=false,
	columns=flexible,
	basicstyle={\small\ttfamily},
	numbers=none,
	numberstyle=\tiny\color{gray},
	keywordstyle=\color{blue},
	commentstyle=\color{dkgreen},
	stringstyle=\color{mauve},
	breaklines=true,
	breakatwhitespace=true,
	tabsize=3
}

\newcommand{\prob}{\text{P}}
%\newcommand{\complement}{\mathsf{c}}

% Define an environment called "ex" (for Exercise) so that I can do: \begin{ex}{1.5}...\end{ex}
\newenvironment{ex}[2][Exercise]
{\par\medskip\noindent \textbf{#1 #2.}}
{\medskip}

% Define a solution environment, similar to ex (exercise) environment.
\newenvironment{sol}[1][Solution]
{\par\medskip\noindent \textbf{#1.} }
{\medskip}

\begin{document}
	\noindent Sergio E. Garcia Tapia \hfill
	
	\noindent \emph{Algorithms} by Sedgewick and Wayne (4th edition) \cite{sedgewick_wayne}\hfill
	
	\noindent October 12th, 2024\hfill 
	\section*{2.1: Elementary Sorts}
	\begin{ex}{1}
		Show, in the style of the example trace with Algorithm 2.1, how selection sort sorts
		the array \texttt{E A S Y Q U E S T I O N}.
	\end{ex}
	\begin{sol}
		\begin{center}
			\begin{tabular}{cc|cccccccccccc}
				\texttt{i} & \texttt{min} & 0 & 1 & 2 & 3 & 4 & 5 & 6 & 7 & 8 & 9 & 10 & 11 \\
				\hline
				{} & {} & E & A & S & Y & Q & U & E & S & T & I & O & N \\
				0 & 1 & E & {\color{red} A} & S & Y & Q & U & E & S & T & I & O & N\\
				1 & 1 & {\color{gray} A} & {\color{red} E} & S & Y & Q & U & E & S & T & I & O & N \\
				2 & 6 & {\color{gray} A} & {\color{gray} E} & S & Y & Q & U & {\color{red} E} & S & T & I & O & N\\
				3 & 9 & {\color{gray} A} & {\color{gray} E} & {\color{gray} E} & Y & Q & U & S & S & T & {\color{red} I} & O & N\\
				4 & 11 & {\color{gray} A} & {\color{gray} E} & {\color{gray} E} & {\color{gray}I} & Q & U & S & S & T & Y & O & {\color{red} N}\\
				5 & 10 & {\color{gray} A} & {\color{gray} E} & {\color{gray} E} & {\color{gray}I} & {\color{gray} N} & U & S & S & T & Y & {\color{red} O} & Q\\
				6 & 11 & {\color{gray} A} & {\color{gray} E} & {\color{gray} E} & {\color{gray}I} & {\color{gray} N} & {\color{gray}O} & S & S & T & Y & U & {\color{red}Q}\\
				7 & 7 & {\color{gray} A} & {\color{gray} E} & {\color{gray} E} & {\color{gray}I} & {\color{gray} N} & {\color{gray}O} & {\color{gray}Q} & {\color{red}S} & T & Y & U & S\\
				8 & 11 & {\color{gray} A} & {\color{gray} E} & {\color{gray} E} & {\color{gray}I} & {\color{gray} N} & {\color{gray}O} & {\color{gray}Q} & {\color{gray}S} & T & Y & U & {\color{red}S}\\
				
				9 & 11 & {\color{gray} A} & {\color{gray} E} & {\color{gray} E} & {\color{gray}I} & {\color{gray} N} & {\color{gray}O} & {\color{gray}Q} & {\color{gray}S} & {\color{gray}S} & Y & U & {\color{red}T}\\
				
				10 & 10 & {\color{gray} A} & {\color{gray} E} & {\color{gray} E} & {\color{gray}I} & {\color{gray} N} & {\color{gray}O} & {\color{gray}Q} & {\color{gray}S} & {\color{gray}S} & {\color{gray}T} & {\color{red}U} & Y\\
				
				11 & 11 & {\color{gray} A} & {\color{gray} E} & {\color{gray} E} & {\color{gray}I} & {\color{gray} N} & {\color{gray}O} & {\color{gray}Q} & {\color{gray}S} & {\color{gray}S} & {\color{gray}T} & {\color{gray}U} & {\color{red}Y}\\
				
				{} & {} & A & E & E & I & N & O & Q & S & S & T & U & Y\\
			\end{tabular}
		\end{center}
	\end{sol}
	\begin{ex}{2}
		What is the maximum number of exchanges involving any particular item during selection sort?
		What is the average number of exchanges involving an item?
	\end{ex}
	\begin{sol}
		Since the algorithms does exactly $n$ exchanges, the maximum number of exchanges is $n$.
		If we have an array of $n$ distinct items, then the maximum is attained when we place the
		largest item in front, followed by the remaining items in ascending order. Then the largest
		item is exchanged $n$ times.
		
		Let $f_i$ be the number of times the $i$th element is involved in an exchange. Since there
		are $n$ items, with exactly $n$ exchanges, and each exchange involves two items, we conclude
		that $\sum_{i=1}^{n}f_i = 2n$. Therefore, the average number of exchanges is
		\begin{align*}
			\text{average \# of exchanges} = \frac{\sum_{i=1}^{n}f_i}{n} = \frac{2n}{n}=2
		\end{align*}
	\end{sol}
	\begin{ex}{3}
		Give  an example of an array of $n$ items that maximizes the number of times
		the test \texttt{a[j] < a[min]} succeeds (and therefore, \texttt{min} gets updated)
		during the operation of selection sort (Algorithm 2.1).
	\end{ex}
	\begin{sol}
		The best example I can think of is an array that is sorted in reverse.
		The first iteration there are $n$ successes, then $n-2$, then, then $n-2$,
		and so on, until we reach the halfway point of the array. At that point,
		none of the comparisons will succeed. In essence, each exchange places two items
		in the final destination at once.
	\end{sol}
	\begin{ex}{4}
		Show, in the style of the example trace with Algorithm 2.2, how insertion sort sorts the
		array \texttt{E A S Y Q U E S T I O N}.
	\end{ex}
	\begin{sol}
		\begin{center}
			\begin{tabular}{cc|cccccccccccc}
				\texttt{i} & \texttt{j} & 0 & 1 & 2 & 3 & 4 & 5 & 6 & 7 & 8 & 9 & 10 & 11 \\
				\hline
				{} & {} & E & A & S & Y & Q & U & E & S & T & I & O & N \\
				
				1 & 0 & {\color{red} A} & {\color{black}E} & {\color{gray}S} & {\color{gray}Y}
				& {\color{gray}Q} & {\color{gray}U} & {\color{gray}{E}} & {\color{gray}S} & {\color{gray}T}
				& {\color{gray}I} & {\color{gray}O} & {\color{gray}N} \\
				
				2 & 2 & {\color{gray} A} & {\color{gray}E} & {\color{red}S} & {\color{gray}Y}
				& {\color{gray}Q} & {\color{gray}U} & {\color{gray}{E}} & {\color{gray}S} & {\color{gray}T}
				& {\color{gray}I} & {\color{gray}O} & {\color{gray}N} \\
				
				3 & 3 & {\color{gray} A} & {\color{gray}E} & {\color{gray}S} & {\color{red}Y}
				& {\color{gray}Q} & {\color{gray}U} & {\color{gray}{E}} & {\color{gray}S} & {\color{gray}T}
				& {\color{gray}I} & {\color{gray}O} & {\color{gray}N} \\
				
				4 & 2 & {\color{gray} A} & {\color{gray}E} & {\color{red}Q} & {\color{black}S}
				& {\color{black}Y} & {\color{gray}U} & {\color{gray}{E}} & {\color{gray}S} & {\color{gray}T}
				& {\color{gray}I} & {\color{gray}O} & {\color{gray}N} \\
				
				5 &4 & {\color{gray} A} & {\color{gray}E} & {\color{gray}Q} & {\color{gray}S}
				& {\color{red}U} & {\color{black}Y} & {\color{gray}{E}} & {\color{gray}S} & {\color{gray}T}
				& {\color{gray}I} & {\color{gray}O} & {\color{gray}N} \\
				
				6 & 2 & {\color{gray} A} & {\color{gray}E} & {\color{red}{E}} & {\color{black}Q} & {\color{black}S}
				& {\color{black}U} & {\color{black}Y} & {\color{gray}S} & {\color{gray}T}
				& {\color{gray}I} & {\color{gray}O} & {\color{gray}N} \\
				
				7 & 5 & {\color{gray} A} & {\color{gray}E} & {\color{gray}{E}} & {\color{gray}Q} & {\color{gray}S} & {\color{red}S}
				& {\color{black}U} & {\color{black}Y} & {\color{gray}T}
				& {\color{gray}I} & {\color{gray}O} & {\color{gray}N} \\
				
				8 & 6 & {\color{gray} A} & {\color{gray}E} & {\color{gray}{E}} & {\color{gray}Q} & {\color{gray}S} & {\color{gray}S} & {\color{red}T}
				& {\color{black}U} & {\color{black}Y} 
				& {\color{gray}I} & {\color{gray}O} & {\color{gray}N} \\
				
				9 & 3 & {\color{gray} A} & {\color{gray}E} & {\color{gray}{E}} &  {\color{red}I} &
				{\color{black}Q} & {\color{black}S} & {\color{black}S} & {\color{black}T}
				& {\color{black}U} & {\color{black}Y} 
				& {\color{gray}O} & {\color{gray}N} \\
				
				10 & 4 & {\color{gray} A} & {\color{gray}E} & {\color{gray}{E}} &  {\color{gray}I} &
				{\color{red}O} & {\color{black}Q} & {\color{black}S} & {\color{black}S} & {\color{black}T}
				& {\color{black}U} & {\color{black}Y} 
				& {\color{gray}N} \\
				
				11 & 4 & {\color{gray} A} & {\color{gray}E} & {\color{gray}{E}} &  {\color{gray}I}
				& {\color{red}N} &
				{\color{black}O} & {\color{black}Q} & {\color{black}S} & {\color{black}S} & {\color{black}T}
				& {\color{black}U} & {\color{black}Y}  \\
				
				{} & {} & E & A & S & Y & Q & U & E & S & T & I & O & N \\
			\end{tabular}
		\end{center}
	\end{sol}
	\begin{ex}{5}
		For each of the two conditions in the inner \texttt{for} loop in insertion sort (Algorithm 2.2),
		describe an array of $n$ items where that condition is always false when the loop terminates.
	\end{ex}
	\begin{sol}
		The condition \texttt{less(a[j], a[j-1])} is always \texttt{false} when the loop
		terminates for an array whose first element is the smallest element.
		For the condition that \texttt{j > 0}, the loop always terminates when it is
		\texttt{false} for an array that is in reverse order.
	\end{sol}
	\begin{ex}{6}
		Which method runs faster for an array with all keys identical, selection sort or insertion sort?
	\end{ex}
	\begin{sol}
		Selection sort is quadratic regardless of the input. Insertion sort is linear for such
		an array. Therefore, insertion sort is faster.
	\end{sol}
	\begin{ex}{7}
		Which method runs faster for an array in reverse order, selection sort or insertion sort?
	\end{ex}
	\begin{sol}
		For such an array, the number of inversions are maximized, so it is the worst-case for
		insertion sort. As a result, it leads to $~\sim n^2/2$  exchanges. Meanwhile, selection
		sort always requires exactly $n$ exchanges. The number of compares is similar.
		In this case, selection sort is better.
	\end{sol}
	\begin{ex}{8}
		Suppose that we use insertion sort on a randomly ordered array where items have only one
		of three values. Is the running time linear, quadratic, or somewhere in between?
	\end{ex}
	\begin{sol}
		If all three items have equal frequency, and the array is sorted in reverse,
		then the number of inversions is quadratic in the array size. A randomly ordered
		array would have about half as many inversions, and hence would still be
		quadratic in performance.
	\end{sol}
	\begin{ex}{9}
		Show, in the style of the example trace with Algorithm 2.3, how shellsort sorts
		the array \texttt{E A S Y S H E L L S O R T Q U E S T I O N}.
	\end{ex}
	\begin{sol}
		No, this is a very long exercise.
	\end{sol}
	\begin{ex}{10}
		Why not use selection sort for $h$-sorting in shellsort?
	\end{ex}
	\begin{sol}
		Selection sort does not take advantage of the order that shellsort has attained
		through taking care of inversions at the intermediate stages.
	\end{sol}
	\begin{ex}{11}
		Implement a version of shellsort that keeps the increment sequence in an array
		rather than computing it.
	\end{ex}
	\begin{sol}
		A decided a \texttt{Stack} was more natural so I did that instead of an array.
		See the \texttt{com.segarciat.algs4.ch2.sec1.ex11.Shell} class.
	\end{sol}
	\begin{ex}{12}
		Instrument shellsort to print the number of compares divided by the array length for
		each increment. Write a test client that tests the hypothesis that this number is
		a small constant, by sorting arrays of random \texttt{Double} values, using array
		lengths that are increasing powers o f 10, starting at 100.
	\end{ex}
	\begin{sol}
		See the class \texttt{com.segarciat.algs4.ch2.sec1.ex12.ShellCompares}.
	\end{sol}
	\begin{ex}{13}
		\emph{Deck sort}. Explain how you would put a deck of cards in order by suit
		(in the order spades, hearts, clubs, diamonds) and by rank within each suit, with the
		restriction that the cards must be laid out face down in a row, and the only allowed
		operations are to check the values of two cards and to exchange two cards
		(keeping them face down).
	\end{ex}
	\begin{sol}
		I would employ the insertion sort algorithm, which is:
		\begin{enumerate}
			\item Start at the second card, flipping that card and the one immediately
			next to it to the left.
			\item If the suits are different and out of order, exchange them. Move left
			one pair and continue until this condition is false. Otherwise check the
			next condition.
			\item Otherwise, if the suits are the same and are out of order, exchange them.
			Move left one pair and flip two cards again, until this condition is false.
			Otherwise, check the next condition.
			\item Otherwise, leave them as is and move on to to the next pair to the right of them.
			Now move to the third position in the row and start over.
			\item ...
			\item Move to the $i$ th position in the row and start over.
		\end{enumerate}
	\end{sol}
	\begin{ex}{14}
		\emph{Deque sort}. Explain how you would sort a deck of cards, with the restriction that
		the only allowed operations are to look at the values of the top two cards, to exchange
		the top two cards, and to move the top card to the bottom of the deck.
	\end{ex}
	\begin{sol}
		\begin{enumerate}
			\item Inspect the top to wards.
			\begin{enumerate}
				\item If they are of different suits and the second card is lower (for example,
				the top card is hearts and the second is spades), swap them.
				\item Otherwise, if they are of the same suit and the top card is of higher
				rank, swap them.
				\item Otherwise, send the card top card to the bottom of the deck.
			\end{enumerate}
			\item Repeat until sorted.
		\end{enumerate}
	\end{sol}
	\begin{ex}{15}
		\emph{Expensive exchange}. A clerk at a shipping company is charged with the task of rearranging
		a number of large crates in order of the time they are to be shipped out. Thus, the cost
		of compares is very log (just look at the two labels) relative to the cost of exchanges
		(move the crates). The warehouse is nearly full --- there is extra space sufficient to hold
		any one of he crates, but not two. What sorting method should the clerk use?
	\end{ex}
	\begin{sol}
		Selection sort. It uses a minimal number of exchanges.
	\end{sol}
	\begin{ex}{17}
		\emph{Animation}. Add code to \texttt{Insertion}, \texttt{Selection}, and \texttt{Shell}
		to make them draw the array contents as vertical bars, like the visual traces in this
		section, redrawing the bars after each pass, to produce an animated effect, ending in
		a ``sorted" picture where the bars appear in the order of height.
		\emph{Hint}: Use a client like the one in the text that generates random \texttt{Double}
		values, insert calls to \texttt{show()} as appropriate in the sort code, and implement
		a \texttt{show()} method that clears the canvas and draws the bars.
	\end{ex}
	\begin{sol}
		See the classes in the \texttt{com.segarciat.algs4.ch2.sec1.ex17} package.
		The programs \texttt{Selection}, \texttt{Insertion}, and \texttt{Shell}
		use common functionality from the \texttt{SortUtil} class. The
		assumption is that they are working with random \texttt{double} values
		between 0 and 1.
	\end{sol}
	\begin{ex}{24}
		\emph{Insertion sort with sentinel}. Develop an implementation of insertion sort
		that eliminates the \texttt{j>0} test in the inner loop by first putting the
		smallest item into position. Use \texttt{SortCompare} to evaluate the effectiveness
		of doing so. \emph{Note}: It is often possible to avoid an index-out-of-bounds
		test i this way --- the element that enables the test to be eliminated is known
		as a \emph{sentinel}.
	\end{ex}
	\begin{sol}
		See the \texttt{com.segarciat.algs4.ch2.sec1.ex24.InsertionSentinel} class.
	\end{sol}
	\begin{ex}{25}
		\emph{Insertion sort without exchanges}. Develop an implementation of insertion
		sort that moves larger elements to the right one position with one array access
		per entry, rather than using \texttt{exchange()}.
		Use \texttt{SortCompare} to evaluate the effectiveness of doing so.
	\end{ex}
	\begin{sol}
		See the \texttt{com.segarciat.algs4.ch2.sec1.ex25.InsertionHalfExchanges} class.
	\end{sol}
	\pagebreak
	\printbibliography
\end{document}