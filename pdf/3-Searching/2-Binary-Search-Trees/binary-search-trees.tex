\documentclass[12pt, a4paper]{article}

\usepackage[utf8]{inputenc}
% Limit the page margin to only 1 inch.
\usepackage[margin=1in]{geometry}

%Imports biblatex package
\usepackage[
backend=biber,
style=alphabetic
]{biblatex}
\addbibresource{../../algs4e.bib}

% Enables the `align' environment.
\usepackage{amsmath}
% Provides useful environments, such as:
% - \begin{proof} ...\end{proof}
\usepackage{amsthm}
\usepackage[most]{tcolorbox}

\newtheorem*{proposition}{Proposition}

% Enables using \mathbb{}, for example \mathbb{N} for the set of natural numbers.
\usepackage{amssymb}

% Allows using letters in enumerate list environment. Use, for example:
%\begin{enumerate}[label=(\alph*)]
% ...
%\end{enumerate}
\usepackage[inline]{enumitem}

% Enable importing external graphic files and provides useful commannds, like \graphicspath{}
\usepackage{graphicx}
% Images are located in a directory called images in the current directory.
\graphicspath{{./images/}}

% Make links look better by default.
% See: https://tex.stackexchange.com/questions/823/remove-ugly-borders-around-clickable-cross-references-and-hyperlinks
\usepackage[hidelinks]{hyperref}
\usepackage{xcolor}
\hypersetup{
	colorlinks,
	linkcolor={red!50!black},
	citecolor={blue!50!black},
	urlcolor={blue!80!black}
}


% Code Listings. Source:
% https://stackoverflow.com/questions/3175105/inserting-code-in-this-latex-document-with-indentation
\usepackage{listings}
\usepackage{color}

\definecolor{dkgreen}{rgb}{0,0.6,0}
\definecolor{gray}{rgb}{0.5,0.5,0.5}
\definecolor{mauve}{rgb}{0.58,0,0.82}

\lstset{frame=tb,
	language=Java,
	aboveskip=3mm,
	belowskip=3mm,
	showstringspaces=false,
	columns=flexible,
	basicstyle={\small\ttfamily},
	numbers=none,
	numberstyle=\tiny\color{gray},
	keywordstyle=\color{blue},
	commentstyle=\color{dkgreen},
	stringstyle=\color{mauve},
	breaklines=true,
	breakatwhitespace=true,
	tabsize=3
}

\newcommand{\prob}{\text{P}}
%\newcommand{\complement}{\mathsf{c}}

% Define an environment called "ex" (for Exercise) so that I can do: \begin{ex}{1.5}...\end{ex}
\newenvironment{ex}[2][Exercise]
{\par\medskip\noindent \textbf{#1 #2.}}
{\medskip}

% Define a solution environment, similar to ex (exercise) environment.
\newenvironment{sol}[1][Solution]
{\par\medskip\noindent \textbf{#1.} }
{\medskip}

\begin{document}
	\noindent Sergio E. Garcia Tapia \hfill
	
	\noindent \emph{Algorithms} by Sedgewick and Wayne (4th edition) \cite{sedgewick_wayne}\hfill
	
	\noindent December 1st, 2024\hfill 
	\section*{3.2: Binary Search Trees}
	\begin{ex}{1}
		Draw the BST that results when you insert the keys \texttt{E A S Y Q U E S T I O N},
		in that order (associating the value \texttt{i} with the \texttt{i}th key, as per the
		convention in the text) into an initially empty tree. How many compares are needed
		to build the tree?
	\end{ex}
	\begin{sol}
		The sequence of binary search trees is shown in Figure~\ref{fig:ex-01}. The boxes next
		two the root show the number of compares necessary to complete the insertion at
		that step. The sequence of compare counts is: 0, 1, 1, 2, 2, 3, 1, 2, 4, 3, 4, 5.
		Adding gives a total of 28 compares.
		\begin{figure}
			\centering
			\includegraphics[width=0.7\textwidth]{exercise-01}
			\caption{Exercise 1: Sequence of trees formed when inserting the keys \texttt{E A S Y Q U E S T I O N}
			into a binary search tree.}
			\label{fig:ex-01}
		\end{figure}
	\end{sol}
	\begin{ex}{2}
		Inserting the keys in the order \texttt{A X C S E R H} into an initially empty BST
		gives a worst-case tree where every node has one null link, except one at the bottom,
		which has two null links. Give five other orderings of these keys that produce
		worst-case trees. 
	\end{ex}
	\begin{sol}
		We can insert them in the following ways:
		\begin{enumerate}[label=(\roman*)]
			\item \texttt{A X C S E R H} (given).
			\item \texttt{X S R H E C A}.
			\item \texttt{A C E H R S X}.
			\item \texttt{X A S C R E H}.
			\item \texttt{X A S R H E C}.
			\item \texttt{A X C E H R S}.
		\end{enumerate}
		See Figure~\ref{fig:ex-02}.
		\begin{figure}
			\centering
			\includegraphics[width=0.7\textwidth]{exercise-02}
			\caption{Exercise 2: Worst-case binary search trees created by inserting the
			keys \texttt{A X C S E R H} into an initially empty tree in a particular order}
			\label{fig:ex-02}
		\end{figure}
	\end{sol}
	\begin{ex}{3}
		Give five orderings of the keys \texttt{A X C S E R H} that, when inserted into an
		initially empty BST, produce the \emph{best-case} tree.
	\end{ex}
	\begin{sol}
		\begin{enumerate}[label=(\roman*)]
			\item \texttt{H C S A E R X}.
			\item \texttt{H S C A E R X}.
			\item \texttt{H S C E A X R}.
			\item \texttt{H S R X C A E}.
			\item \texttt{H S X R C A E}.
		\end{enumerate}
		These all result in the same tree, depicted in Figure~\ref{fig:ex-03}.
		\begin{figure}
			\centering
			\includegraphics[width=0.3\textwidth]{exercise-03}
			\caption{Exercise 3: Best-case binary search trees created by inserting the
				keys \texttt{A X C S E R H} into an initially empty tree in a particular order}
			\label{fig:ex-03}
		\end{figure}
	\end{sol}
	\begin{ex}{4}
		Suppose that a certain BST has keys that are integers between \texttt{1} and \texttt{10},
		and we search for \texttt{5}. Which sequence below \emph{cannot} be the sequence
		of keys examined?
		\begin{enumerate}[label=(\alph*)]
			\item \texttt{10, 9, 8, 7, 6, 5}
			\item \texttt{4, 10, 8, 6, 5}
			\item \texttt{1, 10, 2, 9, 3, 8, 4, 7, 6, 5}
			\item \texttt{2, 7, 3, 8, 4, 5}
			\item \texttt{1, 2, 10, 4, 8, 5}
		\end{enumerate}
	\end{ex}
	\begin{sol}
		\begin{enumerate}[label=(\alph*)]
			\item This is a valid sequence. For example, this could be a tree where the
			keys were inserted in descending order.
			\item This is a valid sequence.
			\item This is a valid sequence.
			\item This sequence cannot happen. The sequence of compares suggests the following:
			\begin{enumerate}[label=(\roman*)]
				\item 2 is the root, and 5 is larger, so we go right.
				\item 5 is smaller than 7, so we go left.
				\item 5 is smaller than 3, so we go right.
				\item 5 is compared against 8.
			\end{enumerate}
			This last step is impossible because since 8 is larger than 7, but it
			appears on the subtree rooted at its left child, consisting of smaller
			keys,  a contradiction.
			\item This is a valid sequence.
		\end{enumerate}
	\end{sol}
		\begin{ex}{5}
		Suppose that we have an estimate ahead of time of how often search keys are
		to be accessed in a BST, and the freedom to insert them in any order that we
		desire. Should the keys be inserted into the tree in increasing order, decreasing
		order of likely frequency of access, or some other order? Explain your answer.
	\end{ex}
	\begin{sol}
		Increasing order is not desirable because it leads to a worst-case tree where
		every internal node hade has 1 null link and the leaf node has 2 null links
		(in essence, a linked list).
		
		Decreasing order of likely frequency of access biases towards the most frequency
		accessed nodes. However, if that order corresponds to the increasing or decreasing
		order of the keys, for example, then we may again have a worst-case tree.
		
		The safest thing to do is to insert them in an order that yields a balanced tree.
		Suppose that there are $n$ keys, and suppose we place them in sorted order,
		$k_0,k_1,\ldots,k_{n-1}$, so that $k_{i-1} \leq k_{i}$ for all $1\leq i\leq n$.
		Let \texttt{lo = 0}, and \texttt{hi = n - 1}
		Then we can insert them as follows:
		\begin{enumerate}[label=(\roman*)]
			\item If \texttt{lo > hi}, stop.
			\item Compute the middle key at index \texttt{mid = lo + (hi - lo) / 2}.
			This key $k_{mid}$ is the root.
			\item Set the left child of $k_{mid}$ to the tree resulting from (recursively) applying
			the algorithm to the subset of keys $k_0,\ldots,k_{mid-1}$.
			\item Set the right child of $k_{mid}$ to the tree resulting from (recursively) applying
			the algorithm to the subset of keys $k_{mid+1},\ldots,k_{hi}$.
		\end{enumerate}
		Notice this is effectively placing the partitioning the array like quicksort.
	\end{sol}
	\begin{ex}{6}
		Add to \texttt{BST} a method \texttt{height()} that computes the height of the tree.
		Develop two implementations: a recursive method (which takes linear time and space
		proportional to the height), and a method like \texttt{size()} that adds a field to
		each node in the tree (and takes linear space and constant time per query).
	\end{ex}
	\begin{sol}
		See \texttt{com.segarciat.algs4.ch3.sec2.ex06.BST}.
	\end{sol}
	\begin{ex}{7}
		Add to \texttt{BST} a recursive method \texttt{avgCompares()} that computes the average
		number of compares reuired by a random search hit in a given BST (the internal path length
		of the tree divided by its size, plus one). Develop two implementations: a recursive method
		(which takes linear time and space proportional to the height), and a method like
		\texttt{size()} that adds a field to each node in the tree (and takes linear space and
		constant time per query).
	\end{ex}
	\begin{sol}
		See \texttt{com.segarciat.algs4.ch3.sec2.ex07.BST}.
	\end{sol}
	\begin{ex}{8}
		Write a static method \texttt{optCompares()} that takes an integer argument \texttt{n}
		and computes the number of compares required by a random search hit in an optional
		(perfectly balanced) BST with \texttt{n} nodes, where all the null links are on the
		same level if the number of links is a power of 2 or one of two levels otherwise.
	\end{ex}
	\begin{sol}
		See \texttt{com.segarciat.algs4.ch3.sec2.ex08.BSTOptCompares}.
	\end{sol}
	\begin{ex}{9}
		Draw all the different BST shapes that can result when \texttt{n} keys are inserted
		into an initially empty tree, for \texttt{n} = \texttt{2, 3, 4, 5}, and \texttt{6}.
	\end{ex}
	\begin{sol}
		See Figure~\ref{fig:ex-09} for \texttt{n = 2, 3, 4}. I considered the reflections
		on the $y$-axis to be of the same shape, so I chose not to draw them.
		\begin{figure}
			\centering
			\includegraphics[width=0.6\textwidth]{exercise-09}
			\caption{Exercise 9: All binary-search tree shapes for $n=2,3,4$ vertices.}
			\label{fig:ex-09}
		\end{figure}
	\end{sol}
	\begin{ex}{10}
		Write a test client for \texttt{BST} that tests the implementations of \texttt{min()},
		\texttt{max()}, \texttt{floor()}, \texttt{ceiling()}, \texttt{select()}, \texttt{rank()},
		\texttt{delete()}, \texttt{deleteMin()}, \texttt{deleteMax()}, and \texttt{keys()} that
		are given in the text. Start with the standard indexing client given on page 370.
		Add code to take additional command-line arguments, as appropriate.
	\end{ex}
	\begin{ex}{11}
		How many binary tree shapes of $n$ nodes are there with height $n$? How many different
		ways are there to insert $n$ different keys into an initially empty BST that result
		in a tree of height $n-1$? (See \textbf{Exercise 3.2.2})
	\end{ex}
	\begin{sol}
		There are 0 binary tree shapes of $n$ nodes with height $n$, because the height
		of a tree is the maximum depth among all of the nodes, and the smallest node
		is that of the root which is $0$.
		
		However, there are $2^{n-1}$ binary ways to insert $n$ different keys into an
		initially empty BST that results in a tree of height $n-1$. A path from the root
		to the leaf in such a tree of height $n-1$ consists of a sequence of left and right
		turns. At each node, we make a decision of whether to go left or right. Since there
		are $n-1$ edges, and we choose to lean left or right at each step, we get $2^{n-1}$.
	\end{sol}
	\begin{ex}{12}
		Develop a \texttt{BST} implementation that omits \texttt{rank()} and \texttt{select()}
		and does not use a count field in \texttt{Node}.
	\end{ex}
	\begin{sol}
		See \texttt{com.segarciat.algs4.ch3.sec2.ex11.BST}.
	\end{sol}
	\begin{ex}{13}
		Give nonrecursive implementations of \texttt{get()} and \texttt{put()} for \texttt{BST}.
		The implementation of \texttt{put()} is more complicated because of the need to save
		a pointer to the parent node to link in the new node at the bottom. Also, you need
		a separate pass to check whether the key is already in the table because of the need
		to update the counts. Since there are many more searches than inserts in performance-critical
		implementations, using this code for \texttt{get()} is justified; the corresponding change
		for \texttt{put()} might not be noticed.
	\end{ex}
	\begin{sol}
		See \texttt{com.segarciat.algs4.ch3.sec2.ex13.BSTNonRec}.
	\end{sol}
	\begin{ex}{14}
		Give nonrecursive implementations of \texttt{min()}, \texttt{max()}, \texttt{floor()},
		\texttt{ceiling()}, \texttt{rank()}, \texttt{select()}, and \texttt{keys()}.
	\end{ex}
	\begin{sol}
		See \texttt{com.segarciat.algs4.ch3.sec2.ex14.BSTNonRec}.
	\end{sol}
	\begin{ex}{15}
		Give the sequences of nodes examined when the methods in \texttt{BST} are used to
		compute each of the following quantities for the tree drawn in Figure~\ref{fig:ex-15}.
		\begin{figure}
			\centering
			\includegraphics[width=0.4\textwidth]{exercise-15}
			\caption{Exercise 15: Binary Search Tree}
			\label{fig:ex-15}
		\end{figure}
		\begin{enumerate}[label=(\alph*)]
			\item \texttt{floor("Q")}
			\item \texttt{select(5)}
			\item \texttt{ceiling("Q")}
			\item \texttt{rank("J")}
			\item \texttt{size("D", "T")}
			\item \texttt{keys("D", "T")}
		\end{enumerate}
	\end{ex}
	\begin{sol}
		\begin{enumerate}[label=(\alph*)]
			\item \texttt{"E"}, \texttt{"Q"}
			\item \texttt{"E"}, \texttt{"Q"}
			\item \texttt{"E"}, \texttt{"Q"}
			\item \texttt{"E"}, \texttt{"Q"}, \texttt{"J"}
			\item First the \texttt{contains(hi)} call leads to the following
			sequence: \texttt{"E"}, \texttt{"Q"}, \texttt{"T"}. Then the call
			to \texttt{rank(hi)} leads to the sequence \texttt{"E"}, \texttt{"Q"},
			\texttt{"T"}. Then, the call \texttt{rank(lo)} leads to the sequence
			\texttt{"E", "D"}.
			\item \texttt{"E"}, \texttt{"D"}, \texttt{"Q"}, \texttt{"J"}, \texttt{"M"},
			\texttt{"T"}
		\end{enumerate}
	\end{sol}
	\begin{ex}{16}
		Define the \emph{external path length} of a tree to be the sum of the number of
		nodes on the paths from the root to all null links. Prove that the difference between
		the external and internal path length in any binary tree with $n$ nodes is $2n$
		(see \textbf{Proposition C}).
	\end{ex}
	\begin{sol}
		\begin{proof}
			Let $E_n$ represent the external path length of a tree of size $n$,
			and $I_n$ represent the internal path length of a tree of size $n$,
			where $n$ is a positive integer. The proof is by induction on $n$.
			
			For the base case, let $n=1$. Since the root has one node, that node
			is the root at depth $0$, and the internal node path length is $0$.
			The root has two null links, and since the root is the only node on
			the path to them, it follows that the external path length is $2$.
			Since $I_1=0$, $E_1=2$, and $E_1-I_1=2=2\cdot 1$, we see that the base
			case holds.
			
			For the inductive case, suppose we have a tree of $n\geq 1$ nodes,
			and that $E_n-I_n=2n$. Suppose that we add a new node, increasing
			the tree size to $n+1$. If $k$ is the depth of the new node, then
			the internal path length of the tree is now $E_{n+1}=I_n+k$.
			
			When the new node was inserted, it replaced a null link, which contributed precisely
			$k$ to the external path length; on the other hand, the new node
			contributed two new null links to the tree, and reaching each requires
			traversing through $k+1$ nodes. Hence, the external path length
			of the new tree is $E_{n+1}=E_n-k+2(k+1)=E_n+k+2$. Now see that
			\begin{align*}
				E_{n+1}-I_{n+1}&=(E_n+k+2)-(I_n+k)\\
				&=(E_n-I_n)+2\\
				&=2n+2\\
				&=2(n+1),
			\end{align*}
			so the inductive step is asserted. By the principle of mathematical induction,
			the result now holds for all natural numbers.
		\end{proof}
	\end{sol}
	\begin{ex}{17}
		Draw the sequence of BSTs that results when you delete the keys from the tree of
		\textbf{Exercise 3.2.1}, one by one, in the order they were inserted.
	\end{ex}
	\begin{sol}
		See Figure~\ref{fig:ex-17}.
		\begin{figure}
			\centering
			\includegraphics[width=0.7\textwidth]{exercise-17}
			\caption{Exercise 17: Sequence of binary trees produced when deleting
			the keys from Exercise 1 in the order they were inserted.}
			\label{fig:ex-17}
		\end{figure}
	\end{sol}
	\begin{ex}{18}
		Draw the sequence of BSTs that results when you delete the keys of \textbf{Exercise 3.2.1},
		one by one, in alphabetical order.
	\end{ex}
	\begin{sol}
		See Figure~\ref{fig:ex-18}.
		\begin{figure}
			\centering
			\includegraphics[width=0.7\textwidth]{exercise-18}
			\caption{Exercise 18: Sequence of binary trees produced when deleting
				the keys from Exercise 1 in alphabetical order.}
			\label{fig:ex-18}
		\end{figure}
	\end{sol}
	\begin{ex}{19}
		Draw the sequence of BSTs that results when you delete the keys from the tree
		of \textbf{Exercise 3.2.1}, one by one, by successively deleting the key at the
		root.
	\end{ex}
	\begin{sol}
		See Figure~\ref{fig:ex-19}.
		\begin{figure}
			\centering
			\includegraphics[width=0.7\textwidth]{exercise-19}
			\caption{Exercise 19: Sequence of binary trees produced when successively
				deleting the roots from the final tree in Exercise 1 .}
			\label{fig:ex-19}
		\end{figure}
	\end{sol}
	\begin{ex}{20}
		Prove that the running time of the two-argument \texttt{keys()} in a BST
		is at most proportional to the tree height plus the number of keys in the
		range.
	\end{ex}
	\begin{sol}
		\begin{proof}
			Suppose that the two arguments of \texttt{keys()} are \texttt{lo}
			and \texttt{hi}. The recursive depth is dictated by the ceiling
			of \texttt{lo} or the floor of \texttt{hi}.
			
			The algorithm descends into a left or right subtree effectively searching
			for \texttt{ceiling(lo)} and \texttt{floor(hi)}, taking time proportional
			to the tree height in the worst case. Once found, further recursive calls
			that lead to keys out of range are skipped, and the recursive calls return
			to their caller to complete the inorder traversal. This means that the
			duration of the traversal corresponds directly to the number of keys
			in the range \texttt{lo..hi}.
			
			Altogether, the running time of the two-argument \texttt{keys()}
			is bounded by the tree height plus the number of keys in the search
			range \texttt{lo..hi}.
		\end{proof}
	\end{sol}
	\begin{ex}{21}
		Add a BST method \texttt{randomKey()} that returns a random key from the symbol table
		in time proportional to the tree height, in +the worst case.
	\end{ex}
	\begin{sol}
		We can easily implement this by using \texttt{select()}, passing as
		the rank \texttt{k} a random integer between 0 and \texttt{size() - 1};
		see \texttt{com.segarciat.algs4.ch3.sec2.ex21.BST}.
	\end{sol}
	\begin{ex}{22}
		Prove that if a node in a BST has two children, its successor has no left child
		and its predecessor has no right child.
	\end{ex}
	\begin{sol}
		\begin{proof}
			Suppose \texttt{x} is a node in a BST with left child \texttt{y} and right child
			\texttt{z}. The successor is defined to be the next largest node in the binary
			tree. Since nodes with keys larger than the key of \texttt{x} are in the right
			subtree of \texttt{x}, this means the successor of \texttt{x} is in the subtree
			rooted at \texttt{z}.
			
			Suppose the successor of \texttt{x} is called \texttt{z*} By definition of the
			successor, there are no nodes with keys between \texttt{x} and \texttt{z*}, and
			hence no nodes with keys smaller than the key of \texttt{z*} but larger than the
			key of \texttt{x}. This implies that the key of \texttt{z*} is the minimum of the
			keys of the nodes in the subtree rooted at \texttt{z}. Since \texttt{z*} has the
			minimum key of all nodes in the subtree rooted at \texttt{z}, it has no left
			child (otherwise, its left child would have the minimum key).
			
			Similarly, the predecessor
			of \texttt{x} is the maximum node \texttt{y*} of the subtree rooted at its left child \texttt{y}. By a similar reasoning \texttt{y*} has no right child.
		\end{proof}
	\end{sol}
	\begin{ex}{23}
		Is \texttt{delete()} commutative? (Does deleting \texttt{x}, then \texttt{y} give
		the same result as deleting \texttt{y}, the \texttt{x})?
	\end{ex}
	\begin{ex}{24}
		Prove that no compare-based algorithm can build a BST using fewer than $\lg n!\sim n\lg n$
		compares.
	\end{ex}
	\begin{sol}
		\begin{proof}
			Suppose that a compared-based algorithm was able to build a BST using
			fewer than $\lg n!\sim n\lg n$ compares. Since the keys in the BST are sorted,
			this implies that the algorithm can sort $n$ items with fewer than
			$\lg n!$ compares, contradicting \textbf{Proposition I} in Section 2.2
			of \cite{sedgewick_wayne}.
		\end{proof}
	\end{sol}
	\begin{ex}{25}
		\emph{Perfect balance}. Write a program that inserts a set of keys into an initially
		empty BST such that the tree produced is equivalent to binary search, in the
		sense that the sequence of compares done in the search for any key in the BST
		is the same as the sequence of compares used by binary search for the same
		key.
	\end{ex}
	\begin{sol}
		During binary search, the middle key of the current range is compared at
		each step against the search key. See \texttt{com.segarciat.algs4.ch3.sec2.ex25.PerfectBalance}.
	\end{sol}
	\begin{ex}{27}
		\emph{Memory usage}. Compare the memory usage of \texttt{BST} with the memory usage
		of \texttt{BinarySearchST} and \texttt{Sequential SearchST} for $n$ key-value pairs,
		under the assumptions described in Section 1.4 (see \textbf{Exercise 3.1.21}).
		Do not count memory for the keys and values themselves, but do count references to them.
		Then draw a diagram that depicts the precise memory usage of a BST with \texttt{String}
		keys and \texttt{Integer} values (such as the ones built by \texttt{FrequencyCounter}),
		and then estimate the memory usage (in bytes) for the BST built when \texttt{FrequencyCounter}
		uses \texttt{BST} for \emph{Tale of Two cities}.
	\end{ex}
	\begin{sol}
		In \textbf{Exercise 3.1.21}, I determined that the memory usage for \texttt{SequentialSearchST}
		for $n$ key-value pairs would be $32+48n$ bytes plus the cost of the keys and values
		themselves. I also determined that the cost of \texttt{BinarySearchST} is between $72+16n$
		and $72+64n$ bytes, plus the cost for the keys and values themselves.
		
		For \texttt{BST}, first we consider the cost of each \texttt{Node}. This consists
		of 16 bytes of overhead, 8 bytes overhead for the reference to the enclosing class, 8
		bytes for the reference to the \texttt{key} field, 8 bytes for the reference
		to the \texttt{val} field, \texttt{16} bytes total to account for the references
		to \texttt{left} and \texttt{right}, \texttt{4} bytes for the reference to
		\texttt{n}, and 4 bytes of padding. The total memory cost for a \texttt{Node} is
		thus 64 bytes. The \texttt{BST} class has 16 bytes of overhead, 8 bytes for the
		reference to \texttt{root}, and $64n$ bytes for all $n$ nodes (key-value pairs),
		making the memory cost $24+64n$ bytes
		plus the cost of the $n$ keys and values themselves.
		
		An \texttt{Integer} takes up \texttt{24} bytes, so $n$ integer values amounts to
		$24n$ bytes. The cost of a \texttt{String} according to Section 1.4 on page
		202 of \cite{sedgewick_wayne} is $56+2m$ where $m$ is the total number of characters in
		the \texttt{String} object. If $M$ is the length of the largest \texttt{String}
		key, then the cost is bounded by
		\[
		24+64n+24n+n\cdot (56+2M)=24+144n+2Mn
		\]
		
		The longest word in \emph{Tale of Two Cities} is \texttt{farmergeneralhowsoever},
		which has 22 characters, and the shortest ``word" has 1 character. There are
		$10,674$ distinct words in the text. Therefore, letting $n=10674$ and $M=22$,
		we see that the memory cost $T$ is bounded by
		\begin{align*}
			24+144\cdot 10674+2\cdot 1\cdot 10674&\leq T\leq 24+144\cdot 10674+2\cdot 22\cdot 10674\\
			1558428& \leq T\leq 2006736
		\end{align*}
		This means it takes up at least 1.486 MiB and at most 1.913 MiB.
	\end{sol}
	\begin{ex}{30}
		\emph{BST reconstruction}. Given the preorder (or postorder) traversal of a BST
		(not including null nodes), design an algorithm to reconstruct the BST.
	\end{ex}
	\begin{sol}
		For preorder, we add the keys in the same order; for postorder, we can add
		the keys to a stack and traverse the postorder in reverse.
		
		See \texttt{com.segarciat.algs4.ch3.sec2.ex30.BSTReconstruction}.
	\end{sol}
	\begin{ex}{32}
		\emph{Subtree count check}. Write a recursive method that takes a \texttt{Node}
		as argument and returns \texttt{true} if the subtree count field \texttt{n} is
		consistent in the data structure rooted at that node, \texttt{false} otherwise.
	\end{ex}
	\begin{sol}
		See \texttt{com.segarciat.algs4.ch3.sec2.ex32.BST}.
	\end{sol}
	\begin{ex}{33}
		\emph{Select/rank check}. Write a method that checks, for all \texttt{i} from \texttt{0}
		to \texttt{size() - 1}, whether \texttt{i} is equal to \texttt{rank(select(i))} and,
		for all keys in the BST, whether \texttt{key} is equal to \texttt{select(rank(key))}.
	\end{ex}
	\begin{sol}
		See \texttt{com.segarciat.algs4.ch3.sec2.ex33.SelectRankCheck}.
	\end{sol}
	\begin{ex}{37}
		\emph{Level-order traversal}. Write a method \texttt{printLevel()} that takes
		a \texttt{Node} as argument and prints the keys in the subtree rooted at that
		node in level order (in order of their distance from the root, with nodes
		on each level from left to right). \emph{Hint}: Use a \texttt{Queue}.
	\end{ex}
	\begin{sol}
		See \texttt{com.segarciat.algs4.ch3.sec2.ex37.BST}.
	\end{sol}
	\begin{ex}{38}
		\emph{Tree drawing}. Add a method \texttt{draw()} to \texttt{BST} that draws
		BST figures in the style of the text. \emph{Hint}: Use instance variables to hold
		node coordinates, and use a recursive method to set the values of these variables.
	\end{ex}
	\begin{sol}
		The $x$ coordinate corresponds to the rank, since we can arrange the keys of
		a tree from left to right according to their rank. The $y$ coordinate corresponds
		to the depth of a node.
		
		See \texttt{com.segarciat.algs4.ch3.sec2.ex38.BST}.
	\end{sol}
	\pagebreak
	\printbibliography
\end{document}