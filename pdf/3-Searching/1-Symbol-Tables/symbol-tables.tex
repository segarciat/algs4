\documentclass[12pt, a4paper]{article}

\usepackage[utf8]{inputenc}
% Limit the page margin to only 1 inch.
\usepackage[margin=1in]{geometry}

%Imports biblatex package
\usepackage[
backend=biber,
style=alphabetic
]{biblatex}
\addbibresource{../../algs4e.bib}

% Enables the `align' environment.
\usepackage{amsmath}
% Provides useful environments, such as:
% - \begin{proof} ...\end{proof}
\usepackage{amsthm}
\usepackage[most]{tcolorbox}

\newtheorem*{proposition}{Proposition}

% Enables using \mathbb{}, for example \mathbb{N} for the set of natural numbers.
\usepackage{amssymb}

% Allows using letters in enumerate list environment. Use, for example:
%\begin{enumerate}[label=(\alph*)]
% ...
%\end{enumerate}
\usepackage[inline]{enumitem}

% Enable importing external graphic files and provides useful commannds, like \graphicspath{}
\usepackage{graphicx}
% Images are located in a directory called images in the current directory.
\graphicspath{{./images/}}

% Make links look better by default.
% See: https://tex.stackexchange.com/questions/823/remove-ugly-borders-around-clickable-cross-references-and-hyperlinks
\usepackage[hidelinks]{hyperref}
\usepackage{xcolor}
\hypersetup{
	colorlinks,
	linkcolor={red!50!black},
	citecolor={blue!50!black},
	urlcolor={blue!80!black}
}


% Code Listings. Source:
% https://stackoverflow.com/questions/3175105/inserting-code-in-this-latex-document-with-indentation
\usepackage{listings}
\usepackage{color}

\definecolor{dkgreen}{rgb}{0,0.6,0}
\definecolor{gray}{rgb}{0.5,0.5,0.5}
\definecolor{mauve}{rgb}{0.58,0,0.82}

\lstset{frame=tb,
	language=Java,
	aboveskip=3mm,
	belowskip=3mm,
	showstringspaces=false,
	columns=flexible,
	basicstyle={\small\ttfamily},
	numbers=none,
	numberstyle=\tiny\color{gray},
	keywordstyle=\color{blue},
	commentstyle=\color{dkgreen},
	stringstyle=\color{mauve},
	breaklines=true,
	breakatwhitespace=true,
	tabsize=3
}

\newcommand{\prob}{\text{P}}
%\newcommand{\complement}{\mathsf{c}}

% Define an environment called "ex" (for Exercise) so that I can do: \begin{ex}{1.5}...\end{ex}
\newenvironment{ex}[2][Exercise]
{\par\medskip\noindent \textbf{#1 #2.}}
{\medskip}

% Define a solution environment, similar to ex (exercise) environment.
\newenvironment{sol}[1][Solution]
{\par\medskip\noindent \textbf{#1.} }
{\medskip}

\begin{document}
	\noindent Sergio E. Garcia Tapia \hfill
	
	\noindent \emph{Algorithms} by Sedgewick and Wayne (4th edition) \cite{sedgewick_wayne}\hfill
	
	\noindent November 10th, 2024\hfill 
	\section*{3.1: Symbol Tables}
	\begin{ex}{1}
		Write a client that creates a symbol table mapping letter grades to numerical
		scores, as in the table below, then reads from standard input a list of letter grades
		and computes and prints the GPA (the average of the numbers corresponding to the
		grades).
		
		\begin{center}
			\begin{tabular}{c|c|c|c|c|c|c|c|c|c|c}
				A+ & A & A- & B+ & B & B- & C+ & C & C- & D & F\\
				\hline
				4.33 & 4.00 & 3.67 & 3.33 & 3.00 & 2.67 & 2.33 & 2.00 & 1.67 & 1.00 & 0.00
			\end{tabular}
		\end{center}
	\end{ex}
	\begin{sol}
		See \texttt{com.segarciat.algs4.ch3.sec1.ex01.GPA}
	\end{sol}
	\begin{ex}{2}
		Develop a symbol-table implementation \texttt{ArrayST} that uses an (unordered)
		array as the underlying data structure to implement our basic symbol-table API.
	\end{ex}
	\begin{sol}
		See \texttt{com.segarciat.algs4.ch3.sec1.ex02.ArrayST}.
	\end{sol}
	\begin{ex}{3}
		Develop a symbol-table implementation of \texttt{OrderedSequentialSearchST} that
		uses an ordered linked list as the underlying data structure to implement our
		ordered symbol-table API.
	\end{ex}
	\begin{sol}
		See \texttt{com.segarciat.algs4.ch3.sec1.ex03.OrderedSequentialSearchST}.
	\end{sol}
	\begin{ex}{4}
		Develop \texttt{Time} and \texttt{Event} ADTs that allow processing of data as in
		the example illustrated on page 367.
	\end{ex}
	\begin{sol}
		See \texttt{com.segarciat.algs4.ch3.sec1.ex04.Time}. The class is immutable, and it
		it implements \texttt{Comparable<Time>}, so that it has a natural order.
		I was unclear about what an \texttt{Event} ADT would include, so I did not provide
		an implementation for this ADT.
	\end{sol}
	\begin{ex}{5}
		Implement \texttt{size()}, \texttt{delete()}, and \texttt{keys()} for
		\texttt{SequentialSearchST}.
	\end{ex}
	\begin{sol}
		See \texttt{com.segarciat.algs4.ch3.sec1.ex05.SequentialSearchST}.
	\end{sol}
	\begin{ex}{6}
		Give the number of calls to \texttt{put()} and \texttt{get()} issued by
		\texttt{FrequencyCounter}, as a function of the number $W$ of words and
		the number $D$ of distinct words in the input.
	\end{ex}
	\begin{sol}
		The following assumes that the minimum length accepted for a word is $1$,.
		
		During the first phase, the program builds the symbol tables by processing
		all $W$ words. For each word, there is a call to \texttt{contains()},
		for a total of $W$ such calls. Since a call to \texttt{put()} is made
		regardless of the result, there are $W$ calls to \texttt{put()} during
		this phase. Each result of \texttt{false} from the call to \texttt{contains()}
		corresponds to a distinct word, so there are $D$ such outcomes. Thus,
		there are $W-D$ direct calls to \texttt{get()} in the branch of the
		\texttt{if-else}, where \texttt{get()} is used to retrieve the count
		of a previously-seen word. Note also that each call to \texttt{contains()}
		leads to a call to \texttt{get()}, accounting for $W$ more calls.
		
		In the second phase, one addition call to \texttt{put()} is made, which enters the
		empty string, so that there are now $D+1$ keys in the symbol table.
		In the loop, $2$ calls to \texttt{get()} are made in each iteration, for
		a total of $2(D+1)$ calls. A final call to \texttt{get()} is made after the
		loop.
		
		Thus, if $f$ is the number of calls made to \texttt{put()}, and $g$
		is the number of calls made to \texttt{get()}, then
		\begin{align*}
			f(W, D) &= W + 1\\
			g(W, D) &= W + (W - D) + 2\dot (D+1)+1\\
			&=2W+D+3
		\end{align*}
	\end{sol}
		\begin{ex}{7}
		What is the average number of distinct keys that \texttt{FrequencyCounter} will
		find among $N$ random nonnegative integers less than 1,000, for
		$N=10,10^2,10^3,10^4,10^5$, and $10^6$?
	\end{ex}
	\begin{sol}
		Consider the random experiment of picking $N$ integers at random, where
		each integer is between 0 and 999, and is chosen independently of the other.
		Then each outcome is an $N$-tuple, where each component is an integer between 1 and 1,000.
		Let $X$ be a random variable that counts the number of distinct keys in an $N$-tuple.
		Then $X$ is a discrete random variables, whose values range from $1$ through
		$\min\{N, 1000\}$.
		
		Consider the number of outcomes with $X=k$ distinct integers. If $k$
		is not an integer, or $k >\min\{N, 1000\}$, or $k \leq 0$, then there are
		$0$ such outcomes. Otherwise, we can count in two steps:
		\begin{enumerate}[label=(\roman*)]
			\item Choose $k$ distinct integers. There are $\binom{1000}{k}$ ways
			of doing this, for $1\leq k\leq 1000$, and $0$ for $k>1000$.
			\item Having chosen the $k$ distinct keys, there are $\binom{N}{k}\cdot k!$
			possible positions for them.
			\item To ensure $k$ distinct integers, each of the remaining integers
			must be one of the $k$ integers we have seen before. Since there are $N-k$
			positions to fill, and $k$ integers to choose from, there are $k^{N-k}$ ways
			to do this (we repeat an experiment of choosing among $k$ values a total
			of $N-k$ times).
		\end{enumerate}
		By the multiplication principle of counting, we find that there are
		$\binom{1000}{k}\cdot \binom{N}{k}\cdot  k^{N-K}$ ways to choose $k$ distinct integers
		when choosing a total of $N$ integers. There are a total of $1000^N$ outcomes.
		Since every outcome is equally likely, the probability of an outcome having
		$k$ distinct keys is therefore equal to:
		\begin{align*}
			P(\{X=k\})=\frac{\binom{1000}{k}\cdot \binom{N}{k}\cdot k!\cdot  k^{N-k}}{1000^N},
			\quad 1\leq k\leq \min\{1000, N\},\quad N \text{ fixed}.
		\end{align*}
		Unfortunately, writing a closed form for this is difficult, and so is evaluating it
		as-is, and it's hard to verify its correctness. My goal was then to compute
		the expectation as:
		\begin{align*}
			E[X] = \sum_{k=1}^{\min\{N, 1000\}}P(\{X=k\})
		\end{align*}
		I found an alternative approach in this \href{https://math.stackexchange.com/a/2770036}{Stack Overflow answer by qwr}.
		The idea is to let $A=\{0,\ldots,999\}$, the set of numbers that we choose from
		(sample) at random, and the let $X_j$ be an indicator random variable.
		That is, if we sample $N$ elements from $A$, then $X_j=1$ if $j$ is in the
		sample, and $0$ otherwise, where $j\in A$.
		
		The idea is that we have a collection of random variables $X_0,\ldots,X_{999}$,
		and since we sample at random. Then, by defining
		$X=\sum_{j=0}^{999}X_j$, we obtain a random variable $X$ that gives
		the number of distinct keys in the sample. Then, we use the fact that
		the expectation is linear (regardless of independence), meaning:
		\begin{align*}
			E[X] = E\left[\sum_{j=0}^{999}X_j\right]=\sum_{j=0}^{999}E[X_j]
		\end{align*}
		Therefore, this reduces the problem of finding the expectation of $X$
		(the average number of distinct values in a sample) to finding the
		expectation of the indicator random variables. The latter turns out
		to be simple, and user \texttt{dwr} argues as follows. Consider
		the number of ways to choose a sample of $N$ items without $j$ in it.
		There are $1000-1$ other numbers, and $N$ numbers to choose (with replacement),
		so this amounts to $(1000-1)^N$ choices. Meanwhile, there's $1000^N$ ways to
		choose $N$ integers from the sample.
		
		Thus,
		\begin{align*}
			\prob(\{X_j=0\})=\frac{999^N}{1000^N}
		\end{align*}
		which means that
		\begin{align*}
			\prob(\{X_j=1\})=1-\left(\frac{999}{1000}\right)^N
		\end{align*}
		Now the expectation of $X_j$ is simple:
		\begin{align*}
			E[X_j]=0\cdot P(\{X_j=0\}) + 1\cdot P(\{X_j=1\})=1-\left(\frac{999}{1000}\right)^N
		\end{align*}
		Hence, by linearity of expectation:
		\begin{align*}
			E[X]=1000\cdot \left(1-\left(\frac{999}{1000}\right)^N\right)
		\end{align*}
		Now, plugging in the different $N$ values:
		\begin{center}
			\begin{tabular}{c|cccccc}
				$N$ & 10 & $10^2$ & $10^3$ & $10^4$ & $10^5$ & $10^6$ \\
				\hline
				Average & 9.96 & 95.21 & 632.30 & 999.95 & 1000.00 & 1000.00
			\end{tabular}
		\end{center}
		See \texttt{com.segarciat.algs4.ch3.sec1.ex07.AverageDistinct}, which
		verifies these results.
	\end{sol}
	\pagebreak
	\printbibliography
\end{document}