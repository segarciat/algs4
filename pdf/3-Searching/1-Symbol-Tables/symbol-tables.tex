\documentclass[12pt, a4paper]{article}

\usepackage[utf8]{inputenc}
% Limit the page margin to only 1 inch.
\usepackage[margin=1in]{geometry}

%Imports biblatex package
\usepackage[
backend=biber,
style=alphabetic
]{biblatex}
\addbibresource{../../algs4e.bib}

% Enables the `align' environment.
\usepackage{amsmath}
% Provides useful environments, such as:
% - \begin{proof} ...\end{proof}
\usepackage{amsthm}
\usepackage[most]{tcolorbox}

\newtheorem*{proposition}{Proposition}

% Enables using \mathbb{}, for example \mathbb{N} for the set of natural numbers.
\usepackage{amssymb}

% Allows using letters in enumerate list environment. Use, for example:
%\begin{enumerate}[label=(\alph*)]
% ...
%\end{enumerate}
\usepackage[inline]{enumitem}

% Enable importing external graphic files and provides useful commannds, like \graphicspath{}
\usepackage{graphicx}
% Images are located in a directory called images in the current directory.
\graphicspath{{./images/}}

% Make links look better by default.
% See: https://tex.stackexchange.com/questions/823/remove-ugly-borders-around-clickable-cross-references-and-hyperlinks
\usepackage[hidelinks]{hyperref}
\usepackage{xcolor}
\hypersetup{
	colorlinks,
	linkcolor={red!50!black},
	citecolor={blue!50!black},
	urlcolor={blue!80!black}
}


% Code Listings. Source:
% https://stackoverflow.com/questions/3175105/inserting-code-in-this-latex-document-with-indentation
\usepackage{listings}
\usepackage{color}

\definecolor{dkgreen}{rgb}{0,0.6,0}
\definecolor{gray}{rgb}{0.5,0.5,0.5}
\definecolor{mauve}{rgb}{0.58,0,0.82}

\lstset{frame=tb,
	language=Java,
	aboveskip=3mm,
	belowskip=3mm,
	showstringspaces=false,
	columns=flexible,
	basicstyle={\small\ttfamily},
	numbers=none,
	numberstyle=\tiny\color{gray},
	keywordstyle=\color{blue},
	commentstyle=\color{dkgreen},
	stringstyle=\color{mauve},
	breaklines=true,
	breakatwhitespace=true,
	tabsize=3
}

\newcommand{\prob}{\text{P}}
%\newcommand{\complement}{\mathsf{c}}

% Define an environment called "ex" (for Exercise) so that I can do: \begin{ex}{1.5}...\end{ex}
\newenvironment{ex}[2][Exercise]
{\par\medskip\noindent \textbf{#1 #2.}}
{\medskip}

% Define a solution environment, similar to ex (exercise) environment.
\newenvironment{sol}[1][Solution]
{\par\medskip\noindent \textbf{#1.} }
{\medskip}

\begin{document}
	\noindent Sergio E. Garcia Tapia \hfill
	
	\noindent \emph{Algorithms} by Sedgewick and Wayne (4th edition) \cite{sedgewick_wayne}\hfill
	
	\noindent November 10th, 2024\hfill 
	\section*{3.1: Symbol Tables}
	\begin{ex}{1}
		Write a client that creates a symbol table mapping letter grades to numerical
		scores, as in the table below, then reads from standard input a list of letter grades
		and computes and prints the GPA (the average of the numbers corresponding to the
		grades).
		
		\begin{center}
			\begin{tabular}{c|c|c|c|c|c|c|c|c|c|c}
				A+ & A & A- & B+ & B & B- & C+ & C & C- & D & F\\
				\hline
				4.33 & 4.00 & 3.67 & 3.33 & 3.00 & 2.67 & 2.33 & 2.00 & 1.67 & 1.00 & 0.00
			\end{tabular}
		\end{center}
	\end{ex}
	\begin{sol}
		See \texttt{com.segarciat.algs4.ch3.sec1.ex01.GPA}
	\end{sol}
	\begin{ex}{2}
		Develop a symbol-table implementation \texttt{ArrayST} that uses an (unordered)
		array as the underlying data structure to implement our basic symbol-table API.
	\end{ex}
	\begin{sol}
		See \texttt{com.segarciat.algs4.ch3.sec1.ex02.ArrayST}.
	\end{sol}
	\begin{ex}{3}
		Develop a symbol-table implementation of \texttt{OrderedSequentialSearchST} that
		uses an ordered linked list as the underlying data structure to implement our
		ordered symbol-table API.
	\end{ex}
	\begin{sol}
		See \texttt{com.segarciat.algs4.ch3.sec1.ex03.OrderedSequentialSearchST}.
	\end{sol}
	\begin{ex}{4}
		Develop \texttt{Time} and \texttt{Event} ADTs that allow processing of data as in
		the example illustrated on page 367.
	\end{ex}
	\begin{sol}
		See \texttt{com.segarciat.algs4.ch3.sec1.ex04.Time}. The class is immutable, and it
		it implements \texttt{Comparable<Time>}, so that it has a natural order.
		I was unclear about what an \texttt{Event} ADT would include, so I did not provide
		an implementation for this ADT.
	\end{sol}
	\begin{ex}{5}
		Implement \texttt{size()}, \texttt{delete()}, and \texttt{keys()} for
		\texttt{SequentialSearchST}.
	\end{ex}
	\begin{sol}
		See \texttt{com.segarciat.algs4.ch3.sec1.ex05.SequentialSearchST}.
	\end{sol}
	\begin{ex}{6}
		Give the number of calls to \texttt{put()} and \texttt{get()} issued by
		\texttt{FrequencyCounter}, as a function of the number $W$ of words and
		the number $D$ of distinct words in the input.
	\end{ex}
	\begin{sol}
		The following assumes that the minimum length accepted for a word is $1$,.
		
		During the first phase, the program builds the symbol tables by processing
		all $W$ words. For each word, there is a call to \texttt{contains()},
		for a total of $W$ such calls. Since a call to \texttt{put()} is made
		regardless of the result, there are $W$ calls to \texttt{put()} during
		this phase. Each result of \texttt{false} from the call to \texttt{contains()}
		corresponds to a distinct word, so there are $D$ such outcomes. Thus,
		there are $W-D$ direct calls to \texttt{get()} in the branch of the
		\texttt{if-else}, where \texttt{get()} is used to retrieve the count
		of a previously-seen word. Note also that each call to \texttt{contains()}
		leads to a call to \texttt{get()}, accounting for $W$ more calls.
		
		In the second phase, one addition call to \texttt{put()} is made, which enters the
		empty string, so that there are now $D+1$ keys in the symbol table.
		In the loop, $2$ calls to \texttt{get()} are made in each iteration, for
		a total of $2(D+1)$ calls. A final call to \texttt{get()} is made after the
		loop.
		
		Thus, if $f$ is the number of calls made to \texttt{put()}, and $g$
		is the number of calls made to \texttt{get()}, then
		\begin{align*}
			f(W, D) &= W + 1\\
			g(W, D) &= W + (W - D) + 2\dot (D+1)+1\\
			&=2W+D+3
		\end{align*}
	\end{sol}
		\begin{ex}{7}
		What is the average number of distinct keys that \texttt{FrequencyCounter} will
		find among $N$ random nonnegative integers less than 1,000, for
		$N=10,10^2,10^3,10^4,10^5$, and $10^6$?
	\end{ex}
	\begin{sol}
		Consider the random experiment of picking $N$ integers at random, where
		each integer is between 0 and 999, and is chosen independently of the other.
		Then each outcome is an $N$-tuple, where each component is an integer between 1 and 1,000.
		Let $X$ be a random variable that counts the number of distinct keys in an $N$-tuple.
		Then $X$ is a discrete random variables, whose values range from $1$ through
		$\min\{N, 1000\}$.
		
		Consider the number of outcomes with $X=k$ distinct integers. If $k$
		is not an integer, or $k >\min\{N, 1000\}$, or $k \leq 0$, then there are
		$0$ such outcomes. Otherwise, we can count in two steps:
		\begin{enumerate}[label=(\roman*)]
			\item Choose $k$ distinct integers. There are $\binom{1000}{k}$ ways
			of doing this, for $1\leq k\leq 1000$, and $0$ for $k>1000$.
			\item Having chosen the $k$ distinct keys, there are $\binom{N}{k}\cdot k!$
			possible positions for them.
			\item To ensure $k$ distinct integers, each of the remaining integers
			must be one of the $k$ integers we have seen before. Since there are $N-k$
			positions to fill, and $k$ integers to choose from, there are $k^{N-k}$ ways
			to do this (we repeat an experiment of choosing among $k$ values a total
			of $N-k$ times).
		\end{enumerate}
		By the multiplication principle of counting, we find that there are
		$\binom{1000}{k}\cdot \binom{N}{k}\cdot  k^{N-K}$ ways to choose $k$ distinct integers
		when choosing a total of $N$ integers. There are a total of $1000^N$ outcomes.
		Since every outcome is equally likely, the probability of an outcome having
		$k$ distinct keys is therefore equal to:
		\begin{align*}
			P(\{X=k\})=\frac{\binom{1000}{k}\cdot \binom{N}{k}\cdot k!\cdot  k^{N-k}}{1000^N},
			\quad 1\leq k\leq \min\{1000, N\},\quad N \text{ fixed}.
		\end{align*}
		Unfortunately, writing a closed form for this is difficult, and so is evaluating it
		as-is, and it's hard to verify its correctness. My goal was then to compute
		the expectation as:
		\begin{align*}
			E[X] = \sum_{k=1}^{\min\{N, 1000\}}P(\{X=k\})
		\end{align*}
		I found an alternative approach in this \href{https://math.stackexchange.com/a/2770036}{Stack Overflow answer by qwr}.
		The idea is to let $A=\{0,\ldots,999\}$, the set of numbers that we choose from
		(sample) at random, and the let $X_j$ be an indicator random variable.
		That is, if we sample $N$ elements from $A$, then $X_j=1$ if $j$ is in the
		sample, and $0$ otherwise, where $j\in A$.
		
		The idea is that we have a collection of random variables $X_0,\ldots,X_{999}$,
		and since we sample at random. Then, by defining
		$X=\sum_{j=0}^{999}X_j$, we obtain a random variable $X$ that gives
		the number of distinct keys in the sample. Then, we use the fact that
		the expectation is linear (regardless of independence), meaning:
		\begin{align*}
			E[X] = E\left[\sum_{j=0}^{999}X_j\right]=\sum_{j=0}^{999}E[X_j]
		\end{align*}
		Therefore, this reduces the problem of finding the expectation of $X$
		(the average number of distinct values in a sample) to finding the
		expectation of the indicator random variables. The latter turns out
		to be simple, and user \texttt{dwr} argues as follows. Consider
		the number of ways to choose a sample of $N$ items without $j$ in it.
		There are $1000-1$ other numbers, and $N$ numbers to choose (with replacement),
		so this amounts to $(1000-1)^N$ choices. Meanwhile, there's $1000^N$ ways to
		choose $N$ integers from the sample.
		
		Thus,
		\begin{align*}
			\prob(\{X_j=0\})=\frac{999^N}{1000^N}
		\end{align*}
		which means that
		\begin{align*}
			\prob(\{X_j=1\})=1-\left(\frac{999}{1000}\right)^N
		\end{align*}
		Now the expectation of $X_j$ is simple:
		\begin{align*}
			E[X_j]=0\cdot P(\{X_j=0\}) + 1\cdot P(\{X_j=1\})=1-\left(\frac{999}{1000}\right)^N
		\end{align*}
		Hence, by linearity of expectation:
		\begin{align*}
			E[X]=1000\cdot \left(1-\left(\frac{999}{1000}\right)^N\right)
		\end{align*}
		Now, plugging in the different $N$ values:
		\begin{center}
			\begin{tabular}{c|cccccc}
				$N$ & 10 & $10^2$ & $10^3$ & $10^4$ & $10^5$ & $10^6$ \\
				\hline
				Average & 9.96 & 95.21 & 632.30 & 999.95 & 1000.00 & 1000.00
			\end{tabular}
		\end{center}
		See \texttt{com.segarciat.algs4.ch3.sec1.ex07.AverageDistinct}, which
		verifies these results.
	\end{sol}
	\begin{ex}{8}
		What is the most frequency used word of ten letters or more in \emph{Tale of Two Cities}?
	\end{ex}
	\begin{sol}
		It is \texttt{monseigneur}, which I found out by running the program
		directly. See \texttt{com.segarciat.algs4.ch3.sec1.ex08}.
	\end{sol}
	\begin{ex}{9}
		Add code to \texttt{FrequencyCounter} to keep track of the \emph{last} call to
		\texttt{put()}. Print the last word inserted and the number of words that
		were processed in the input stream prior to this intersection. Run your program
		for \texttt{tale.txt} with length cutoffs 1, 8, and 10.
	\end{ex}
	\begin{sol}
		See \texttt{com.segarciat.algs4.ch3.sec1.ex09.FrequencyCounter}.
	\end{sol}
	\begin{ex}{12}
		Modify \texttt{BinarySearchST} to maintain one array of \texttt{Item} objects that
		conntain keys and values, rather than two parallel arrays. Add a constructor that
		takes an array of \texttt{Item} values as argument and uses mergesort to sort the array.
	\end{ex}
	\begin{sol}
		See \texttt{com.segarciat.algs4.ch3.sec1.ex12.BinarySearchST}.
	\end{sol}
	\begin{ex}{13}
		Which of the symbol-table implementations in this section would you use for an
		application that does $10^3$ \texttt{put()} operations and $10^6$ \texttt{get()}
		operations, randomly intermixed? Justify your answer.
	\end{ex}
	\begin{sol}
		I would use a \texttt{BinarySearchST} because the relative number of search
		operations is far greater. It's true that they are intermixed, which means
		the cost of an insert (\texttt{put()}) is linear in the current length
		off the symbol-table. However, if we were to use a \texttt{SequentialSearchST},
		both operations are have a linear time complexity on the average. Meanwhile,
		logarithmic performance of the search operation for \texttt{BinarySearchST}
		gives us an edge since we have such a high proportion of searches.
	\end{sol}
	\begin{ex}{14}
		Which of the symbol-table implementations would you use for an application that
		does $10^6$ \texttt{put()} operations and $10^3$ \texttt{get()} operations, randomly
		intermixed? Justify your answer.
	\end{ex}
	\begin{sol}
		Although inserts are slow for \texttt{BinarySearchST}, I would still opt for
		them in this scenario because \texttt{SequentialSearchST} perform poorly
		on large symbol-tables. For example, to find out that a key is not in the 
		linked list, we must search the entire linked list.
	\end{sol}
	\begin{ex}{16}
		Implement the \texttt{delete()} method for \texttt{BinarySearchST}.
	\end{ex}
	\begin{sol}
		See \texttt{com.segarciat.algs4.ch3.sec1.ex16.BinarySearchST}.
	\end{sol}
	\begin{ex}{17}
		Implement the \texttt{floor()} method for \texttt{BinarySearchST}.
	\end{ex}
	\begin{sol}
		See \texttt{com.segarciat.algs4.ch3.sec1.ex17.BinarySearchST}.
	\end{sol}
	\begin{ex}{18}
		Prove that the \texttt{rank()} method in \texttt{BinarySearchST} is correct.
	\end{ex}
	\begin{sol}
		\begin{proof}
			Suppose that \texttt{key} is not \texttt{null}, and that the array we are
			searching for \texttt{key} is ordered with all keys distinct.
			
			If after computing \texttt{mid}, the index of the middle of key, we find that the array
			at that index matches \texttt{key}, then the algorithms stops. Since the array
			is ordered an all keys are unique, any item at an index larger than \texttt{mid}
			cannot be less than \texttt{key}, and all keys at an index less than
			\texttt{mid} compare less than \texttt{key}. Therefore, the number of keys in the
			array that are less than the key at index \texttt{mid} is precisely \texttt{mid},
			because arrays are 0-indexed.
			
			Suppose that after comparison we find that \texttt{cmp < 0}. This means that
			\texttt{keys[mid]} is larger than \texttt{key}. In reducing the search space to \texttt{lo..mid-1}, the value of \texttt{lo} (which is its known rank so far) does not change, consistent with the fact that all keys at an index higher than or equal
			to \texttt{mid} are strictly larger than \texttt{key}. That is, they cannot contribute
			ot the rank.
			
			Suppose that after comparison we find that \texttt{cmp > 0}. That is,
			\texttt{key} compares larger than \texttt{keys[mid]}. By the transitivity
			of the total ordering of \texttt{compareTo()}, this means \texttt{key}
			is larger than the keys in the index range \texttt{0..mid}, which constitutes
			a total of \texttt{mid + 1} keys. This is consistent with the update of \texttt{lo}
			to \texttt{mid + 1} in this range.
			
			After every iteration, we maintain the invariant that \texttt{key} is greater
			than all of the keys in the index range \texttt{0..lo-1}, a total of \texttt{lo}
			keys. The search eventually ends because the length of the interval reduces by
			half each time, and when the length is 0, it reduces by 1, so that eventually
			the condition \texttt{lo <= hi} becomes \texttt{false}. With the invariant
			maintained after each iteration, we are certain that \texttt{lo} indeed represents
			the number of keys less than \texttt{key}, its rank.
		\end{proof}
	\end{sol}
	\begin{ex}{19}
		Modify \texttt{FrequencyCounter} to print all of the values having the highest frequency
		of occurrence, not just one of them. \emph{Hint}: Use a \texttt{Queue}.
	\end{ex}
	\begin{sol}
		See \texttt{com.segarciat.algs4.ch3.sec1.ex19.FrequencyCounter}. Each time a key with
		a larger frequency is encounter, the queue is cleared, and the whenever a key
		with the same frequency as the current largest frequency is encountered, we add
		to the queue.
	\end{sol}
	\begin{ex}{20}
		Complete the proof of \textbf{Proposition B} (show that it holds for all values
		of $n$). \emph{Hint}: Start by showing that $C(n)$ is monotonic: $C(n)\leq C(n+1)$
		for all $n\geq 0$.
	\end{ex}
	\begin{sol}
		\begin{proof}
			We begin by showing that $C(n)$ is monotone by strong induction. That is, we first
			show that $C(n+1)\geq C(n)$ for all non-negative integers $n$. The base
			case is certainly true since $C(0)=0$ and $C(1)=1$, so $C(1)\geq C(0)$.
			For the inductive hypothesis, suppose that $n\geq 1$, and that $C(k+1)\geq C(k)$
			for all $0\leq k< n$.
			
			\begin{enumerate}[label=(\roman*)]
				\item Suppose that $n$ is even, meaning that $n=2 k$ for a non-negative
				integer $k$. The proof of \textbf{Proposition B} established that
				\begin{align*}
					C(n)=C(2\cdot k)\leq C(\lfloor 2k / 2\rfloor) + 1 = C(k) + 1
				\end{align*}
				by arguing that the middle element is examine (justifying the $+1$)
				and that that search must continue in either half.
				Since $n$ is even, we know that $n+1$ is odd, and the algorithm divides
				the array into two pieces of size $k$ after examining the middle entry.
				This implies that the above inequality becomes an inequality in this
				case:
				\begin{align*}
					C(n + 1) = C(\lfloor \frac{2k+1}{2}\rfloor) + 1 = C(k)+1,\quad \text{n is even}.
				\end{align*}
				Hence we see that $C(n)\leq C(k)+1=C(n+1)$ when $n$ is even.
				\item Now if $n$ is odd, meaning $n=2k+1$, where for some integer $k\in\mathbb{N}$,
				once again we have
				\begin{align*}
					C(n)=C(k) + 1
				\end{align*}
				Then $n+1=2k+2$ is even. After examining the middle entry, the search may go left
				into the array of size $k$, requiring $C(k)$ comparisons, or will may go right
				into the array size $k+1$ and will require $C(k+1)$ comparisons. Since $k<n$,
				we know by the inductive hypothesis that $C(k)\leq C(k+1)$, so regardless
				of which piece it descends into, we have $C(n+1)\leq C(k)+1=C(n)$, when
				$n$ is odd.
			\end{enumerate}
			We conclude by mathematical induction on $n$ that $C(n)$ is monotone.
			Now let $n>0$, and let $m$ be the largest power of $m$ such that
			$2^m\leq n\leq  2^{m+1} - 1$. Then
			\begin{align*}
				C(n)\leq C(2^{m+1}-1)\leq m+1\leq \lfloor \lg n\rfloor +1 \leq \lg n + 1
			\end{align*}
		\end{proof}
	\end{sol}
	\begin{ex}{21}
		\emph{Memory usage}. Compare the memory usage of \texttt{BinarySearchST} with that
		of \texttt{SequentialSearchST} for $n$ key-value pairs, under the assumptions described
		in Section 1.4. Do not count the memory for the keys and values themselves, but
		do count references to them. For \texttt{BinarySearchST}, assume that array resizing
		is used, so that hte array is between 25 percent and 100 percent .
	\end{ex}
	\begin{sol}
		For \texttt{SequentialSearchST}, we have 16 bytes of class overhead, and 8 bytes
		for its reference to the  \texttt{first} field, and 4 bytes for its \texttt{int}
		instance variable \texttt{n} for the size, and 4 bytes of padding, for a total
		of 32 bytes. For each key-value pair there is a \texttt{Node} that takes up 16
		nodes of overhead, 8 bytes of overhead for a reference to the enclosing class,
		8 bytes for the key reference, 8 bytes for the value reference, and 8 bytes for a
		reference to the \texttt{next} \texttt{Node} object, a total of 48 bytes per node.
		Hence, the cost is about $32+48n$ bytes.
		
		For the \texttt{BinarySearchST} implementation with two arrays (one for the keys
		and one of the values), there are 8 bytes for a reference to each of them,
		and 24 bytes of overhead for each of them. There's also 4 bytes for the length
		\texttt{n} of the symbol-table, and 4 bytes
		of padding, which is 72 bytes total. For each key and each value, there are
		8 bytes for the key reference and 8 bytes for the value reference. In the
		best scenario, when the array is 100 percent full (meaning no references
		are \texttt{null}), this amounts of $16n$ bytes. If it's 25 percent full,
		the amount of space is roughly quadrupled even though there are \texttt{null}
		references, for a total of $64n$ bytes. Therefore, this implementation takes
		up between $72+16n$ and $72+64n$ bytes.
	\end{sol}
	\begin{ex}{22}
		\emph{Self-organizing search}. A self-organizing search algorithm is one that
		rearranges items to make those that are accessed frequently likely to be found
		early in the search. Modify your search implementation for \textbf{Exercise 3.1.2}
		to perform the following action on every search hit: move the key-value pair
		found to the beginning of the list, moving all pairs between the beginning of the
		list and the vacated position to the right one position. This procedure is called
		the \emph{move-to-front} heuristic.
	\end{ex}
	\begin{sol}
		See \texttt{com.segarciat.algs4.ch3.sec1.ex22.ArrayST}.
	\end{sol}
	\begin{ex}{23}
		\emph{Analysis of binary search}. Prove that the maximum number of compares used
		for a binary search in a table of size $n$ is precisely the number of bits in
		the binary representation of $n$, because the operation of shifting 1 bit to the
		right converts the binary representation of $n$ into the binary representation
		of $\lfloor n / 2\rfloor$.
	\end{ex}
	\begin{sol}
		\begin{proof}
			Suppose that $n$ is a posive integer. If $n$ has $m$ bits, then we have to
			show that the maximum number of compares is $m$.
			
			During each iteration, the index \texttt{mid} is computed by finding the
			halfway point of the current subarray. At the start, the subarray spans
			the indices \texttt{0..n - 1}, which has length $n$. The index \texttt{mid}
			splits the array into two subarrays \texttt{lo..mid-1} and \texttt{mid+1..hi},
			with the rightmost one having length $\lfloor n / 2\rfloor$, and the leftmost one
			being either the same length or 1 smaller. The algorithm will require
			a maximum number of compares if the search descends into the larger subarray
			after each iteration (or either subarray when their lengths are the same).
			Hence if $k$ is the current subarray length (where $k=n$ at the start), then
			at each step we would choose the subarray of length $\lfloor k / 2 \rfloor$.
			
			A compare is done as long as \texttt{lo <= hi}, meaning that the subarray
			length is at least 1, meaning $k\geq 1$ and hence the binary representation
			of $k$ has at least 1 nonzero bit. Since integer division by $2$ is equivalent
			to a right shift by 1 bit, each iteration shifts the bits in the binary
			representation of $n$. Since the binary representation of $n$ has $m$ bits,
			the algorithm will continue for at most $m$ right shifts (and hence $m$ divisions
			by 2).
		\end{proof}
	\end{sol}
	\begin{ex}{25}
		\emph{Software caching}. Since the default implementation of \texttt{contains()}
		calls \texttt{get()}, the inner loop of \texttt{FrequencyCounter}
		\begin{lstlisting}[language=java]
			if (!st.contains(word))
			    st.put(word, 1);
			else
			    st.put(word, st.get(word) + 1);
		\end{lstlisting}
		leads to two or three searches for the same key. To enable clear code like this
		without sacrificing efficiency, we can use a technique known as \emph{software caching},
		where we save the location of the most recently accessed key in an instance variable.
		Modify \texttt{SequentialSearchST} and \texttt{BinarySearchST} to take advantage of
		this idea.
	\end{ex}
	\begin{sol}
		See \texttt{com.segarciat.algs4.ch3.sec1.ex25}.
	\end{sol}
	\begin{ex}{26}
		\emph{Frequency count from a dictionary}. Modify \texttt{FrequencyCounter} to take
		the name of a dictionary file as its argument, count frequencies of the words from
		standard input that are also in that file, and print two tables of the words with
		their frequencies, one sorted by frequency, the other in the order found in the
		dictionary file.
	\end{ex}
	\begin{sol}
		See \texttt{com.segarciat.algs4.ch3.sec1.ex26.FrequencyCounter}.
	\end{sol}
	\begin{ex}{27}
		\emph{Small tables}. Suppose that a \texttt{BinarySearchST} client has $S$ search
		operations and $n$ distinct keys. Give the order of growth of $S$ such that the cost
		of building the table is the same as the cost of all searches.
	\end{ex}
	\begin{sol}
		By \textbf{Proposition B} in \cite{sedgewick_wayne}, inserting $n$ key-value pairs
		into an initially empty table uses about $2n^2$ array accesses, and binary search
		on an array of $n$ keys takes no more than $\lg n + 1$ compares. Since the order
		of growth of ta search operation corresponds to the order of growth of binary search,
		we can let the cost $C$ of the search operations be given by $C=S\cdot \lg n$,
		and we seek $S$ such that $C(n)\sim 2n^2$. For this to happen, we need
		$S\sim \frac{2n^2}{\lg n}$.
	\end{sol}
	\begin{ex}{28}
		\emph{Ordered insertions}. Modify \texttt{BinarySearchST} so that inserting a key that
		is larger than all keys in the table takes constant time (so that building a table by
		calling \texttt{put()} for keys that are in ordered takes linear time).
	\end{ex}
	\begin{sol}
		See \texttt{com.segarciat.algs4.ch3.sec1.ex28,.BinarySearchST}.
	\end{sol}
	\pagebreak
	\printbibliography
\end{document}