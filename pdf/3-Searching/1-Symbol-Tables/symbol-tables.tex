\documentclass[12pt, a4paper]{article}

\usepackage[utf8]{inputenc}
% Limit the page margin to only 1 inch.
\usepackage[margin=1in]{geometry}

%Imports biblatex package
\usepackage[
backend=biber,
style=alphabetic
]{biblatex}
\addbibresource{../../algs4e.bib}

% Enables the `align' environment.
\usepackage{amsmath}
% Provides useful environments, such as:
% - \begin{proof} ...\end{proof}
\usepackage{amsthm}
\usepackage[most]{tcolorbox}

\newtheorem*{proposition}{Proposition}

% Enables using \mathbb{}, for example \mathbb{N} for the set of natural numbers.
\usepackage{amssymb}

% Allows using letters in enumerate list environment. Use, for example:
%\begin{enumerate}[label=(\alph*)]
% ...
%\end{enumerate}
\usepackage[inline]{enumitem}

% Enable importing external graphic files and provides useful commannds, like \graphicspath{}
\usepackage{graphicx}
% Images are located in a directory called images in the current directory.
\graphicspath{{./images/}}

% Make links look better by default.
% See: https://tex.stackexchange.com/questions/823/remove-ugly-borders-around-clickable-cross-references-and-hyperlinks
\usepackage[hidelinks]{hyperref}
\usepackage{xcolor}
\hypersetup{
	colorlinks,
	linkcolor={red!50!black},
	citecolor={blue!50!black},
	urlcolor={blue!80!black}
}


% Code Listings. Source:
% https://stackoverflow.com/questions/3175105/inserting-code-in-this-latex-document-with-indentation
\usepackage{listings}
\usepackage{color}

\definecolor{dkgreen}{rgb}{0,0.6,0}
\definecolor{gray}{rgb}{0.5,0.5,0.5}
\definecolor{mauve}{rgb}{0.58,0,0.82}

\lstset{frame=tb,
	language=Java,
	aboveskip=3mm,
	belowskip=3mm,
	showstringspaces=false,
	columns=flexible,
	basicstyle={\small\ttfamily},
	numbers=none,
	numberstyle=\tiny\color{gray},
	keywordstyle=\color{blue},
	commentstyle=\color{dkgreen},
	stringstyle=\color{mauve},
	breaklines=true,
	breakatwhitespace=true,
	tabsize=3
}

\newcommand{\prob}{\text{P}}
%\newcommand{\complement}{\mathsf{c}}

% Define an environment called "ex" (for Exercise) so that I can do: \begin{ex}{1.5}...\end{ex}
\newenvironment{ex}[2][Exercise]
{\par\medskip\noindent \textbf{#1 #2.}}
{\medskip}

% Define a solution environment, similar to ex (exercise) environment.
\newenvironment{sol}[1][Solution]
{\par\medskip\noindent \textbf{#1.} }
{\medskip}

\begin{document}
	\noindent Sergio E. Garcia Tapia \hfill
	
	\noindent \emph{Algorithms} by Sedgewick and Wayne (4th edition) \cite{sedgewick_wayne}\hfill
	
	\noindent November 10th, 2024\hfill 
	\section*{3.1: Symbol Tables}
	\begin{ex}{1}
		Write a client that creates a symbol table mapping letter grades to numerical
		scores, as in the table below, then reads from standard input a list of letter grades
		and computes and prints the GPA (the average of the numbers corresponding to the
		grades).
		
		\begin{center}
			\begin{tabular}{c|c|c|c|c|c|c|c|c|c|c}
				A+ & A & A- & B+ & B & B- & C+ & C & C- & D & F\\
				\hline
				4.33 & 4.00 & 3.67 & 3.33 & 3.00 & 2.67 & 2.33 & 2.00 & 1.67 & 1.00 & 0.00
			\end{tabular}
		\end{center}
	\end{ex}
	\begin{sol}
		See \texttt{com.segarciat.algs4.ch3.sec1.ex01.GPA}
	\end{sol}
	\begin{ex}{2}
		Develop a symbol-table implementation \texttt{ArrayST} that uses an (unordered)
		array as the underlying data structure to implement our basic symbol-table API.
	\end{ex}
	\begin{sol}
		See \texttt{com.segarciat.algs4.ch3.sec1.ex02.ArrayST}.
	\end{sol}
	\begin{ex}{3}
		Develop a symbol-table implementation of \texttt{OrderedSequentialSearchST} that
		uses an ordered linked list as the underlying data structure to implement our
		ordered symbol-table API.
	\end{ex}
	\begin{sol}
		See \texttt{com.segarciat.algs4.ch3.sec1.ex03.OrderedSequentialSearchST}.
	\end{sol}
	\begin{ex}{4}
		Develop \texttt{Time} and \texttt{Event} ADTs that allow processing of data as in
		the example illustrated on page 367.
	\end{ex}
	\begin{sol}
		See \texttt{com.segarciat.algs4.ch3.sec1.ex04.Time}. The class is immutable, and it
		it implements \texttt{Comparable<Time>}, so that it has a natural order.
		I was unclear about what an \texttt{Event} ADT would include, so I did not provide
		an implementation for this ADT.
	\end{sol}
	\begin{ex}{5}
		Implement \texttt{size()}, \texttt{delete()}, and \texttt{keys()} for
		\texttt{SequentialSearchST}.
	\end{ex}
	\begin{sol}
		See \texttt{com.segarciat.algs4.ch3.sec1.ex05.SequentialSearchST}.
	\end{sol}
	\begin{ex}{6}
		Give the number of calls to \texttt{put()} and \texttt{get()} issued by
		\texttt{FrequencyCounter}, as a function of the number $W$ of words and
		the number $D$ of distinct words in the input.
	\end{ex}
	\begin{sol}
		The following assumes that the minimum length accepted for a word is $1$,.
		
		During the first phase, the program builds the symbol tables by processing
		all $W$ words. For each word, there is a call to \texttt{contains()},
		for a total of $W$ such calls. Since a call to \texttt{put()} is made
		regardless of the result, there are $W$ calls to \texttt{put()} during
		this phase. Each result of \texttt{false} from the call to \texttt{contains()}
		corresponds to a distinct word, so there are $D$ such outcomes. Thus,
		there are $W-D$ direct calls to \texttt{get()} in the branch of the
		\texttt{if-else}, where \texttt{get()} is used to retrieve the count
		of a previously-seen word. Note also that each call to \texttt{contains()}
		leads to a call to \texttt{get()}, accounting for $W$ more calls.
		
		In the second phase, one addition call to \texttt{put()} is made, which enters the
		empty string, so that there are now $D+1$ keys in the symbol table.
		In the loop, $2$ calls to \texttt{get()} are made in each iteration, for
		a total of $2(D+1)$ calls. A final call to \texttt{get()} is made after the
		loop.
		
		Thus, if $f$ is the number of calls made to \texttt{put()}, and $g$
		is the number of calls made to \texttt{get()}, then
		\begin{align*}
			f(W, D) &= W + 1\\
			g(W, D) &= W + (W - D) + 2\dot (D+1)+1\\
			&=2W+D+3
		\end{align*}
	\end{sol}
	\pagebreak
	\printbibliography
\end{document}