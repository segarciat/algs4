\documentclass[12pt, a4paper]{article}

\usepackage[utf8]{inputenc}
% Limit the page margin to only 1 inch.
\usepackage[margin=1in]{geometry}

%Imports biblatex package
\usepackage[
backend=biber,
style=alphabetic
]{biblatex}
\addbibresource{../../algs4e.bib}

% Enables the `align' environment.
\usepackage{amsmath}
% Provides useful environments, such as:
% - \begin{proof} ...\end{proof}
\usepackage{amsthm}
\usepackage[most]{tcolorbox}

\newtheorem*{proposition}{Proposition}

% Enables using \mathbb{}, for example \mathbb{N} for the set of natural numbers.
\usepackage{amssymb}

% Allows using letters in enumerate list environment. Use, for example:
%\begin{enumerate}[label=(\alph*)]
% ...
%\end{enumerate}
\usepackage[inline]{enumitem}

% Enable importing external graphic files and provides useful commannds, like \graphicspath{}
\usepackage{graphicx}
% Images are located in a directory called images in the current directory.
\graphicspath{{./images/}}

% Make links look better by default.
% See: https://tex.stackexchange.com/questions/823/remove-ugly-borders-around-clickable-cross-references-and-hyperlinks
\usepackage[hidelinks]{hyperref}
\usepackage{xcolor}
\hypersetup{
	colorlinks,
	linkcolor={red!50!black},
	citecolor={blue!50!black},
	urlcolor={blue!80!black}
}


% Code Listings. Source:
% https://stackoverflow.com/questions/3175105/inserting-code-in-this-latex-document-with-indentation
\usepackage{listings}
\usepackage{color}

\definecolor{dkgreen}{rgb}{0,0.6,0}
\definecolor{gray}{rgb}{0.5,0.5,0.5}
\definecolor{mauve}{rgb}{0.58,0,0.82}

\lstset{frame=tb,
	language=Java,
	aboveskip=3mm,
	belowskip=3mm,
	showstringspaces=false,
	columns=flexible,
	basicstyle={\small\ttfamily},
	numbers=none,
	numberstyle=\tiny\color{gray},
	keywordstyle=\color{blue},
	commentstyle=\color{dkgreen},
	stringstyle=\color{mauve},
	breaklines=true,
	breakatwhitespace=true,
	tabsize=3
}

\newcommand{\prob}{\text{P}}
%\newcommand{\complement}{\mathsf{c}}

% Define an environment called "ex" (for Exercise) so that I can do: \begin{ex}{1.5}...\end{ex}
\newenvironment{ex}[2][Exercise]
{\par\medskip\noindent \textbf{#1 #2.}}
{\medskip}

% Define a solution environment, similar to ex (exercise) environment.
\newenvironment{sol}[1][Solution]
{\par\medskip\noindent \textbf{#1.} }
{\medskip}

\begin{document}
	\noindent Sergio E. Garcia Tapia \hfill
	
	\noindent \emph{Algorithms} by Sedgewick and Wayne (4th edition) \cite{sedgewick_wayne}\hfill
	
	\noindent January 13, 2025\hfill 
	\section*{4.2: Directed Graphs}
	\begin{ex}{1}
		What is the maximum number of edges in a digraph with $V$ vertices and no parallel edges?
		What is the minimum number of edges in a digraph with $V$ vertices, none of which are isolated?
	\end{ex}
	\begin{sol}
		Suppose parallel edges are not allowed, but self-loops are. If there are $V$
		vertices, and $v_i$ is a given vertex, then there are $|V|$ possible edges
		incident from $v_i$, including the self-loop $\texttt{v}_i\texttt{->}\texttt{v}_i$.
		If we do this for $i=1,\ldots,|V|$, then we see that there are $|V|^2$ possible
		edges.
		
		Now suppose that parallel edges and self-loops are allowed, but we care for
		the case where no vertices are isolated. This means that the outdegree
		of the vertex cannot be 0 (what about indegree)? If there are $v_1,\ldots,v_n$
		vertices, where $n=|V|$, then we can arrange for exactly one edge to leave
		each vertex, namely $\text{v}_i\text{->}\text{v}_{i+1}$, for $1\leq i<n$. Then,
		we add one more edge $\text{v}_n\text{->}\text{v}_i$, for example. Thus we have a
		minimum of $|V|$ edges.
	\end{sol}
	\begin{ex}{2}
		Draw, in the style of the figure in the text (page 524), the adjacency lists
		built by \texttt{Digraph}'s input stream constructor for the file
		\texttt{tinyDGex2.txt} (see Figure~\ref{fig:tinyGex2}).
		\begin{lstlisting}[language={}]
12
16
 8  4
 2  3
 0  5
 0  6
 3  6
10  3
 7 11
 7  8
11  8
 2  0
 6  2
 5  2
 5 10
 3 10
 8  1
 4  1
		\end{lstlisting}
		\begin{figure}
			\centering
			\includegraphics[width=0.2\textwidth]{exercise-02-tinyGex2-graph}
			\caption{Digraph formed by using the input stream constructor to \texttt{Digraph}
			with \texttt{tinyGex2.txt}.}
			\label{fig:tinyGex2}
		\end{figure}
	\end{ex}
	\begin{sol}
		See Figure~\ref{fig:ex-02}.
		\begin{figure}
			\centering
			\includegraphics[width=0.6\textwidth]{exercise-02-adjacency-list}
			\caption{Adjacency list for \texttt{Digraph} built from \texttt{tinyGex2.txt}.}
			\label{fig:ex-02}
		\end{figure}
	\end{sol}
	\begin{ex}{3}
		Create a copy constructor for \texttt{Digraph} that takes as input a digraph
		\texttt{G} and creates and initializes a copy of the digraph. Any changes
		a client makes to \texttt{G} should not affect the newly created digraph.
	\end{ex}
	\begin{sol}
		See \texttt{com.segarciat.algs4.ch4.sec2.ex03}.
	\end{sol}
	\begin{ex}{4}
		Add a method \texttt{hasEdge()} to \texttt{Digraph} which takes two \texttt{int}
		arguments \texttt{v} and \texttt{w} and returns \texttt{true} if the graph has
		an edge \texttt{v->w}, \texttt{false} otherwise.
	\end{ex}
	\begin{sol}
		See \texttt{com.segarciat.algs4.ch4.sec2.ex04}.
	\end{sol}
	\begin{ex}{5}
		Modify \texttt{Digraph} to disallow parallel edges and self-loops.
	\end{ex}
	\begin{sol}
		See \texttt{com.segarciat.algs4.ch4.sec2.ex05}.
	\end{sol}
	\begin{ex}{6}
		Develop a test client for \texttt{Digraph}.
	\end{ex}
	\begin{sol}
		See \texttt{com.segarciat.algs4.ch4.sec2.ex06}.
	\end{sol}
	\begin{ex}{7}
		The \emph{indegree} of a vertex in a digraph is the number of directed edges
		that point to that vertex. The \emph{outdegree} of vertex in a digraph is the
		number of directed edges that emanate from that vertex. No vertex is
		reachable from a vertex of outdegree 0, which is called a \emph{sink}; a
		vertex of indegree 0, which is called a \emph{source}, is not reachable from
		any other vertex. A digraph where self-loops are allowed \emph{and} every
		vertex has outdegree 1 is called a \emph{map} (a function from the set of
		integers from $0$ to $V-1$ onto itself). Write a program \texttt{Degrees.java}.
		that implements the following API:
		\begin{lstlisting}[language=java]
		  public class Degrees
	               int Degrees(Digraph G)    // constructor
	               int indegree(int v)       // indegree of v 
	               int outdegree(int v)      // outdegree of v
	Iterable<Integer> sources()              // sources
	Iterable<Integer> sinks()                // sinks
	           boolean isMap()               // is G a map?
		\end{lstlisting}
	\end{ex}
	\begin{sol}
		See \texttt{com.segarciat.algs4.ch4.sec2.ex07}.
	\end{sol}
	\begin{ex}{9}
		Write a method that checks whether a given permutation of a DAG's vertices
		is a topological order of that DAG.
	\end{ex}
		\begin{sol}
		See \texttt{com.segarciat.algs4.ch4.sec2.ex09}.
	\end{sol}
	\begin{ex}{10}
		Given a DAG, does there exist a topological order that cannot result from
		applying a DFS-based algorithm, no matter in what order the vertices adjacent
		to each vertex are chosen? Prove your answer.
	\end{ex}
	\begin{sol}
		No, such a topological order does not exist. It is possible to obtain any
		topological order by using a DFS-based algorithm.
		\begin{proof}
			Let $G$ be a DAG of $V$ vertices, $n=V$, and $\sigma$ be a topological order on $G$.
			If $\sigma_k$ is the $k$th vertex in the order, then $\sigma_n$ must be a sink. Otherwise,
			a vertex would follow it, and we would not have a topological order. If we
			apply DFS to $\sigma_n$, it would return immediately.
			When considering $\sigma_i$, where $i<n$, all vertices that come after $\sigma_i$
			have been marked. Once again, either $\sigma_i$ is a sink (and DFS immediately
			returns), or it points to a vertex that has already been marked. In either case,
			the result is that the vertex is done being processed. We continue this way
			until reaching $i=1$, at which point our DFS-based algorithm ends. Along the way,
			vertices were done in reverse order of $\sigma$, and hence the algorithm
			computes the topological order to be the reverse order of the reverse order
			of $\sigma$, which of course is $\sigma$ itself.
		\end{proof}
	\end{sol}
	
	\begin{ex}{11}
		Describe a family of sparse digraphs whose number of directed cycles grows
		exponentially in the number of vertices.
	\end{ex}
	\begin{sol}
		Digraphs with a center vertex. For example, consider a graph $G$ of $n+1$
		vertices, where for each \texttt{k} there is an edge \texttt{k->0}
		and an edge \texttt{0->k}, where $1\leq k\leq n$. Also, for $k$ and
		$k+1$, there is a pair of edges \texttt{k->(k+1)} and \texttt{(k+1)->k},
		for $k>1$, and for $k=n$, we have \texttt{k->1} and \texttt{1->k}.
		Then $0$ is the center vertex of such a graph. Such a graph is sparse
		because there are $6\cdot n$ edges when there are $n+1$ vertices.
		See Figure~\ref{fig:ex-11}.
		
		\begin{figure}
			\centering
			\includegraphics[width=0.2\textwidth]{exercise-11}
			\caption{Digraph for Exercise 11.}
			\label{fig:ex-11}
		\end{figure}
		
		Such a graph is strongly connected. Given subset of the vertices that contains
		$0$, we can create a directed cycle containing $0$. If $S_0$ is the set of
		vertices without $0$, and $\mathcal{P}(S_0)$ is the power set of $S_0$,
		$|\mathcal{P}(S_0)|=2^n$. By appending $0$ to each set, we have at least
		one unique cycle for each set in $\mathcal{P}(S_0)$, meaning at least $2^n$
		cycles.
	\end{sol}
	\begin{ex}{12}
		Prove that the strong components in $G^R$ are the same as in $G$.
	\end{ex}
	\begin{sol}
		\begin{proof}
			Suppose that $C$ is a strong component of $G$, and let $u,v\in C$.
			Then there is a path $p_{uv}$ from $u$ to $v$ and there is a path $p_{vu}$
			from $v$ to $u$. In $G^R$, edges change direction, so the edges in
			the path $p_{uv}$ and reverse to become path $p_{uv}^R$, which is
			now a path from $v$ to $u$. Similarly, $p_{vu}^R$ is a path from $u$
			to $v$. Hence, $u$ and $v$ belong to the same strong component in $G^R$.
			Thus, if $C^R$ is the strong component in $G^R$ that $u$ and $v$ belong
			to, we see that $C\subseteq C^R$.
			
			Now suppose that $w\notin C$, but $w$ belongs to the same component
			as $u$ and $v$ in $G^R$ (that is, $u,v,w\in C^R$). Then, without loss of
			generality, there is a path
			$p_{vw}^R$ from $v$ to $w$ and a path $p_{wv}^R$ from $w$ to $v$ in $G^R$.
			If we reverse the edges of $G^R$ to obtain $(G^R)^R=G$, then the edges
			in both paths are reversed, and we obtain a path $(p_{vw}^R)^R$, from $w$
			to $v$ and a path $(p_{wv}^R)^R$ from $v$ to $w$ in $G$. This implies that
			$w$ and $v$ are in the same strong component, which contradicts the definition
			of $w$. Hence, $w\notin C^R$.
			
			We've just argued that if $w\notin C$, then $w\notin C^R$, which implies that
			$C^R\subseteq C$. We conclude $C=C^R$.
		\end{proof}
	\end{sol}
	\begin{ex}{13}
		Prove that two vertices in a digraph $G$ are in the same strong component
		if and only if there is a directed cycle (not necessarily simple) containing
		them both.
	\end{ex}
	\begin{sol}
		\begin{proof}
			Let $v,w\in G$.
			
			Suppose that $v$ and $w$ belong to the same strong component. Then $w$
			is reachable from $v$ through a directed path $p_{vw}$ and $v$ is reachable
			from $w$ through a directed path $p_{wv}$. By concatenating the paths, we
			create a cycle that contains both $v$ and $w$.
			
			Now suppose that there is a cycle $u_1,u_2,\ldots,u_n,u_1$ containing
			$v$ an $w$. Suppose $v=u_i$ and $w=u_j$, where $j>i$. Then there is a path
			$v_iv_{i+1}\cdots v_{j-1}v_j$ from $v_i$ to $v_j$ and a path
			$v_jv_{j+1}\cdots v_nv_1\cdot v_i$ from $v_j$ to $v_i$. Hence,
			$v$ and $w$ are strongly connected.
		\end{proof}
	\end{sol}
	\begin{ex}{14}
		Let $C$ be a strong component in a digraph $G$ and let $v$ be any vertex not in $C$.
		Prove that if there is an edge $e$ leaving $v$ to any vertex in $C$, then
		vertex $v$ appears before \emph{every} vertex in $C$ in the reverse post order
		of $G$.
	\end{ex}
	\begin{sol}
		\begin{proof}
			Suppose that $u$ is any vertex in $C$. If \texttt{dfs(u)} is called \emph{before}
			\texttt{dfs(v)}, then every vertex in $C$ will be marked and added
			to the reverse post order because $C$ is a strong component. However,
			$v$ will not be added, because if it did, that would imply a path
			from $u$ to $v$. Together with each $e$ from $v$ to a component in $C$,
			this would imply that $u$ and $v$ are strongly connected, which would
			contradict the fact that $v\notin C$. The conclusion is that all
			vertices in $C$ are pushed onto the stack prior to $v$.
			
			On the other hand, if \texttt{dfs(v)} is called \emph{before} \texttt{dfs(u)},
			then since there is an edge $e$ from $v$ to a component in $C$, this will result
			in all components in $C$ being added to the stack before $v$. Once again,
			$v$ is added to the stack after all vertices in $C$, and hence appears
			\emph{before} every vertex in $C$ in the reverse postorder of $G$.
		\end{proof}
	\end{sol}
	\begin{ex}{15}
		Let $C$ be a strong component in a digraph $G$ and let \texttt{v} be any vertex not in $C$.
		Prove that if there is an edge $e$ leaving any vertex in $C$ to $v$, then vertex
		\texttt{v} appears before \emph{every} vertex in $C$ in the reverse postorder of $G^R$.
	\end{ex}
	\begin{sol}
		\begin{proof}
			Consider the reverse graph $G^R$ of $G$. By \textbf{Exercise 4.2.12}, $C$ is also a strong
			component of $G^R$. Since $e$ is an edge in $G$ leaving $C$ and incident to \texttt{v},
			its reverse $e^R$ is an edge of $G^R$ leaving \texttt{v} incident to a vertex
			in $C$. By \textbf{Exercise 4.2.14} applied to $G^R$, we conclude that \texttt{v} appears
			before every vertex in $C$ in the reverse post order of $G^R$.
		\end{proof}
	\end{sol}
	\begin{ex}{16}
		Given a digraph $G$, prove that the first vertex in the reverse postorder of $G$
		is in a strong component that is a \emph{source} of $G$'s kernel DAG. Then, prove
		that the first vertex in the reverse postorder of $G^R$ is in a strong component
		that is a \emph{sink} of $G$'s kernel DAG. \emph{Hint}: Apply \textbf{Exercises 4.2.14}
		and \textbf{4.2.15}.
	\end{ex}
	\begin{sol}
		\begin{proof}
			Let $v$ be the first vertex in the reverse postorder of $G$, and let
			$C$ be the strong component of $G$ that $v$ belongs to. Suppose that $C$
			is not a source of the kernel DAG of $G$. Then there is a component $C'$
			distinct from $C$ and at least one edge from that component to $C$.
			In particular, there is a vertex $w\in C'$ that points to a vertex in $C$.
			By \textbf{Exercise 4.2.14}, $w$ must appears before every vertex in $C$
			in the reverse post order of $G$, including before $v$, contradicting the
			definition of $v$. This implies that $C'$ does not exist, and hence,
			that $C$ is indeed a source of the kernel DAG of $G$.
			
			Now let $u$ be the first vertex in the reverse post order of $G^R$,
			and let $D$ be the strong component that it belongs to. If $D$ is
			not a sink of the kernel DAG of $G$, then there is a strong component
			$D'$ of $G$ (and $G^R$) and edge $f$ from $D$ to $D'$. If $z$
			is the tail of $f$, the $z\in D'$. By \textbf{Exercise 4.2.15},
			$z$ must appear before every vertex in $D$ in the reverse post order
			of $G^R$, contradicting the definition of $u$. This implies that
			$D'$ does not exist, and hence, that $D$ is indeed a sink of the
			kernel DAG of $G$.
		\end{proof}
	\end{sol}
	\begin{ex}{17}
		How many strong components are there in the digraph on page 591?
	\end{ex}
	\begin{sol}
		See \texttt{com.segarciat.algs4.ch4.sec2.ex17}.
	\end{sol}
	\begin{ex}{18}
		What are the strong components of a DAG?
	\end{ex}
	\begin{sol}
		The strong components of a DAG are the singleton vertices of the DAG.
		Indeed, by definition, a vertex is reachable from itself, so each
		strong component must have at least 1 vertex. On the other hand, if
		any component had 2 or more vertices, call them $v$ and $w$, then
		there would be a cycle involving $v$ and $w$. This cannot happen
		in a DAG, so no such component can exist.
	\end{sol}
	\begin{ex}{19}
		What happens if you run the Kosaraju-Sharir algorithm on a DAG?
	\end{ex}
	\begin{sol}
		Assuming a total of $V$ vertices, it will identify a total of $V$ singleton strong
		components. 
	\end{sol}
	\begin{ex}{20}
		True or false: The reverse postorder of a digraph's reverse is the same as the
		postorder of the digraph.
	\end{ex}
	\begin{sol}
		False. Consider digraph $G$ with vertices \texttt{1}, \texttt{2}, and \texttt{3},
		and the single edge \texttt{1->2}. Its reverse $G^R$ has the single edge
		\texttt{2->1}.
		
		The algorithm for the reverse postorder (applied to $G^R$) starts with \texttt{dfs(1)},
		which is immediately added to the stack, followed by \texttt{2}, and followed by
		\texttt{3}. Hence, the algorithm computes the reverse postorder \texttt{3},
		\texttt{2}, \texttt{1} for $G$.
		
		Applying the algorithm to $G$, we starts with \texttt{dfs(1)}, which causes
		a call to \texttt{dfs(2)}, which returns and is added to the queue, followed by
		\texttt{1}. Since \texttt{2} is marked, it is skipped, and then \texttt{dfs(3)}
		is called, causing \texttt{3} to be added to the queue. Now the queue contains
		the postorder \texttt{2}, \texttt{1}, \texttt{3} of $G$, which is distinct from
		the reverse postorder of $G^R$.
	\end{sol}
	\begin{ex}{21}
		True or false: If we consider the vertices of a digraph $G$ (or its reverse)
		in postorder, then vertices in the same strong component will be consecutive in
		that order.
	\end{ex}
	\begin{sol}
		False. See Figure~\ref{fig:ex-21}.
		\begin{figure}
			\centering
			\includegraphics[width=0.2\textwidth]{exercise-21}
			\caption{Digraph for Exercise 21. The postorder of $G$ is not
			the same as the reverse postorder of $G^R$.}
			\label{fig:ex-21}
		\end{figure}
		The call \texttt{dfs(0)} will result in the postorder \texttt{1}, \texttt{3},
		\texttt{2}, and \texttt{0} of $G$. However, \texttt{1} and \texttt{0} are in
		the same component, and they do not appear consecutively.
	\end{sol}
	\begin{ex}{22}
		True or false: If we modify the Kosaraju-Sharir algorithm to run the depth-first
		search in the digraph $G$ (instead of the reverse digraph $G^R$) and the second
		depth-first search in $G^R$ (instead of $G$), then it will still find the strong
		components.
	\end{ex}
	\begin{sol}
		True.
		\begin{proof}
			Recall that the first vertex in the reverse postorder of $G$ is in a source
			of the kernel DAG of $G$. However, it is in a sink component of the kernel DAG
			of $G^R$. If we let $H=G^R$, and $H^R=G$, then the algorithm computes
			the reverse postorder of $H^R$, and then uses DFS to visit the vertices
			of $H$ in that order, which is precisely applying the Kosaraju-Sharir
			algorithm to $H$. By \textbf{Proposition H}, the algorithm identifies the
			strong components of $H=G^R$. By \textbf{Exercise 4.2.12}, the strong
			components of $G^R$ are precisely the same as the strong components of $G$.
		\end{proof}
	\end{sol}
	\begin{ex}{23}
		True or false: If we modify the Kosaraju-Sharir algorithm to replace the second
		depth-first search with breadth-first search, then it will still the strong
		components.
	\end{ex}
	\begin{sol}
		True. Recall that the first vertex in the reverse postorder of $G^R$ appears
		in a sink of the kernel DAG of $G$ (see \textbf{Exercise 4.2.16}). When
		BFS is used with that first vertex as the source, all vertices in that
		component are visited (by definition of strong connectivity), and no
		other vertices. Suppose that a vertex $v$ that did not belong to the
		sink is visited. This would imply that an edge from a sink of the
		kernel DAG to the component containing $v$, contradicting the definition
		of a sink. Now that all vertices in that sink component are visited,
		they are effectively removed, and the algorithm continues in the
		same way in the remainder of the digraph.
		
		The only thing that changes is the order in which the vertices on a given
		component are visited.
	\end{sol}
	\begin{ex}{24}
		Compute the memory usage of a \texttt{Digraph} with $V$ vertices and $E$ edges,
		under the memory cost model of Section 1.4.
	\end{ex}
	\begin{sol}
		First, there are 16 bytes of object overhead for \texttt{Digraph}. The class
		needs 4 bytes for its \texttt{int E} instance variable, and 4 bytes for its
		\texttt{int V} instance variable. There is also 8 bytes for its reference
		to the \texttt{adj} array that contains the adjacency lists of the vertices
		of the digraph. The array itself takes up 24 bytes. So far, that's a flat
		cost of $56$ bytes.
		
		The \texttt{adj} array it has $V$ references to \texttt{Bag} objects, and
		each reference is $8$ bytes. Each \texttt{Bag}
		contains the adjacency list of a given vertex. A \texttt{Bag} object uses
		16 bytes of object overhead, \texttt{8} bytes for its reference to the
		\texttt{first} node, \texttt{4} bytes for its to the \texttt{int size}
		instance variable, and 4 bytes of padding for a total of 32 bytes.
		Thus, that's $40V$ bytes for the \texttt{Bag} objects themselves.
		
		Now, there are $E$ edges, and each is represented by a single \texttt{Node}
		object. A \texttt{Node} uses 40 bytes, and since it points to \texttt{Integer}
		objects, which take up 24 bytes, the overall cost is \texttt{64} bytes
		for an edge. Hence, we require $64E$ bytes for all edges.
		
		The overall cost of \texttt{Digraph} is $56+40V+64E$ bytes.
	\end{sol}
	\begin{ex}{25}
		How many edges are there in the transitive closure of a digraph that
		is a simple directed path with $V$ vertices and $V-1$ edges?
	\end{ex}
	\begin{sol}
		A simple directed path has no repeated edges and no repeated vertices.
		Let $n=V$, and suppose the simple path is $\texttt{v}_1\texttt{v}_2\cdots\texttt{v}_n$.
		Note that $\texttt{v}_n$ is reachable from all previous vertices, so
		the transitive closure contains an edge for each of them, a total of
		$n-1$ edges. Similarly, the transitive closure has $n-2$ edges incident to
		$\texttt{v}_{n-1}$, and more generally, the transitive closure has
		$i-1$ edges incident to $v_i$, where $1\leq i\leq n$. Thus, we sum:
		\begin{align*}
			\sum_{i=1}^{n}(i-1)=\sum_{1\leq i\leq n}(i-1)=\sum_{0\leq i-1\leq n}(i-1)=\frac{(n-1)n}{2}
		\end{align*}
		Hence, the transitive closure has $(V-1)V/2$ edges
	\end{sol}
	
	\pagebreak
	\printbibliography
\end{document}